\begin{remarkBox}{Local Properties}
    Usually, one says that a space \( X \) satisfies a given property "locally" if 
    every \( x \in X \) has "arbitrarily small" neighborhoods having the given property.
\end{remarkBox}

\begin{remarkBox}{Local Compactness vs Compactness}
    In this section, we introduce another commonly used formulation of 
    compactness.
    We note that it is weaker in general than compactness, though it 
    coincides with compactness for metrizable spaces (as shown in 
    [\hyperlink{thm:28.2}{Theorem 28.2}]).
\end{remarkBox}

\begin{remarkBox}{Hausdorff One-Point Compactifications}
    Hausdorff one-point compactifications are unique up to homeomorphism, 
    provided that they exist in the first place.
\end{remarkBox}

\begin{remarkBox}{On [\hyperlink{thm:29.2}{Theorem 29.2}]}
    This theorem gives us equivalent definition of what it means for a space to be 
    locally compact -- in fact, this is the definition that is given in lecture, where 
    \( \mathrm{Cl} \ V \) is our compact neighborhood in question in the case when 
    \( X \) is Hausdorff.
\end{remarkBox}

\begin{remarkBox}{Difference between Definitions}
    One key thing to keep in mind is that the definition given in lecture is only 
    equivalent to the definition that Munkres provides when we are given \( X \) to 
    be Hausdorff. 
    However, the definition given in lecture does not require the space to be 
    Hausdorff.

    \baseSkip 

    For all the proofs in the variations 
    (after [\hyperlink{cor:29.4}{Corollary 29.4}]), we used the definition of local 
    compactness given in lecture -- not the definition given in Munkres. 
    However, when the space is Hausdorff, it does not matter which definition that we 
    choose to use.
\end{remarkBox}

\begin{remarkBox}{On [\hyperlink{cor:29.3}{Corollary 29.3}]}
    We saw that any \textit{closed} subspace of a locally compact space is 
    locally compact as well. 
    However, in order to say the same for \textit{open} subspaces, we need that the 
    larger space to be Hausdorff as well in order to say that such open subspaces 
    are locally compact.
\end{remarkBox}

\begin{remarkBox}{More Examples and Proofs}
    The proofs as well as more examples can be found in \S 29 of Munkres.
\end{remarkBox}