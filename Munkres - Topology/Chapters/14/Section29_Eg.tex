\begin{egBox}{Compactification}[eg:29.1]
    Sadly, many reasonable spaces are not compact.
    However, most reasonable spaces can be compactified!
    We consider the open interval \( ( 0, 1 ) \) and show intuitively how to 
    add a \textit{single} point in \( ( 0, 1 ) \) so that it becomes compact!

    \baseSkip

    We can turn \( ( 0, 1 ) \) into a compact subspace of \( \mathbb{R} \) by
    adding on the endpoints \( 0 \) and \( 1 \).
    However, it is possible to turn \( ( 0, 1 ) \) into a compact subspace
    using only one point; notice that \( ( 0, 1 ) \) is homeomorphic to 
    a circle minus its north pole.
    If we were to add on the north pole, then we get a full circle, which we
    know is compact.
    In this case, the north pole was our distinguished point \( \star \).
\end{egBox}

\begin{egBox}{Locally Compact}[eg:29.2]
    \( \mathbb{R}^{ n } \) is locally compact.

    \baseRule

    Let \( K = \{ 1, \frac{ 1 }{ 2 }, \frac{ 1 }{ 3 } , \ldots \} \).
    We show that there is no compact neighborhood of \( 0 \) in 
    \( \mathbb{R} \setminus K \) with the standard topology.
    Let \( A \subset \mathbb{R} \setminus K \) contain an open neighborhood
    of \( 0 \) in \( \mathbb{R} \setminus K \).
    Then, we see that \( A \) contains one of the maximal open intervals,
    but not its endpoints.
    \( A \) is not closed in \( \mathbb{R} \).
    The Heine-Borel theorem tells us that \( A \) is not compact in 
    \( \mathbb{R} \).
    Hence, we see that \( A \) is not compact in \( R \setminus A \) since 
    compactness is an inherent property.
\end{egBox}

\begin{egBox}{Open Disk in \( \mathbb{R}^{ 2 } \)}[eg:29.3]
    Is a Hausdorff one-point compactification of the open disk
    \( x^{ 2 } + y^{ 2 } < 1 \) with the standard topology guaranteed to exist?
    If so, can you surmise what it is?

    \baseSkip

    Notice that the open disk is not compact since it is not closed in 
    \( \mathbb{R}^{ 2 } \) (Heine-Borel).
    We start by noting that \( \mathbb{R}^{ 2 } \) under the standard topology
    is locally compact.
    Since the open disk is an open subspace of \( \mathbb{R}^{ 2 } \), we see 
    that the open disk must also be locally compact.
    Furthermore, we have that the open disk is Hausdorff since 
    \( \mathbb{R}^{ 2 } \) is Hausdorff.
    Since the open disk is locally compact and Hausdorff, we have that it must
    have a Hausdorff one-point compactification.

    \baseSkip

    To surmise what it is, we start with a previous example.
    For any open interval in \( \mathbb{R} \), we found that its one-point
    compactification resulted to be a circle; recall that any open interval
    in \( \mathbb{R} \) is homeomorphic to a circle minus its north pole 
    (i.e., \( S^{ 1 } \setminus \{ N \} \)).
    In a similar fashion we see that any open disk in \( \mathbb{R}^{ 2 } \)
    is homeomorphic to a sphere in \( \mathbb{R}^{ 3 } \) minus its north pole 
    (i.e. \( S^{ 2 } \setminus \{ N \} \)).
    Thus, the one-point compactification of any open disk in \( R^{ 2 } \)
    is a sphere in \( \mathbb{R}^{ 3 } \) (that is, \( S^{ 2 } \)), where the
    distinguished point is the north pole.

    \baseSkip

    In fact, the one-point compactification of \( \mathbb{R} \) is \( S^{ 1 } \)
    since we have that \( \mathbb{R} \cong ( 0, 1 ) \).
\end{egBox}

\begin{egBox}{Compactness implies Local Compactness}[eg:29.4]
    We first remark that this implication holds for the definition of local 
    compactness given by Munkres -- not in lecture!
    To make this implication work with the definition given in lecture, we need our 
    space to be Hausdorff as well.

    \baseSkip 

    Let's say that \( X \) is a compact topological space. 
    Our goal is to show that \( X \) is locally compact. 
    To do so, we just need to show that for any given point \( x \in X \), there 
    is some compact subspace \( C \) of \( X \) that contains a neighborhood of \( x \).
    Let \( U \) be any open neighborhood of \( x \) in \( X \). 
    We can see that \( \mathrm{Cl} \ U \) is closed in \( X \), which results in 
    \( \mathrm{Cl} \ U \) to be compact in \( X \) as well.
    If we define \( C = \mathrm{Cl} \ U \), then we see that the condition of local 
    compactness is met. 
    Hence \( X \) is locally compact.
\end{egBox}

\begin{egBox}{Discrete Topology is locally compact Hausdorff}[eg:29.5]
    Let \( X \) be under the discrete topology.
    Our goal is to show that \( X \) is locally compact Hausdorff.
    To show that \( X \) is Hausdorff, we start by letting two distinct points 
    \( x, y \in X \) be given. 
    Because we are under the discrete topology, we have that \( \{ x \} \) and 
    \( \{ y \} \) open in \( X \), and we can see that they are disjoint as well. 
    Thus, we get that \( X \) is Hausdorff.

    \baseSkip

    Note that \( X \) is only compact under the discrete topology if and only if it is 
    finite.
    Let's say that \( X \) is compact under the discrete topology.
    If \( X \) were not finite, then the open cover consisting of singleton 
    point sets would have not finite subcover.
    Now, if \( X \) were given to be finite and under the discrete topology, we see 
    that any open cover of \( X \) is finite, which tells us that \( X \) is compact.

    \baseSkip

    With this, we can show that \( X \) is locally compact. 
    To do so, we want to show that for any \( x \in X \), there is some compact 
    subspace \( C \) of \( X \) that contains a neighborhood of \( x \).
    Let \( U \) be any finite set that contains \( x \). 
    We see that \( U \) must be compact in \( X \) by what we found earlier. 
    Furthermore, we see that \( \{ x \} \subset U \) as \( U \) contains \( x \), 
    which tells us that \( U \) is a compact subspace of \( X \) that contains a 
    neighborhood of \( x \).
    Thus, \( X \) is locally compact. 
\end{egBox}