\begin{thmBox}{29.1}[thm:29.1]
    Let \( X \) be a space. 
    Then \( X \) is locally compact Hausdorff if and only if there exists a 
    space \( Y \) satisfying the following conditions:

    \begin{enumerate}
        \item \( X \) is a subspace of \( Y \)
        \item The set \( Y \setminus X \) consists of a single point
        \item \( Y \) is a compact Hausdorff space.
    \end{enumerate}

    If \( Y \) and \( Y' \) are two spaces satisfying these conditions, then 
    there is a homeomorphism of \( Y \) with \( Y' \) that equals the identity
    map on \( X \).

    \baseRule

    \begin{proofBox}
        The full details can be seen in pages \( 183-184 \).
    \end{proofBox}
\end{thmBox}

\begin{thmBox}{29.2}[thm:29.2]
    Let \( X \) be a Hausdorff space.
    Then \( X \) is locally compact if and only if given \( x \in X \), and 
    given a neighborhood \( U \) of \( x \), there is a neighborhood 
    \( V \) of \( x \) such that \( \mathrm{Cl} \ V \) is compact and
    \( \mathrm{Cl} \ V \subset U \).

    \baseRule

    \begin{proofBox}*
        \wrapBox{\( \implies \)}

        Let \( X \) be locally compact.
        Let \( x \in X \) and a neighborhood \( U \) of \( x \) in \( X \) be given.
        We want to show that there is a neighborhood \( V \) of \( x \) such that 
        \( \mathrm{Cl} \ V \) is compact and \( \mathrm{Cl} \ V \subset U \).
        
        \baseSkip 

        Since we have that \( X \) is locally compact Hausdorff, it follows 
        from [\hyperlink{thm:29.1}{Theorem 29.1}] that there exists a Hausdorff 
        one-point compactification \( Y \) of \( X \).
        We shall define \( C \) to be the set \( Y \setminus U \); it follows 
        immediately that \( C \) is closed in \( Y \).
        By [\hyperlink{thm:26.2}{Theorem 26.2}], it follows that \( C \) is compact 
        in \( Y \) as well. 
        We know from [\hyperlink{lem:26.4}{Lemma 26.4}] that there exists disjoint open 
        sets \( V \) and \( W \) such that they contain \( x \) and \( C \), 
        respectively. 

        \baseSkip 

        With this, we see that the closure \( \mathrm{Cl} \ V \) of \( V \) in \( Y \)
        is compact (as it is closed in \( Y \)).
        Furthermore, we see that \( \mathrm{Cl} \ V \) is disjoint from \( W \).
        To show why, let's say that \( w \in W \) is given.
        Because \( W \) is an open neighborhood of \( w \) that does not intersect 
        \( V \), we see that \( w \) cannot be a limit point of \( V \). 
        Hence, we find that no point \( w \in W \) can be a limit point of \( V \),
        which tells us that \( W \cap \mathrm{Cl} \ V = \emptyset \).
        From this, it immediately follows that \( \mathrm{Cl} \ V \) and \( C \) 
        are disjoint from each other as \( C \subset W \). 
        Thus, we have that for each \( x \in \mathrm{Cl} \ V \), it follows that 
        \( x \notin C \), which implies that \( x \in U \).
        Thus, \( \mathrm{Cl} \ V \subset U \) as intended. 

        \baseSkip 
        \wrapBox{\( \impliedby \)}

        Conversely, let's say that \( x \in X \) and an open neighborhood \( U \) of 
        \( x \) is given.
        Furthermore, we are also given that there exists a neighborhood \( V \) of 
        \( x \) such that \( \mathrm{Cl} \ V \) is compact and 
        \( \mathrm{Cl} \ V \subset U \).
        If we define \( C = \mathrm{Cl} \ V \), then we see that it is a compact 
        set containing a neighborhood of \( x \). 
        Since this is true for all \( x \in X \), we see that \( X \) is locally 
        compact. 
    \end{proofBox}
\end{thmBox}

\begin{thmBox}[Corollary]{29.3}[cor:29.3]
    Let \( X \) be locally compact Hausdorff; let \( A \) be a subspace of 
    \( X \). If \( A \) is closed in \( X \) or open in \( X \), then 
    \( A \) is locally compact.

    \baseRule

    \begin{proofBox}
        Suppose that \( A \) is closed in \( X \).
        We want to show that for any \( x \in A \), we can find some compact subspace 
        of \( A \) that contains a neighborhood of \( x \) in \( A \).
        Since \( X \) is locally compact, we see that there exists some compact
        subspace \( C \) of \( X \) that contains some neighborhood \( U \) of \( x \)
        in \( X \).
        Under the subspace topology, we can see that \( C \cap A \) is closed in \( C \)
        since we are given that \( A \) is closed in \( X \). 
        Hence, it follows that \( C \cap A \) must also be compact as \( C \) is 
        compact.
        Furthermore, we can see that \( C \cap A \) contains the open neighborhood
        \( U \cap A \) of \( x \) in \( A \).
        Thus, we have that \( C \cap A \) is a compact subspace of \( A \) that 
        contains a neighborhood of \( x \) in \( A \), which tells us that 
        \( A \) is locally compact. 
        Notice that we have not used the Hausdorffness condition here. 

        \baseSkip

        Suppose now that \( A \) is open in \( X \). 
        Given \( x \in A \), we apply [\hyperlink{thm:29.2}{Theorem 29.2}] to choose 
        a neighborhood \( V \) of \( x \) in \( X \) such that \( \mathrm{Cl} \ V \) is 
        compact and \( \mathrm{Cl} \ V \subset A \). 
        With this, we see that \( C = \mathrm{Cl} \ V \) is a compact subspace of
        \( A \) containing the neighborhood \( V \) of \( x \) in \( A \). 
        Thus, \( A \) is locally compact.
    \end{proofBox}
\end{thmBox}

\begin{thmBox}[Corollary]{29.4}[cor:29.4]
    A space \( X \) is homeomorphic to an open subspace of a compact Hausdorff 
    space if and only if \( X \) is locally compact Hausdorff.

    \baseRule

    \begin{proofBox}
        This follows from [\hyperlink{thm:29.1}{Theorem 29.1}] and 
        [\hyperlink{cor:29.3}{Corollary 29.3}]
    \end{proofBox}
\end{thmBox}

\begin{thmBox}{Hausdorff One-Point Compactification}[thm:29.3]
    Let \( X \) be a non-compact topological space having \textit{Hausdorff}
    one-point compactifications \( Y_{ 1 } \) and \( Y_{ 2 } \) with 
    distinguished points \( \star_{ 1 } \in Y_{ 1 } \) and \( \star_{ 2 } \in
    Y_{ 2 } \).

    \baseSkip

    Then \( Y_{ 1 } \) is homeomorphic to \( Y_{ 2 } \).

    \baseRule

    \begin{proofBox}
        We start by noting that because \( Y_{ 1 } \) and \( Y_{ 2 } \) are 
        one-point compactifications of \( X \) with distinguished points 
        \( \star_{ 1 } \) and \( \star_{ 2 } \), respectively, it follows that
        \begin{equation*}
            Y_{ 1 } \setminus \{ \star_{ 1 } \} 
            \cong 
            X
            \quad \mathrm{and} \quad 
            Y_{ 2 } \setminus \{ \star_{ 2 } \} 
            \cong 
            X
        \end{equation*}
        Since \( \cong \) is an equivalence relation, we have that it must be 
        transitive. 
        Thus, we see that 
        \begin{equation*}
            Y_{ 1 } \setminus \{ \star_{ 1 } \}
            \cong 
            Y_{ 2 } \setminus \{ \star_{ 2 } \}
        \end{equation*}
        which tells us that there exists a homeomorphism 
        \( 
            f: Y_{ 1 } \setminus \{ \star_{ 1 } \} \rightarrow 
            Y_{ 2 } \setminus \{ \star_{ 2 } \} 
        \)
        between them.

        \baseSkip

        Let us construct another function \( g: Y_{ 1 } \rightarrow Y_{ 2 } \),
        where
        \begin{equation*}
            g ( \star_{ 1 } ) = \star_{ 2 }
            \quad \mathrm{and} \quad 
            g ( x ) = f ( x )
            \quad 
            \forall x \in Y_{ 1 } \text{ s.t. } x \neq \star_{ 1 }
        \end{equation*} 
        Our claim is that \( g \) is a homeomorphism.
        Indeed, it is clear to see that \( g \) is bijective since we have 
        constructed it to be so (note that \( f \) is bijective).
        Our goal is to show that both \( g \) and \( g^{ -1 } \) are continuous.
        However, it suffices to only show that \( g \) is continuous since
        its domain is compact and codomain is Hausdorff
        [\hyperlink{thm:26.6}{Theorem 26.6}].
        
        \baseSkip

        Let \( C \subset Y_{ 2 } \) be closed.
        Our goal is to show that \( g^{ -1 } ( C ) \) is closed.
        Because \( Y_{ 2 } \) is compact, \( C \) being closed in \( Y_{ 2 } \) means 
        that it must be compact in \( Y_{ 2 } \) as well.
        We shall now look at two cases:

        \baseSkip
        \wrapBox{Case 1: \( \star_{ 2 } \notin C \)}

        In this case, we see that \( C \subset Y_{ 2 } \setminus \{ \star_{ 2 } \} \).
        By the [\hyperlink{thm:26.1}{variation of Lemma 26.1}], we have that \( C \) is 
        compact in \( Y_{ 2 } \setminus \{ \star_{ 2 } \} \) as well.
        Furthermore, we see that \( g^{ -1 } ( C ) = f^{ -1 } ( C ) \).
        Because 
        \( 
            f^{ -1 }: 
            Y_{ 2 } \setminus \{ \star_{ 2 } \} 
            \rightarrow 
            Y_{ 1 } \setminus \{ \star_{ 1 } \} 
        \) 
        is continuous and \( C \) is compact in 
        \( Y_{ 2 } \setminus \{ \star_{ 2 } \} \), we find that \( f^{ -1 } ( C ) \) is 
        compact in \( Y_{ 1 } \setminus \{ \star_{ 1 } \} \).
        Thus, we get that \( f^{ -1 } ( C ) \) is compact in the larger 
        space \( Y_{ 1 } \) by the [\hyperlink{thm:26.1}{variation of Lemma 26.1}].
        Now since \( Y_{ 1 } \) is Hausdorff, it follows that \( f^{ -1 } ( C ) \) 
        is closed.
        I.e., we see that \( g^{ -1 } ( C ) \) is closed.

        \baseSkip
        \wrapBox{Case 2: \( \star_{ 2 } \in C \)}

        In this case, we see that \( Y_{ 2 } \setminus C \) is contained in 
        \( Y_{ 2 } \setminus \{ \star_{ 2 } \} \).
        Furthermore, we have that it is open in \( Y_{ 2 } \) since \( C \) is closed in \( Y_{ 2 } \), meaning that it is also open in 
        \( Y_{ 2 } \setminus \{ \star_{ 2 } \} \) under the subspace topology.
        Because \( f: Y_{ 1 } \setminus \{ \star_{ 1 } \} \rightarrow 
        Y_{ 2 } \setminus \{ \star_{ 2 } \} \) is continuous, we have that 
        \( f^{ -1 } ( Y_{ 2 } \setminus C ) \) is open in 
        \( Y_{ 1 } \setminus \{ \star_{ 1 } \} \).
        Since \( Y_{ 1 } \) is Hausdorff (hence \( T_{ 1 } \)), we see that 
        all singleton sets are closed.
        Thus, \( Y_{ 1 } \setminus \{ \star_{ 1 } \} \) is open, and by the 
        transitivity of openness, we see that 
        \( f^{ -1 } ( Y_{ 2 } \setminus C ) \) is open in \( Y_{ 1 } \).
        Now because \( Y_{ 2 } \setminus C \) does not contain 
        \( \star_{ 2 } \), we have that \( g^{ -1 } ( Y_{ 2 } \setminus C ) \)
        is equal to \( f^{ -1 } ( Y_{ 2 } \setminus C ) \).
        Notice that 
        \begin{equation*}
            \begin{aligned}
                f^{ -1 } ( Y_{ 2 } \setminus C )
                &=
                g^{ -1 } ( Y_{ 2 } \setminus C )
                \\
                &=
                \{ x \in Y_{ 1 } \setminus \{ \star_{ 1 } \} \mid g ( x ) \in 
                Y_{ 2 } \setminus C \}
                \\
                &=
                \{ x \in Y_{ 1 } \setminus \{ \star_{ 1 } \} \mid g ( x ) 
                \notin C \}
                \\
                &=
                \{ 
                    x \in Y_{ 1 } \setminus \{ \star_{ 1 } \} 
                    \mid x \notin g^{ -1 } ( C ) 
                \}
                \\
                &=
                Y_{ 1 } \setminus g^{ -1 } ( C )
            \end{aligned}
        \end{equation*}
        Notice that \( C \) includes \( \star_{ 2 } \), meaning that 
        \( g^{ -1 } ( C ) \) contains \( \star_{ 1 } \).
        Now, because \( f^{ -1 } ( Y_{ 2 } \setminus C ) \) is open in 
        \( Y_{ 1 } \), it follows that \( g^{ -1 } ( C ) \) must be closed in 
        \( Y_{ 1 } \).

        \baseSkip

        In either case, we see that \( g \) is continuous, which results in \( g \) to
        be a homeomorphism between \( Y_{ 1 } \) and \( Y_{ 2 } \).
    \end{proofBox}
\end{thmBox}

\begin{thmBox}{Variation of [\hyperlink{cor:29.3}{Corollary 29.3}]}[thm:29.4]
    Let \( X \) be locally compact and \( U \) be an open subspace of 
    \( X \).
    Then \( U \) is locally compact.

    \baseSkip 

    If \( X \) is also Hausdorff, then any closed subspaces of \( X \) are locally 
    compact.

    \baseRule

    \begin{proofBox}
        Let \( U \) be an open subspace of \( X \).
        Let \( p \in U \) be given.
        Let \( V \) be an open neighborhood of \( p \) in \( U \).
        We want to show that there exists a compact neighborhood of \( p \)
        that is contained in \( V \).
        Because \( U \) is open in \( X \), we find that \( V \) is also
        open in \( X \) by the transitivity of openness.
        Because \( X \) is locally compact, we see that there exists a compact
        neighborhood \( K \) of \( p \) in \( X \) that is contained in 
        \( V \).

        \baseSkip

        Let \( X \) be also Hausdorff; let \( U \) now be a closed subspace of \( X \).
        We have by [\hyperlink{thm:29.2}{Theorem 29.2}] that the definition of 
        local compactness given by Munkres is equivalent to the definition given in 
        lecture. 
        Thus, we are able to utilize [\hyperlink{cor:29.3}{cor:29.3}] to get that 
        \( U \) is also locally compact. 
    \end{proofBox}
\end{thmBox}

\begin{thmBox}[Lemma]{Variation of [\hyperlink{thm:29.2}{Theorem 29.2}]}[lem:29.5]
    Let \( X \) be a compact Hausdorff topological space.
    Let \( A \) be a closed subset of \( X \) and \( U \) be an open 
    neighborhood of \( A \). 
    Prove that there exists an open neighborhood \( V \) of \( A \) such that 
    \( \mathrm{Cl} \ V \subset U \).


    \baseRule

    \begin{proofBox}
        Since both \( A \) and \( X \setminus U \) are closed and \( X \) is 
        compact, it follows that they are both compact.
        Furthermore, it is clear to see that \( A \) and \( X \setminus U \) 
        are disjoint since \( U \) contains \( A \).
        Using the [\hyperlink{thm:26.4}{Extension of Lemma 26.4}], 
        we find that there exists disjoint open sets 
        \( V \) and \( W \) such that they contain \( A \) and 
        \( X \setminus U \), respectively.
    
        \baseSkip
    
        Our goal is to show that \( \mathrm{Cl} \ V \subset U \). 
        Let \( x \in \mathrm{Cl} \ V \) be given; we want to show that 
        \( x \in U \).
        Towards a contradiction, let's suppose that \( x \in X \setminus U \).
        It follows that \( W \) is then some open neighborhood of \( x \) in 
        \( X \).
        Because \( x \in \mathrm{Cl} \ V \), we see that 
        \( W \cap V \neq \emptyset \).
        However, we reach a contradiction since \( W \) and \( V \) are disjoint.
        Hence, it must be the case that \( x \in U \).
        Since this was true for any arbitrary \( x \in \mathrm{Cl} \ V \), it 
        follows that \( \mathrm{Cl} \ V \subset U \).
    \end{proofBox}
\end{thmBox}

\begin{thmBox}{Another Variation of [\hyperlink{cor:29.3}{Corollary 29.3}]}[thm:29.5]
    Let \( X \) be a compact Hausdorff space.
    Then every open (and closed) subspace of \( X \) is locally compact and Hausdorff.

    \baseRule

    \begin{proofBox}
        We have shown previously that Hausdorffness translates to subspaces in 
        general [\hyperlink{thm:17.11}{Theorem 17.11}].
        We have also shown that local compactness translates to open (and closed) 
        subspaces [\hyperlink{Variation of Corollary 29.3}{thm:29.4}].
        Thus, it suffices to show that \( X \) is locally compact (\( X \) is already Hausdorff)

        \baseSkip

        We start off by letting \( p \in X \) and an open neighborhood 
        \( U \) of \( p \) be given.
        By [\hyperlink{lem:29.5}{Variation of Lemma 29.3}], we see that there exists an open 
        neighborhood \( V \) of \( p \) so that \( \mathrm{Cl} \ V \subset U \).
        Now because \( \mathrm{Cl} \ V \) is closed in \( X \), it follows that
        \( \mathrm{Cl} \ V \) is compact since \( X \) is compact.
        Thus, we have that \( \mathrm{Cl} \ V \) is a compact neighborhood 
        of \( p \) in \( U \).
        Hence, we have that \( X \) is locally compact.
    \end{proofBox}
\end{thmBox}

\begin{thmBox}{Existence of One-Point Compactifications}[thm:29.6]
    If a space \( X \) is locally compact and Hausdorff, then there exists a 
    one-point compactification of \( X \).

    \baseRule

    \begin{proofBox}
        This is a direct result of [\hyperlink{thm:29.1}{Theorem 29.1}]
    \end{proofBox}
\end{thmBox}