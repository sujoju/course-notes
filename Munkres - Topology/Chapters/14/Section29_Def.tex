\begin{defBox}{Locally Compact}[def:29_locally_compact]
    Munkres provides the following definition: A topological space \( X \) is said 
    to be \textbf{locally compact at} \( x \) if there is some compact subspace \( C \)
    of \( X \) that contains a neighborhood of \( x \).
    If \( X \) is locally compact at each of its points, \( X \) is said simply 
    to be \textbf{locally compact}.

    \baseSkip 

    In lecture, though, we are given the following:
    A topological space \( X \) is 
    \textbf{locally compact at a given point} \( p \in X \) if and only 
    if every open neighborhood of \( p \) contains a \textit{compact
    neighborhood} of \( p \), which is a compact set that contains an open 
    neighborhood of \( p \).
    A topological space \( X \) is \textbf{locally compact} if and only
    if it is locally compact at each of its points.
\end{defBox}

\begin{defBox}{One-point Compactification}[def:29_one-point_compactification]
    If \( Y \) is a compact Hausdorff space and \( X \) is a proper 
    subspace of \( Y \) whose closure equals \( Y \), then \( Y \) is said to
    be a \textbf{compactification} of \( X \).

    \baseSkip
    
    A \textbf{one-point compactification} of a non-compact topological space
    \( X \) is any \textit{compact} topological space \( Y \) with a 
    distinguished point \( \star \) so that 
    \( Y \setminus \{ \star \} \cong X \) (Here \( \cong \) means homeomorphic).
    I.e., if \( Y \setminus X \) equals a single point, then \( Y \) is called
    the \textbf{one-point compactification} of \( X \).
\end{defBox}