\begin{egBox}{Euclidean Distance}[eg:metric-1]
    The \textbf{Euclidean Distance} is defined as follows: 
    If \( X = ( x_{ 1 }, \ldots, x_{ n } ) \) and 
    \( Y = ( y_{ 1 }, \ldots, y_{ n } ) \) are points in \( \mathbb{R}^{ n } \)
    then the \textbf{Euclidean distance} between \( X \) and \( Y \) is 
    \begin{equation*}
        d( X, Y )
        =
        \sqrt{ 
            ( x_{ 1 } - y_{ 1 } )^{ 2 } 
            + \ldots + 
            ( x_{ n } - y_{ n } )^{ 2 }
        }
    \end{equation*}
    \( \mathbb{R}^{ n } \) with the Euclidean distance as its metric is called 
    the \( n \)-\textbf{dimensional Euclidean Space}.
\end{egBox}

\begin{egBox}{Discrete Metric}[eg:metric-2]
    A more extreme metric is the \textbf{discrete metric} and is defined as 
    follows: 
    \begin{equation*}
        d( x, y )
        =
        \begin{cases} 
            0 \quad &x = y
            \\ 
            1 \quad &x \neq y
        \end{cases}
    \end{equation*}
\end{egBox}

\begin{egBox}{Open Sets in a Discrete Metric Space}[eg:metric-3]
    If \( X \) is a discrete metric space (that is, equipped with the 
    discrete metric), then every subset of \( X \) is open in \( X \).

    \baseSkip

    To show why, let \( U \subseteq X \) be given. Let \( x \in U \) also be 
    given. Notice that in the discrete metric, balls are given as shown: 
    \begin{equation*}
        B_{ r }( x )
        =
        \begin{cases} 
            \{ x \} \quad & 0 < r \leq 1
            \\
            U \quad & r > 1
        \end{cases}
    \end{equation*}
    Thus, if we choose \( r \) such that \( 0 < r \leq 1 \), then we see that 
    \( B_{ r }( x ) = \{ x \} \subseteq U \), which tells us that \( U \) is 
    open in \( X \).
\end{egBox}

\begin{egBox}{Trivial Open Sets}[eg:metric-4]
    For a metric space \( X \), the empty set \( \emptyset \) and the entire set
    \( X \) are always open sets in \( X \).
\end{egBox}

\begin{egBox}{Open Balls are Indeed Open}[eg:metric-5]
    Let \( X \) be a metric space.
    We have that any open ball in \( X \) is open.

    \baseSkip

    To show why, let's start by having center \( p \in X \) and radius 
    \( r > 0 \) be given.
    We want to show that for any \( x \in B_{ r }( p ) \), we are able to find
    some other radius \( \epsilon > 0 \) such that \( B_{ \epsilon }( x ) 
    \subset B_{ r }( p ) \). 
    Let \( d \equiv d( p, x ) \).
    Our claim is that \( B_{ r - d }( x ) \subset B_{ r }( p ) \).

    \baseSkip 

    To prove the claim, we start by letting \( y \in B_{ r - d }( x ) \).
    We eventually want to show that \( d( y, p ) < r \), which will guarantee 
    that \( y \in B_{ r }( p ) \).
    By the triangle inequality, we see that 
    \begin{equation*}
        d ( y, p ) 
        \leq 
        \underbrace{ d ( y, x ) }_{ < r - d } + 
        \underbrace{ d ( x, p ) }_{ d }
        <
        r
    \end{equation*}
    Which proves what we wanted to show.

    \baseSkip 

    Thus, we have that for any \( y \in B_{ r - d }( x ) \), we get 
    \( y \in B_{ r }( p ) \).
    Therefore,  \( B_{ r - d }( x ) \subset B_{ r }( p ) \), which tells us that
    \( B_{ r }( p ) \) is open.
\end{egBox}