\begin{remarkBox}{On the Urysohn Lemma}
    The Urysohn Lemma says that if every pair of disjoint closed sets in \( X \) can
    be separated by disjoint open sets, then each such pair can be separated by a 
    continuous function.

    \baseSkip

    Notice that the converse is trivial -- that is, if a pair of disjoint closed sets in
    \( X \) can be separated by a continuous function, then each such pair can be 
    separated by disjoint open sets; if \( f: X \rightarrow ( 0, 1 ) \) is the function,
    then \( f^{ -1 } ( [ 0, \frac{ 1 }{ 2 } ) ) \) and 
    \( f^{ -1 } ( ( \frac{ 1 }{ 2 }, 1 ] ) \) are disjoint open sets containing \( A \)
    and \( B \), respectively.

    \baseRule

    The proof of the Urysohn lemma cannot be generalized to show that in a regular 
    space, where you can separate points from closed sets by disjoint open sets, you
    can also separate points from closed sets by continuous functions.

    \baseSkip

    At first glance, it seems that the proof of the Urysohn lemma should go through.
    You take a point \( a \) and a closed set \( B \) not containing \( a \), and you
    begin the proof just as before by defining \( U_{ 1 } = X \setminus B \) and 
    choosing \( U_{ 0 } \) to be an open set about \( a \) whose closure is contained in
    \( U_{ 1 } \) (using the regularity of \( X \)).
    But at the very next step of the proof, you run into difficulty. Suppose that
    \( p \) is the next rational number in the sequence after \( 0 \) and \( 1 \).
    You want to find an open set \( U_{ p } \) such that 
    \( \overline{ U_{ 0 } } \subset U_{ p } \) and 
    \( \overline{ U_{ p } } \subset U_{ 1 } \) -- for this, regularity is not enough.
\end{remarkBox}

\begin{remarkBox}{On Completely Regular}
    Requiring that one be able to separate a point from a closed set by a continuous
    function is, in fact, a stronger condition than requiring that one can separate
    them by disjoint open sets.
    This requirement is made into a new separation axiom, which we defined to be
    [\hyperlink{def:33_completely_regular}{completely regular}].
\end{remarkBox}

\begin{remarkBox}{On Well-Ordered}
    Given a set \( A \) without an order relation, it is natural to ask whether there
    exists an order relation for \( A \) that makes it into a well-ordered set.
    If \( A \) is finite, any bijection
    \begin{equation*}
        f: A \rightarrow \{ 1, \ldots, n \}
    \end{equation*}
    can be used to define an order relation on \( A \); under this relation, \( A \) 
    has the same order type as the ordered set \( \{ 1 , \ldots , n \} \).
    In fact, every order relation on a finite set can be obtained this way, as shown in
    [\hyperlink{thm:10.1}{Theorem 10.1}].
\end{remarkBox}