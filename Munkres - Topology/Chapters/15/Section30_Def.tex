\begin{defBox}{First-Countable}[def:30_first-countable]
    A space \( X \) is said to have a \textbf{countable basis at} \( x \) if
    there is a countable collection \( \mathcal{B} \) of neighborhoods of
    \( x \) such that each neighborhood of \( x \) contains at least one of the
    elements of \( \mathcal{B} \).
    A space that has a countable basis at each of its points is said to 
    satisfy the \textbf{first countability axiom}, or to be 
    \textbf{first-countable}.
\end{defBox}

\begin{defBox}{Second-Countable}[def:30_second-countable]
    If a space \( X \) has a countable basis for its topology, then \( X \) is
    said to satisfy the \textbf{second countability axiom}, or to be 
    \textbf{second-countable}.

    \baseSkip

    I.e., A topological space is second-countable if its topology can be
    generated by a countable basis.
\end{defBox}

\begin{defBox}{Dense Set}[def:30_dense]
    A subset \( A \) of a space \( X \) is said to be \textbf{dense} in \( X \)
    if \( \overline{ A } = X \).
\end{defBox}

\begin{defBox}{Lindel\"{o}f space}[def:30_lindelof_space]
    A space for which every open covering contains a countable subcovering
    is called a \textbf{Lindel\"{o}f space}.
\end{defBox}

\begin{defBox}{Separable}[def:30_separable]
    A space having a countable dense subset is often said to be 
    \textbf{separable}.
\end{defBox}