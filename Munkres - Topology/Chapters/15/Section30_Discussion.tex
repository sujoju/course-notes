\begin{remarkBox}{Countable}
    Recall that a set is \textbf{countable} if it is bijective with a subset 
    of \( \mathbb{N} \).
    Countability is a particular instance of the \textbf{cardinality} or size
    of a set.
    If \( X \) is a set, then we use \( \lvert X \rvert \) to denote its
    cardinality.

    \baseSkip

    For example, the set \( \mathbb{Q} \) of rational numbers is countable,
    but the set \( \mathbb{R} \) of real numbers is not.
    In fact, we have that the real numbers have a strictly larger cardinality
    than the rational numbers.

    \baseRule

    Some fundamental properties:
    
    \begin{itemize}
        \item The \textit{finite} product of countable sets is countable.
        \item Any countable union of countable sets is countable.
    \end{itemize}

    Notice a countable product of countable sets is not countable!
    The intuition is that products of sets grow far faster in size than
    unions of sets; products correspond to \textit{multiplying} cardinalities
    while unions correspond to something closer to adding cardinalities.
\end{remarkBox}

\begin{remarkBox}{On First-Countable}
    The most useful fact concerning spaces that satisfy this axiom is the fact
    that in such a space, convergent sequences are adequate to detect limit
    points of sets and to check continuity of functions.
\end{remarkBox}

\begin{remarkBox}{On Second-Countable}
    Why is this axiom interesting? Well, for one thing, many familiar spaces
    do satisfy it.
    For another, it is a crucial hypothesis used in proving such theorems
    as the Urysohn metrization theorem.
\end{remarkBox}

\begin{remarkBox}{On [\hyperlink{thm30.3}{Theorem 30.3}]}
    The two properties listed in [\hyperlink{thm:30.3}{Theorem 30.3}]
    are sometimes taken as alternative countability axioms.

    \baseSkip

    Separable does not imply second-countable -- \( \mathbb{R}_{ \ell } \).
\end{remarkBox}

\begin{remarkBox}{More Examples}
    More examples can be found in \S 30 in Munkres.
\end{remarkBox}