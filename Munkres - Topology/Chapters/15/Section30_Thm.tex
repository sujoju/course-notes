\begin{thmBox}{30.1}[thm:30.1]
    Let \( X \) be a topological space.

    \begin{enumerate}[label = (\alph*)]
        \item Let \( A \) be a subset of \( X \). If there is a sequence of 
            points of \( A \) converging to \( x \), then
            \( x \in \overline{ A } \); the converse hold if \( X \) is 
            first countable.
        \item Let \( f: X \rightarrow Y \). If \( f \) is continuous, then for
            every convergent sequence \( x_{ n } \rightarrow x \) in \( X \),
            the sequence \( f ( x_{ n } ) \) converges to \( f ( x ) \).
            The converse holds if \( X \) is first-countable.
    \end{enumerate}

    \baseRule

    \begin{proofBox}
        The proof for this proof can be seen in \S 21 -- it is just a generalization
        under the hypothesis of metrizability. 
    \end{proofBox}
\end{thmBox}

\begin{thmBox}{30.2}[thm:30.2]
    A subspace of a first-countable space is first-countable, and a countable
    product of first-countable spaces is first-countable.
    
    \baseSkip

    A subspace of a second-countable space is second-countable, and a countable 
    product of second-countable spaces is second-countable.

    \baseRule

    \begin{proofBox}
        Let us first consider the second countability axiom.
        If \( \mathcal{B} \) is a countable basis for \( X \), then we have that
        \begin{equation*}
            \{ B \cap A \mid B \in \mathcal{B} \}
        \end{equation*}
        is a countable basis for the subspace of \( A \) of \( X \).
        Thus, we have that \( A \) is second-countable.
        Now, if \( \mathcal{B}_{ i } \) is a countable basis for the spaces
        \( X_{ i } \), then the collection of all products
        \( \prod U_{ i } \), where \( U_{ i } \in \mathcal{B}_{ i } \) for 
        finitely many values of \( i \) and \( U_{ i } = X_{ i } \) for all 
        other values of \( i \), is a countable basis for \( \prod X_{ i } \).

        \baseSkip

        Let us now consider the first countability axiom.
        Let \( A \) be a given subset of \( X \).
        Let \( x \in A \) be given as well.
        Since \( X \) is first-countable and \( x \in A \subset X \), it 
        follows that there is a countable basis \( \mathcal{B}_{ x } \) at
        \( x \) in the topology of \( X \).
        As a result, we can see that
        \begin{equation*}
            \{ B \cap A \mid B \in \mathcal{B}_{ x } \}
        \end{equation*}
        is a countable basis for \( x \) in the subspace topology of \( A \).
        Since this is true for any \( x \in A \), we have that \( A \) is 
        first-countable.
        Now, let \( \prod X_{ i } \) be a countable product of first-countable
        spaces.
        Let \( \mathbf{x} \in \prod X_{ i } \) be any given point.
        Since we know that each \( X_{ i } \) is first-countable, we have that
        for each coordinate \( x_{ i } \) of \( \textbf{x} \), there is some
        countable basis \( \mathcal{B}_{ x_{ i } } \) at \( x \) in the topology
        of \( X_{ i } \).
        The collection of all products \( \prod U_{ i } \), where
        \( U_{ i } \in \mathcal{B}_{ x_{ i } } \) for finitely many values of 
        \( i \) and \( U_{ i } = X_{ i } \) for all other values of \( i \), 
        is a countable basis for \( \mathbf{x} \).
        Thus, we have that \( \prod X_{ i } \) is first-countable.
    \end{proofBox}
\end{thmBox}

\begin{thmBox}{30.3}[thm:30.3]
    Suppose that \( X \) has a countable basis -- that is, suppose that 
    \( X \) is second countable.
    Then:

    \begin{enumerate}[label = (\alph*)]
        \item Every open covering of \( X \) contains a countable subcollection
            covering \( X \).
            I.e., \( X \) is Lindel\"{o}f.
        \item There exists a countable subset of \( X \) that is dense in 
            \( X \).
            I.e., \( X \) is separable.
    \end{enumerate}

    \baseRule

    \begin{proofBox}
        Let \( X \) be second-countable.
        Let \( \mathcal{B} = \{ B_{ n } \} \) be a countable basis for \( X \).

        \baseSkip
        \wrapBox{a}

        Let \( \mathcal{U} \) be an open cover of \( X \).
        We want to show that there exists a countable subcollection of 
        \( \mathcal{U} \) that still covers \( X \).
        If \( \mathcal{B} \) is \textit{subordinate to the open cover} 
        \( \mathcal{U} \), meaning that every \( B \in \mathcal{B} \) is 
        contained in some \( U_{ B } \in \mathcal{U} \), then:
        \begin{equation*}
            \{ U_{ B } \mid B \in \mathcal{B} \}
        \end{equation*}
        is a countable subcover of \( X \).

        \baseSkip

        Given some open cover \( \mathcal{U} \) of a topological space \( X \),
        and a basis \( \mathcal{B} \) for the topology on \( X \), then there
        exists a basis \( \mathcal{B}_{ \mathcal{U} } \subset \mathcal{B} \) and
        is subordinate to \( \mathcal{U} \).

        \baseSkip

        To show why, let \( B \in \mathcal{B} \) be given such that it is not
        contained in any of the open sets in the collection \( \mathcal{U} \).
        Let \( x \in B \) be given.
        We know that because \( \mathcal{U} \) is an open cover of \( X \),
        we have that \( x \) must belong to at least one open set in this
        cover, say \( U_{ x } \).
        Now, since \( \mathcal{B} \) is a basis for \( X \) and \( U_{ x } \)
        is open in \( X \), it follows by [\hyperlink{lem:13.2}{Lemma 13.2}]
        that there exists some element \( B' \in \mathcal{B} \) such that
        \( x \in B' \subset U_{ x } \).
        Notice that \( x \in B \) and \( x \in B' \); \( \mathcal{B} \) being a 
        basis tells us that there exists a basis element \( B_{ x } \) 
        containing \( x \) such that \( B_{ x } \subset B \cap B' \).
        Doing this for all \( x \in B \) results in us seeing that 
        \( B = \bigcup_{ x \in B } B_{ x } \); note that each \( B_{ x } \)
        is contained in some element \( U_{ x } \) of the open cover
        \( \mathcal{U} \).
        
        \baseSkip

        To recap, we have essentially broken down \( B \) into smaller basis
        elements, where each of these elements are contained in some element
        \( U_{ x } \) of the open cover \( \mathcal{U} \), and whose union
        is all of \( B \).
        If we do this for all \( B \in \mathcal{B} \) such that it is not
        contained in any of the open sets in the collection \( \mathcal{U} \),
        then we end up with a finer basis \( \mathcal{B}_{ \mathcal{U} } \)
        that is subordinate to \( \mathcal{U} \).

        \baseSkip 

        One other way that we could have proved the existence of a subordinate basis 
        is if we were to remove all the basis elements that were not contained in some 
        element of the open cover \( \mathcal{U} \).
        Indeed, we see that the remaining basis elements are subordinate to the open 
        cover \( \mathcal{U} \), and they still form a basis for the topology on 
        \( X \); the reason being similar to the argument made earlier.

        \baseSkip

        Generally, we can replace \( \mathcal{B} \) with a potentially smaller
        basis that is subordinate to \( \mathcal{U} \), and we can apply the 
        same argument as earlier.

        \baseSkip
        \wrapBox{b}

        For each nonempty \( B \in \mathcal{B} \), let us pick some 
        \( x_{ B } \in B \).
        We claim that 
        \begin{equation*}
            A
            =
            \{ x_{ B } \mid B \in \mathcal{B} \}
        \end{equation*}
        is both countable and dense.
        It is clearly countable since \( \mathcal{B} \) is countable.
        To show that \( A \) is dense, we want to show that 
        \( \mathrm{Cl} \ A = X \).
        Let \( x \in X \) and an open neighborhood \( U \) of \( x \) be given.
        Because we have that \( \mathcal{B} \) is a basis, this implies that 
        there exists
        \( B \in \mathcal{B} \) such that \( x \in B \subset U \).
        Thus, we have that \( x_{ B } \in U \), which results in 
        \( U \cap A \neq \emptyset \).
        Thus, we have that \( x \in \mathrm{Cl} \ A \), which tells us 
        that \( X \subset \mathrm{Cl} \ A \).
        Hence, we see that \( \mathrm{Cl} \ A = X \).
    \end{proofBox}
\end{thmBox}

\begin{thmBox}{Extension of [\hyperlink{thm:30.3}{Theorem 30.3}]}[thm:30.4]
    Let \( X \) be a metric space.
    Then the following statements are equivalent:

    \begin{enumerate}
        \item \( X \) is second-countable
        \item \( X \) is separable
        \item \( X \) is Lindel\"{o}f
    \end{enumerate}

    \baseRule

    \begin{proofBox}
        From [\hyperlink{thm:30.3}{Theorem 30.3}], we can see that 
        \( ( 1 ) \implies ( 2 ) \) and \( ( 1 ) \implies ( 3 ) \).
        It suffices to show that \( ( 2 ) \implies ( 1 ) \) and 
        \( ( 3 ) \implies ( 1 ) \).

        \baseSkip
        \wrapBox{\( ( 2 ) \implies ( 1 ) \)}

        Let \( A \) be a countable dense subset of \( X \).
        We define 
        \( 
            \mathcal{B} 
            = 
            \{ B_{ q }( a ) \mid a \in A \text{ and } q \in \mathbb{Q}_{ > 0 } \} 
        \)
        to be the set of balls of rational radius centered at points in \( A \).
        We claim that \( \mathcal{B} \) is a countable basis for \( X \).

        \baseSkip

        We shall first show that \( \mathcal{B} \) is countable. 
        Let us define \( B_{ \mathbb{Q}_{ > 0 } }( a ) \) to be as follows:
        \begin{equation*}
            B_{ \mathbb{Q}_{ > 0 } }( a )
            =
            \{ B_{ q }( a ) \mid q \in \mathbb{Q}_{ > 0 } \}
        \end{equation*}
        Since \( \mathbb{Q}_{ > 0 } \) is countable, we see that 
        \( B_{ \mathbb{Q}_{ > 0 } }( a ) \) is countable as well.
        Thus, we see that 
        \begin{equation*}
            \mathcal{B}
            =
            \{ B_{ q }( a ) \mid a \in A \text{ and } q \in \mathbb{Q}_{ > 0 } \}
            =
            \bigcup_{ a \in A } B_{ \mathbb{Q}_{ > 0 } }( a )
        \end{equation*}
        is a countable union of countable sets, which we know is countable.

        \baseSkip

        From here, we let \( x \in X \) and \( r \in \mathbb{R}_{ > 0 } \) be given.
        We want to show that there exists some \( a \in A \) and 
        \( q \in \mathbb{Q}_{ > 0 } \) so that \( x \in B_{ q }( a ) \) and 
        \( B_{ q }( a ) \subset B_{ r }( x ) \).
        Since we know that \( A \) is dense in \( X \), we have that 
        \( \overline{ A } = X \); thus, it follows that for any \( x \in X \) and any 
        open neighborhood \( U \) of \( x \), we get that \( U \cap A \neq \emptyset \).
        Since \( B_{ r }( x ) \) is an open neighborhood of \( x \), we see that 
        \( B_{ r }( x ) \cap A \neq \emptyset \); in fact, we can do better and
        say that since \( B_{ \frac{ r }{ 2 } }( x ) \) is an open neighborhood of 
        \( x \), we see that \( B_{ \frac{ r }{ 2 } }( x ) \cap A \neq \emptyset \).
        Let \( a \in B_{ \frac{ r }{ 2 } }( x ) \cap A \).

        \baseSkip

        Our goal now is to find some \( q \in \mathbb{Q}_{ > 0 } \) such that 
        \( x \in B_{ q }( a ) \subset B_{ r }( x ) \).
        Since we know that \( a \in B_{ \frac{ r }{ 2 } }( x ) \), we have that 
        \( d ( a, x ) < \frac{ r }{ 2 } \).
        To guarantee that \( x \in B_{ q }( a ) \), we can choose some rational number
        \( q \in \mathbb{Q}_{ > 0 } \) such that \( d ( a, x ) < q < \frac{ r }{ 2 } \);
        such a rational number \( q \) exists as \( \mathbb{Q} \) is dense in 
        \( \mathbb{R} \).
        All that is left to do is to show that \( B_{ q }( a ) \subset B_{ r }( x ) \)
        -- this amounts to showing that for all \( b \in B_{ q }( a ) \), 
        we have that \( b \in B_{ r }( x ) \).
        The triangle inequality tells us that
        \begin{equation*}
            d ( b, x )
            \leq 
            d ( b, a ) + d ( a, x )
        \end{equation*}
        Because \( x \in B_{ q }( a ) \), we see that 
        \( d ( a, x ) < q < \frac{ r }{ 2 } \); the same could be said for \( b \).
        Thus, we see that 
        \begin{equation*}
            d ( b, x )
            \leq 
            d ( b, a ) + d ( a, x ) 
            < 
            q + q
            <
            \frac{ r }{ 2 } + \frac{ r }{ 2 } = r
        \end{equation*}
        Since our choice of \( b \in B_{ q }( a ) \) was arbitrary, we have that 
        \( B_{ q }( a ) \subset B_{ r }( x ) \).

        \baseSkip

        Finally, we want to show that \( \mathcal{B} \) is a basis that generates the 
        topology on \( X \).
        This amounts to showing that for any open set \( U \) of \( X \) and each
        \( x \in U \), there is an element \( B \in \mathcal{B} \) such that
        \( x \in B \subset U \); notice that this is precisely 
        [\hyperlink{lem:13.2}{Lemma 13.2}].
        Let \( U \) be any open set of \( X \); let \( x \in U \) be given.
        Since \( U \) is open in \( X \) and \( X \) is a metric space, we see that 
        there exists some positive real number \( r \) such that
        \( x \in B_{ r }( x ) \subset U \).
        We have just shown that there exists some \( a \in A \) and 
        \( q \in \mathbb{Q}_{ > 0 } \) so that \( x \in B_{ q }( a ) \) and 
        \( B_{ q }( a ) \subset B_{ r }( x ) \).
        If we let \( B = B_{ q }( a ) \), then we have that 
        \( x \in B \subset B_{ r }( x ) \subset U \), which tells us that 
        \( \mathcal{B} \) is indeed a basis that generates the topology on \( X \).
        Hence, we have that \( X \) has a countable basis -- that is, \( X \) is 
        second-countable.


        \baseSkip
        \wrapBox{\( ( 3 ) \implies ( 1 ) \)}

        Let \( X \) be a Lindel\"{o}f metric space.
        Since \( X \) is a metric space, we have that balls of some radius \( r \), 
        where \( r \) is a real number, are open subsets of \( X \).
        Furthermore, since \( X \) is Lindel\"{o}f, we see that for every open cover
        \( \mathcal{A} \), there exists a countable subcollection of \( \mathcal{A} \) 
        that covers \( X \).
    
        \baseSkip
    
        We start off by noting that
        \begin{equation*}
            \mathcal{A}
            =
            \{ B_{ \frac{ 1 }{ n } }( x ) \mid x \in X \}
        \end{equation*}
        is an open cover of \( X \) for all \( n \in \mathbb{Z}_{ + } \).
        Since \( X \) is Lindel\"{o}f, we have that there exists a countable 
        subcollection of \( \mathcal{A} \) that covers \( X \), and this is true for 
        all \( n \in \mathbb{Z}_{ + } \); let's denote \( \mathcal{A}_{ n } \)
        to be this countable covering of \( X \) by \( \frac{ 1 }{ n } \)-balls.
    
        \baseSkip
    
        We shall now define the following:
        \begin{equation*}
            \mathcal{B}
            =
            \bigcup_{ n \in \mathbb{Z} } \mathcal{A}_{ n }
        \end{equation*}
        We claim that \( \mathcal{B} \) is a countable basis.
        It is clear to see that \( \mathcal{B} \) is countable as it is a countable 
        union of countable sets. 
        All that is left to do is to show that \( \mathcal{B} \) is a basis.
        
        \baseSkip
    
        Let \( U \) be any open set of \( X \).
        Let \( x \) be any element of \( U \).
        Since \( U \) is open, we have that there exists some positive real number 
        \( r \) such that \( x \in B_{ r }( x ) \subset U \) -- i.e., \( U \) being 
        open in a metric space means that we can always put a ball of some positive 
        real radius around each point in \( U \) such that this ball is contained in 
        \( U \).
        Now, because \( \mathcal{A}_{ n } \) is a countable subcover of \( X \) for all 
        \( n \in \mathbb{Z}_{ + } \), we see that \( x \) must be contained in some
        ball of radius \( \frac{ 1 }{ n } \) centered at some point \( y_{ n } \) of
        \( X \) -- that is, \( x \in B_{ \frac{ 1 }{ n } }( y_{ n } ) \) for all 
        \( n \in \mathbb{Z}_{ + } \), where \( y_{ n } \) is some element of \( X \) 
        that depends on the particular countable subcover \( \mathcal{A}_{ n } \) and 
        \( x \in X \) that we are considering.
    
        \baseSkip
    
        Our goal is to find some \( n \in \mathbb{Z}_{ + } \) such that 
        \( B_{ \frac{ 1 }{ n } }( y_{ n } ) \subset B_{ r }( x ) \). 
        It suffices to show that for all \( z \in B_{ \frac{ 1 }{ n } }( y_{ n } ) \), 
        we have that \( z \in B_{ r }( x ) \).
        Since we have a metric to work with, say \( d \), the triangle inequality tells 
        us that
        \begin{equation*}
            d ( z, x )
            \leq 
            d ( z, y_{ n } ) + d ( y_{ n }, x )
        \end{equation*}
        Because \( x \in B_{ \frac{ 1 }{ n } }( y_{ n } ) \), we see that 
        \( d ( y_{ n }, x ) < \frac{ 1 }{ n } \); notice that the same can be said with 
        \( z \).
        Thus, we see that 
        \begin{equation*}
            d ( z, x )
            \leq 
            d ( z, y_{ n } ) + d ( y_{ n }, x ) 
            < 
            \frac{ 1 }{ n } + \frac{ 1 }{ n } = \frac{ 2 }{ n }
        \end{equation*}
        In order for \( z \in B_{ r }( x ) \), we need \( d ( z, x ) < r \) -- this is
        guaranteed to happen if \( \frac{ 1 }{ n } < \frac{ r }{ 2 } \).
        Therefore, if we choose \( n \in \mathbb{Z}_{ + } \) such that 
        \( \frac{ 1 }{ n } < \frac{ r }{ 2 } \), then we end up getting that 
        \( z \in B_{ r }( x ) \) for all \( z \in B_{ \frac{ 1 }{ n } }( y_{ n } ) \) 
        -- that is, we get 
        \( x \in B_{ \frac{ 1 }{ n } }( y_{ n } ) \subset B_{ r }( x ) \subset U \).
    
        \baseSkip
    
        Putting everything together, we see that for all \( x \in U \), there exists 
        some \( n \in \mathbb{Z}_{ + } \) such that 
        \( x \in B_{ \frac{ 1 }{ n } }( y_{ n } ) \subset U \), 
        where \( y_{ n } \) is some element of \( X \) that depends on the particular 
        finite subcover \( \mathcal{A}_{ n } \) and \( x \in X \) that we are 
        considering.
        By [\hyperlink{lem:13.2}{Lemma 13.2}], we have that \( \mathcal{B} \) is a 
        countable basis for \( X \).
        Thus, \( X \) is second-countable.
    \end{proofBox}
\end{thmBox}

\begin{thmBox}{Variation of [\hyperlink{thm:30.2}{Theorem 30.2}]}[thm:30.5]
    Let \( X \) be a topological space and Lindel\"{o}f.
    Any closed subspace of \( X \) is Lindel\"{o}f.

    \baseRule

    \begin{proofBox}
        Let \( A \) be a closed subspace of \( X \). 
        Let \( \mathcal{U} \) be an open cover of \( A \) by sets open in \( A \). 
        We want to show that there exists a countable subcover of \( A \) by sets 
        open in \( A \). 
        To start, we first note that for each element \( U_{ i } \in \mathcal{U} \),
        there exists some open set \( V_{ i } \) in \( X \) such that 
        \( U_{ i } = A \cap V_{ i } \) by definition of the subspace topology. 
        From this, we can see that \( \mathcal{V} = \{ V_{ i } \} \) is an open cover 
        of \( A \) by sets open in \( X \). 

        \baseSkip 

        Since \( A \) is closed in \( X \), this means that \( X \setminus A \) must be 
        open in \( X \) as well. 
        Thus, we get that 
        \begin{equation*}
            ( X \setminus A ) \cup \left( \bigcup V_{ i } \right)   
        \end{equation*}
        is an open cover of \( X \) by sets open in \( X \). 
        Since \( X \) is Lindel\"{o}f, it follows that there exists a countable subcover
        of \( X \) for our particular open cover. 
        Let's say that such a countable subcover is as follows: 
        \begin{equation*}
            \left\{ X \setminus A, V_{ i_{ j } } \mid j \in \mathbb{Z}_{ + } \right\}
        \end{equation*}
        With this, we can see that \( \{ V_{ i_{ j } } \}_{ j \in \mathbb{Z}_{ + } } \)
        is still an open cover of \( A \) by sets open in \( X \), which implies that 
        \( \{ U_{ i_{ j } } = A \cap V_{ i_{ j } } \}_{ j \in \mathbb{Z}_{ + } } \)
        is a countable open cover of \( A \) by sets open in \( A \).
        Thus, we have that \( A \) is Lindel\"{o}f.
    \end{proofBox}
\end{thmBox}