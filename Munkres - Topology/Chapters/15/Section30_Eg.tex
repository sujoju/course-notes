\begin{egBox}{Second Axiom Implies First}[eg:30.1]
    The second countability axiom implies the first.

    \baseSkip

    Let \( X \) be second-countable.
    The \( X \) has a countable basis \( \mathcal{B} \) for its topology.
    For any points \( x \in X \), we have that the subset of \( \mathcal{B} \) 
    consisting of those basis elements containing the point \( x \) is a 
    countable basis at \( x \).
    Thus, \( X \) is first-countable as well.
\end{egBox}

\begin{egBox}{Examples of Second-Countable Spaces}[eg:30.2]
    \( \mathbb{R} \) is second-countable; a countable basis for \( \mathbb{R} \)
    would be as follows:
    \begin{equation*}
        \{ ( p, q ) \mid p, q \in \mathbb{Q} \}
    \end{equation*}
    We have shown in a previous HW that this was indeed a basis.
    We also know that this collection is countable since 
    \( \mathbb{Q} \times \mathbb{Q} \) is countable, and the collection 
    can be injectively mapped into \( \mathbb{Q} \times \mathbb{Q} \).
\end{egBox}

\begin{egBox}{\( \mathbb{R}_{ \ell } \) is not Second-Countable}[eg:30.3]
    Let \( \mathcal{B} \) be any basis for the lower-limit topology.
    For all \( x \in \mathbb{R} \), we have that the interval 
    \( [ x, x + 1 ) \) is certainly open in the lower-limit topology.
    Thus, we have that there exists some \( B_{ x } \in \mathcal{B} \) so
    that \( x \in B_{ x } \subset [ x, x + 1 ) \).
    If we have that \( x \neq y \), then \( B_{ x } \neq B_{ y } \);
    this is because the smallest real number that in \( B_{ x } \) is \( x \),
    but the smallest number in \( B_{ y } \) is \( y \) -- thus, these sets 
    cannot be equal.
    Notice that the function \( f: \mathbb{R} \rightarrow \mathcal{B} \)
    given by \( x \mapsto B_{ x } \) is an injection, thus \( \mathcal{B} \)
    must be uncountable (it must have a larger than or equal to cardinality
    than \( \mathbb{R} \), which is uncountable).
\end{egBox}

\begin{egBox}{Dense and Separable}[eg:30.3]
    \( \mathbb{R} \) and \( \mathbb{R}_{ \ell } \) are both separable;
    for \( \mathbb{R} \), we see that \( \mathbb{Q} \) is a dense subset of 
    \( \mathbb{R} \).
    We claim that \( \mathbb{Q} \) is also dense in \( \mathbb{R}_{ \ell } \).
    If it weren't dense, there is some point some \( x \in \mathbb{R} \) such
    that \( x \notin \overline{ \mathbb{Q} } \).
    Thus, there is some open open interval that contains \( x \) that is 
    disjoint from \( \mathbb{Q} \), however, this cannot be the case (flesh
    this point out some more).
\end{egBox}