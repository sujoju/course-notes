\begin{remarkBox}{Open sets in a Topology}
    Notice that the notion of open sets differs from what we originally know 
    them to be. 
    When working with topologies, we first need to figure out the subsets that 
    will make up the topology (there could be many possibilities).
    Once such subsets are found or determined, we have that they are 
    automatically open by definition. 
\end{remarkBox}

\begin{remarkBox}{On the Third Condition of a Topology}
    We can restate the third condition instead to be that \( \mathcal{T} \) is 
    closed under \textit{pairwise intersections} -- that is, if 
    \( U, V \in \mathcal{T} \), then \( U \cap V \in \mathcal{T} \). 
    
    \baseSkip

    Using induction, we can very easily show that this re-stated condition is 
    equivalent to the original third equation. 
\end{remarkBox}

\begin{remarkBox}{Finite Complement Topology}
    This topology is also referred to as the \textbf{cofinite topology};
    Cofinite refers to the subsets of a set \( X \) whose complements in \( X \)
    are finite. 

    \baseSkip

    Thus, we have the cofinite topology to include all the cofinite subsets of 
    \( X \) and the empty set -- slightly different from the definition given 
    in the original example. 
\end{remarkBox}

\begin{remarkBox}{Finer and Coarser Topologies}
    Finally, this section ended with the concept of finer or coarser 
    topologies. 
    In its essence, we can think of a topological space as being something like 
    a bucket of gravel -- the pebbles and all unions of collections of pebbles 
    being the open sets.
    If we now smash the pebbles into smaller ones, the collection of open sets 
    has been enlarged, and the topology, like the gravel, is said to have been 
    made finer by the operation. 
    Such an operation for a topology would correspond to partitioning subsets
    into smaller subsets (while keeping the original subsets). 
    
    \baseSkip 

    Furthermore, the introduction of finer or coarser topologies brought
    forth the notion of topologies being comparable. It should be noted that topologies in general need not be comparable.
\end{remarkBox}

\begin{remarkBox}{Topologies Induced from Metrics}
    We can show that for any finite sets, we can only induce a topology from a 
    metric if the metric is discrete -- that is, the only topology induced from 
    metrics for any finite set is the discrete topology.
\end{remarkBox}