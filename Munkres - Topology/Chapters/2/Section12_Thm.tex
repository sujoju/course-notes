\begin{thmBox}{deMorgan's Laws}[thm:12_deMorgan]
    deMorgan's law essentially states that complements transform unions into 
    intersections and vice-versa. Putting it in a more mathematical statement, 
    we see: 
    \begin{equation*}
        X \setminus \bigcap_{ i \in I } A_{ i }
        =
        \bigcup_{ i \in I } ( X \setminus A_{ i } )
        \quad \mathrm{and} \quad
        X \setminus \bigcup_{ i \in I } A_{ i }
        =
        \bigcap_{ i \in I } ( X \setminus A_{ i } )
    \end{equation*}

    \baseRule

    \begin{proofBox}
        We find that both statements are merely statements of logic. 

        \baseSkip 

        For the first statement, let's say that 
        \( x \in X \setminus \bigcap_{ i \in I } A_{ i } \).
        Notice that this just means that 
        \( x \notin \bigcap_{ i \in I } A_{ i } \), which further means 
        that there exists at least one \( i \in I \) such that 
        \( x \notin A_{ i } \). 
        I.e., we see that there exists at least one \( i \in I \) such that 
        \( x \in X \setminus A_{ i } \), which means that 
        \( x \in \bigcup_{ i \in I } ( X \setminus A_{ i } ) \).
        Therefore we have that 
        \begin{equation*}
            X \setminus \bigcap_{ i \in I } A_{ i }
            \subseteq
            \bigcup_{ i \in I } ( X \setminus A_{ i } )
        \end{equation*}
        Now let's say that 
        \( x \in \bigcup_{ i \in I } ( X \setminus A_{ i } ) \). 
        This means that there exists at least one \( i \in I \) such that 
        \( x \in X \setminus A_{ i } \), or equivalently, such that 
        \( x \notin A_{ i } \). 
        As a result, it must be the case, then, that 
        \( x \notin \bigcap_{ i \in I } A_{ i } \), which is equivalent to 
        \( x \in X \setminus \bigcap_{ i \in I } A_{ i } \)
        Therefore we have that 
        \begin{equation*}
            \bigcup_{ i \in I } ( X \setminus A_{ i } )
            \subseteq
            X \setminus \bigcap_{ i \in I } A_{ i }
        \end{equation*}
        Combining both inclusions results in the equality that we were after.

        \baseSkip

        For the second statement, let's say that 
        \( x \in X \setminus \bigcup_{ i \in I } A_{ i } \).
        Again, this just means that 
        \( x \notin \bigcup_{ i \in I } A_{ i } \), which further means that 
        that for all \( i \in I \), we must have \( x \notin A_{ i } \). 
        I.e., we see that for all \( i \in I \), we must have  
        \( x \in X \setminus A_{ i } \), which means that 
        \( x \in \bigcap_{ i \in I } ( X \setminus A_{ i } ) \).
        Therefore we have that 
        \begin{equation*}
            X \setminus \bigcup_{ i \in I } A_{ i }
            \subseteq
            \bigcap_{ i \in I } ( X \setminus A_{ i } )
        \end{equation*}

        \pagebreak

        Now let's say that 
        \( x \in \bigcap_{ i \in I } ( X \setminus A_{ i } ) \). 
        This means that for all \( i \in I \), we have
        \( x \in X \setminus A_{ i } \), or equivalently, such that 
        \( x \notin A_{ i } \). 
        As a result, it must be the case, then, that 
        \( x \notin \bigcup_{ i \in I } A_{ i } \), which is equivalent to 
        \( x \in X \setminus \bigcup_{ i \in I } A_{ i } \)
        Therefore we have that 
        \begin{equation*}
            \bigcap_{ i \in I } ( X \setminus A_{ i } )
            \subseteq
            X \setminus \bigcup_{ i \in I } A_{ i }
        \end{equation*}
        Combining both inclusions results in the equality that we were after.
    \end{proofBox}
\end{thmBox}