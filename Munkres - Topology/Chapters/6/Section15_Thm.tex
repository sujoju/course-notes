\begin{thmBox}{15.1}[thm:15.1]
    If \( \mathcal{B} \) is a basis for the topology of \( X \) and 
    \( \mathcal{C} \) is a basis for the topology of \( Y \), then the
    collection
    \begin{equation*}
        \mathcal{D} 
        =
        \{ B \times C \mid B \in \mathcal{B} \text{ and } C \in \mathcal{C} \}
    \end{equation*}
    is a basis for the topology of \( X \times Y \).

    \baseSkip

    We can extend this to more generally be the following: 
    If the topologies on \( X_{ 1 } , \ldots ,  X_{ r } \) are generated by the 
    bases \( \mathcal{B}_{ 1 } , \ldots , \mathcal{B}_{ r } \), then the 
    collection of all sets of the form
    \begin{equation*}
        B_{ 1 } \times \ldots \times B_{ r }
    \end{equation*}
    where \( B_{ i } \in \mathcal{B}_{ i } \) for all \( i \) is a basis for the
    topology of \( X_{ 1 } \times \ldots \times  X_{ r } \)

    \baseRule

    \begin{proofBox}
        For this proof, we shall apply [\hyperlink{lem:13.2}{Lemma 13.2}].
        Let \( W \) be some open set of \( X \times Y \), and let 
        \( x \times y \) be some point of \( W \).
        By definition of the product topology, we have that \( W \) being 
        open means that there exists a basis element \( U \times V \) such that
        \( x \times y \in W \subset U \times V \).
        Now because \( \mathcal{B} \) and \( \mathcal{C} \) are bases for 
        \( X \) and \( Y \), respectively, we can choose an element \( B \)
        of \( \mathcal{B} \) such that \( x \in B \subset U \), and an element
        \( C \) of \( \mathcal{C} \) such that \( y \in C \subset V \).
        From this, it follows that \( x \times y \in B \times C \subset W \).
        Thus, we have that the collection \( \mathcal{D} \) meets the criterion
        of [\hyperlink{lem:13.2}{Lemma 13.2}], so we have that
        \( \mathcal{D} \) is a basis for \( X \times Y \).
    \end{proofBox}
\end{thmBox}

\begin{thmBox}{15.2}[thm:15.2]
    The collection 
    \begin{equation*}
        \mathcal{S}
        =
        \{ \mathrm{proj}_{ 1 }^{ -1 } ( U ) \mid U \text{ open in } X \}
        \cup 
        \{ \mathrm{proj}_{ 2 }^{ -1 } ( V ) \mid V \text{ open in } Y \}
    \end{equation*}
    is a subbasis for the product topology on \( X \times Y \).

    \baseRule

    \begin{proofBox}
        Let us start by denoting \( \mathcal{T} \) to be the product topology 
        on \( X \times Y \); let \( \mathcal{T}' \) be the topology generated
        by \( \mathcal{S} \).
        Our goal is to show that \( \mathcal{T} = \mathcal{T}' \).

        \baseSkip

        Because every element of \( \mathcal{S} \) belongs to \( \mathcal{T} \),
        it follows that any arbitrary unions of these elements, or finite 
        intersections of these elements, or even arbitrary unions of finite
        intersections of these elements also belong to \( \mathcal{T} \).
        Thus, we have that \( \mathcal{T}' \subset \mathcal{T} \).

        \baseSkip

        On the other hand, we have that every basis element \( U \times V \)
        for the topology \( \mathcal{T} \) is a finite intersection of elements
        of \( \mathcal{S} \), since 
        \begin{equation*}
            \begin{aligned}
                \mathrm{proj}_{ 1 }^{ -1 } ( U )
                &=
                \{ x \times y \in X \times Y \mid 
                \mathrm{proj}_{ 1 } ( x \times y ) \in U \}
                \\
                &=
                \{ x \times y \in X \times Y \mid x \in U \}
                \\[5mm]
                \mathrm{proj}_{ 2 }^{ -1 } ( V )
                &=
                \{ x \times y \in X \times Y \mid 
                \mathrm{proj}_{ 2 } ( x \times y ) \in V \}
                \\
                &=
                \{ x \times y \in X \times Y \mid y \in V \}
            \end{aligned}
        \end{equation*}
        From this, we can see that 
        \begin{equation*}
            \mathrm{proj}_{ 1 }^{ -1 } ( U ) 
            \cap 
            \mathrm{proj}_{ 2 }^{ -1 } ( V )
            =
            \{ x \times y \in X \times Y \mid x \in U \text{ and } y \in V \}
            =
            U \times V
        \end{equation*}
        Therefore, we see that \( U \times V \) belongs to \( \mathcal{T}' \),
        so that \( \mathcal{T} \subset \mathcal{T}' \) as well.

        \baseSkip

        Putting everything together results in \( \mathcal{T} = \mathcal{T}' \).
    \end{proofBox}
\end{thmBox}

\begin{thmBox}{Product of Closed Subsets}[thm:15_product_closed]
    Let \( X_{ 1 } , \ldots , X_{ r } \) be topological spaces.
    If \( A_{ i } \) is a closed subset of \( X_{ i } \) for each \( i \), then 
    \( A_{ 1 } \times \ldots \times A_{ r } \) is closed in 
    \( X_{ 1 } \times \ldots \times X_{ r } \).

    \baseSkip

    In fact, we see that this holds for arbitrary products of closed sets.

    \baseRule

    \begin{proofBox}
        It suffices to show that \( ( X_{ 1 } \times \ldots \times X_{ r } )
        \setminus ( A_{ 1 } \times \ldots \times A_{ r } ) \) is open.
        We know that 
        \begin{equation*}
            ( X_{ 1 } \times \ldots \times X_{ r } )
            \setminus
            ( A_{ 1 } \times \ldots \times A_{ r } )
            =
            [ 
                ( X_{ 1 } \setminus A_{ 1 } ) \times X_{ 2 } 
                \times \ldots \times 
                X_{ r } 
            ]
            \cup \ldots \cup 
            [ 
                X_{ 1 } 
                \times \ldots \times 
                X_{ r - 1 } \times ( X_{ r } \setminus A_{ r } )
            ]
        \end{equation*}
        Since each \( A_{ i } \) is a closed subset of \( X_{ i } \) for each 
        \( i \), we get that \( X_{ i } \setminus A_{ i } \) is open for each 
        \( i \) as well.
        Thus, we have 
        \(
            [
                ( X_{ 1 } \setminus A_{ 1 } ) \times X_{ 2 } 
                \times \ldots \times 
                X_{ r }
            ]
            , \ldots , 
            [
                X_{ 1 } 
                \times \ldots \times 
                X_{ r - 1 } \times ( X_{ r } \setminus A_{ r } )
            ]
        \)
        to all be open in \( X_{ 1 } \times \ldots \times X_{ r } \).
        Because \( X_{ 1 } \times \ldots \times X_{ r } \) is a topology, 
        we have that arbitrary unions of open sets will remain open, 
        which tells us that \( ( X_{ 1 } \times \ldots \times X_{ r } )
        \setminus ( A_{ 1 } \times \ldots \times A_{ r } ) \) is open.
        Hence, \( A_{ 1 } \times \ldots \times A_{ r } \) is closed in 
        \( X_{ 1 } \times \ldots \times X_{ r } \).
    \end{proofBox}
\end{thmBox}
