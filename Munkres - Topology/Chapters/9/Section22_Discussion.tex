\begin{remarkBox}{On the Quotient Map}
    The condition for a quotient map is \textit{stronger} than continuity; for 
    continuity, we had \( p^{ -1 } ( U ) \) to be open in \( X \) whenever 
    \( U \) is open in \( Y \), which is only one direction of implication.
    However, a quotient map requires an equivalence -- that is, both directions
    of implication.

    \baseRule

    We also note that an equivalent condition for a quotient map is as follows:
    The map \( p \) is said to be a \textbf{quotient map} provided a subset
    \( A \) of \( Y \) is closed in \( Y \) if and only if \( p^{ -1 } ( A ) \)
    is closed in \( X \).
    The equivalence of these two conditions follows from the fact that 
    \begin{equation*}
        f^{ -1 } ( Y \setminus B ) = X \setminus f^{ -1 } ( B )
    \end{equation*}
\end{remarkBox}

\begin{remarkBox}{On the Quotient Space \( X^{ * } \)}
    The topology of \( X^{ * } \) can be described in another way: 
    A subset \( U \) of \( X^{ * } \) is a collection of equivalence classes,
    and the set \( p^{ -1 } ( U ) \) is just the union of equivalence classes
    belonging to \( U \).
    Thus, the typical open set of \( X^{ * } \) is a collection of equivalence
    classes whose union is an open set of \( X \).
    One can see that this does indeed form a topology on \( X^{ * } \).

    \baseSkip

    With this description, we can see that the surjective map 
    \( \pi: X \rightarrow X^{ * } \), where \( \pi ( x ) = [ x ] \), is 
    a quotient map, and thus induces a quotient topology on \( X^{ * } \).

    \baseSkip 

    As a result, when \( X \) is a topological space with an equivalence 
    relation and \( X^{ * } \) is the corresponding set of equivalence classes, 
    it is implied that the quotient topology on \( X^{ * } \) is induced by the 
    function \( X \rightarrow X^{ * } \) that sends a point to its equivalence 
    class;
    that is, \( X^{ * } \) is induced by the function \( \pi: X \rightarrow 
    X^{ * } \), where \( \pi ( x ) = [ x ] \).

    \baseRule

    One can think of \( X^{ * } \) as having been obtained by "identifying"
    each pair of equivalent points. 
    For this reason, the quotient space \( X^{ * } \) is often called an
    \textbf{identification space}, or a \textbf{decomposition space}, of 
    the space \( X \).
\end{remarkBox}

\begin{remarkBox}{On Equivalent Quotient Spaces}
    Let \( p: X \rightarrow Y \) be a quotient map.
    The theorem about equivalent quotient spaces is stating that we can 
    always produce an equivalence 
    relation on \( X \) using the quotient map \( p \) -- this equivalence 
    relation being \( x \sim y \iff p ( x ) = p ( y ) \), which states that
    two elements in \( X \) are similar if they are both sent to the same point
    via \( p \).
    I.e., if \( p \) acts two elements in the same way (that is, sending them
    to the same point), then these two elements are similar to each other. 

    \baseSkip

    Notice that this equivalence relation partitions \( X \) up into equivalence
    classes, and we have the set of these equivalence classes \( X^{ * } \)
    to be a quotient space itself; recall that the the quotient topology on 
    \( X^{ * } \) is induced by the function \( X \rightarrow X^{ * } \) that 
    sends a point to its equivalence class -- 
    that is, \( X^{ * } \) is induced by the function \( \pi: X \rightarrow 
    X^{ * } \), where \( \pi ( x ) = [ x ] \).

    \baseSkip 

    Not only that, but we showed that such an equivalence relation 
    results in the quotient space \( Y \) induced by \( p \) to be homeomorphic
    to \( X^{ * } \) -- i.e., \( Y \) and \( X^{ * } \) are equivalent spaces!
    Thus, we can always think of any quotient space as being a set of 
    equivalent classes instead, which is what Munkres does for his definition 
    of quotient spaces.

    \baseRule 

    Munkres offers the following definition for a quotient space:
    Let \( X \) be a topological space, and let \( X^{ * } \) be a partition of
    \( X \) into disjoint subsets whose union is \( X \).
    Let \( p: X \rightarrow X^{ * } \) be the surjective map that carries each
    point of \( X \) to the element of \( X^{ * } \) containing it.
    In the quotient topology induced by \( p \), the space \( X^{ * } \) is 
    called a \textbf{quotient space} of \( X \).

    \baseSkip

    From this definition, he then states that given \( X^{ * } \), there is an
    equivalence relation on \( X \) of which the elements of \( X^{ * } \) are
    the equivalence classes.
\end{remarkBox}

\begin{remarkBox}{Quotient Maps}
    The composite of two quotient maps is a quotient map; this fact follows 
    from the following equation:
    \begin{equation*}
        p^{ -1 } ( q^{ -1 } ( U ) )
        =
        ( q \circ p )^{ -1 } ( U )
    \end{equation*}

    \baseRule

    On the other hand the product of two quotient maps need not be a quotient
    map.
    This also is true for subspaces: if \( p: X \rightarrow Y \) is a quotient
    map and \( A \) is a subspace of \( X \), then the map \( q: A \rightarrow 
    p ( A ) \) obtained by restricting \( p \) need not be a quotient map.
    However, there are special conditions that makes \( q \) a quotient map --
    c.g. [\hyperlink{thm:22.1}{Theorem 22.1}]

    \baseRule

    Finally, if \( X \) is Hausdorff, then there is no reason that the quotient
    space \( X^{ * } \) needs to be Hausdorff.
    However, there is a simple condition for \( X^{ * } \) to satisfy the 
    \( T_{ 1 } \) axiom; one simply requires that each element of the 
    partition \( X^{ * } \) be a closed subset of \( X \).
\end{remarkBox}

\begin{remarkBox}{Proofs and Examples}
    The proofs for the theorems/corollary as well as more examples can be found
    in \S 22 of Munkres.
\end{remarkBox}