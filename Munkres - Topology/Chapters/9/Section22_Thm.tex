\begin{thmBox}{22.1}[thm:22.1]
    Let \( p: X \rightarrow Y \) be a quotient map; let \( A \) be a subspace of
    \( X \) that is saturated with respect to \( p \); let \( q: A \rightarrow 
    p ( A ) \) be the map obtained by restricting \( p \).
    \begin{enumerate}
        \item If \( A \) is either open or closed in \( X \), then \( q \) is 
            a quotient map.
        \item If \( p \) is either an open map or a closed map, then \( q \) is
            a quotient map.
    \end{enumerate}

    \baseRule

    \begin{proofBox}

    \end{proofBox}
\end{thmBox}

\begin{thmBox}{22.2}[thm:22.2]
    Let \( p: X \rightarrow Y \) be a quotient map.
    Let \( Z \) be a space and let \( g: X \rightarrow Z \) be a map that is 
    constant on each set \( p^{ -1 } ( \{ y \} ) \), for \( y \in Y \).
    Then \( g \) induces a map \( f: Y \rightarrow Z \) such that 
    \( f \circ p = g \). 
    The induced map \( f \) is continuous if and only if \( g \) is continuous;
    \( f \) is a quotient map if and only if \( g \) is a quotient map.
    \begin{center}
        \begin{tikzcd}[ row sep = large ]
            X \arrow[ d, swap, "p" ] \arrow [ dr, "g" ]
            \\
            Y \arrow[ r, dashed, swap, "f" ] & Z
        \end{tikzcd}
    \end{center}

    \baseRule

    \begin{proofBox}

    \end{proofBox}
\end{thmBox}

\begin{thmBox}[Corollary]{22.3}[cor:22.3]
    Let \( g: X \rightarrow Z \) be a surjective continuous map.
    Let \( X^{ * } \) be the following collection of subsets of \( X \):
    \begin{equation*}
        X^{ * }
        =
        \{ g^{ -1 } ( \{ z \} ) \mid z \in Z \}
    \end{equation*}
    Give \( X^{ * } \) the quotient topology.
    
    \begin{enumerate}[label = (\alph*)]
        \item The map \( g \) induces a bijective continuous map \( f: X^{ * } 
            \rightarrow Z \), which is a homeomorphism if and only if \( g \) 
            is a quotient map.
            \begin{center}
                \begin{tikzcd}[ row sep = large ]
                    X \arrow[ d, swap, "p" ] \arrow [ dr, "g" ]
                    \\
                    X^{ * } \arrow[ r, swap, "f" ] & Z
                \end{tikzcd}
            \end{center}
        \item If \( Z \) is Hausdorff, then so is \( X^{ * } \)
    \end{enumerate}

    \baseRule

    \begin{proofBox}

    \end{proofBox}
\end{thmBox}

\begin{thmBox}{Equivalent Quotient Spaces}[thm:22_quotient_space_equivalent]
    Let \( p: X \rightarrow Y \) be a quotient map.
    Then \( Y \) is homeomorphic to the quotient space \( X^{ * } \) determined
    by the equivalence relation: \( x \sim y \) if and only if 
    \( p ( x ) = p ( y ) \).

    \baseRule

    \begin{proofBox}
        Consider this diagram of maps:
        \begin{center}
            \begin{tikzcd}[ row sep = large ]
                X \arrow[ d, swap, "\pi" ] \arrow [ dr, "p" ]
                \\
                X^{ * } \arrow[ r, swap, "\overline{ p }" ] & Y
            \end{tikzcd}
        \end{center}
        We define \( \pi ( x ) = [ x ] \) for all \( x \in X \).
        Let us call \( \overline{ p }: X^{ * } \rightarrow Y \) to be defined by
        \( \overline{ p } ( [ x ] ) = p ( x ) \).
        We claim that \( \overline{ p } \) is a homeomorphism between \( X \)
        and \( Y \), which means that we need to check three conditions:
        the first is that \( \overline{ p } \) is well-defined,
        the second is that \( \overline{ p } \) is a bijection, and the third 
        is that \( \overline{ p } \) and \( \overline{ p }^{ -1 } \) are 
        continuous.

        \baseSkip

        For the first and second conditions can be seen via the following
        equivalences:
        \begin{equation*}
            [ x ] = [ y ]
            \iff 
            x \sim y
            \iff 
            p ( x ) = p ( y )
        \end{equation*}
        where the second equivalence follows from our assumption.
        Going from left-to-right gives us that \( \overline{ p } \) is 
        well-defined,
        Going from right-to-left gives us that \( \overline{ p } \) is 
        injective.
        Furthermore, \( \overline{ p } \) is surjective since \( p \) is a
        surjection.
        Thus, \( \overline{ p } \) is a bijection, which tells us that it has 
        an inverse \( \overline{ p }^{ -1 } \).

        \baseSkip

        We now need to verify the third condition -- that is, we need to show 
        that both \( \overline{ p } \) and \( \overline{ p }^{ -1 } \) are
        continuous. 
        We start by noting that
        \( X^{ * } \) is equipped with the quotient topology induced by 
        \( \pi \) and \( Y \) is equipped with the quotient topology induced 
        by \( p \).
        Further notice that \( p = \overline{ p } \circ \pi \).

        \baseSkip

        We first go about showing that \( \overline{ p } \) is continuous.
        Let \( V \subset Y \) be open.
        Our goal is to show that \( \overline{ p }^{ -1 } ( V ) \) is open
        in \( X^{ * } \) in the quotient topology induced by \( \pi \).
        This means that we need to show that \( \pi^{ -1 } (
        \overline{ p }^{ -1 } ( V ) ) \) is open in \( X \).
        Notice that
        \begin{equation*}
            \pi^{ -1 } ( \overline{ p }^{ -1 } ( V ) )
            =
            ( \overline{ p } \circ \pi )^{ -1 } ( V )
            =
            p^{ -1 } ( V )
        \end{equation*}
        Since we know that \( p \) is continuous, we find that \( p^{ -1 } 
        ( V ) \) is open in \( X \), which means that 
        \( \pi^{ -1 } ( \overline{ p }^{ -1 } ( V ) ) \) is open in \( X \).
        Thus, we have that \( \overline{ p } \) is continuous.

        \baseSkip

        We now need to show that \( \overline{ p }^{ -1 } \), which exists
        since \( \overline{ p } \) is a bijection, is continuous.
        Let \( W \subset X^{ * } \) be open.
        We want to show that \( ( \overline{ p }^{ -1 } )^{ -1 } ( W ) \) is 
        open in \( Y \) under the quotient topology induced by \( p \).
        Now because the inverse of a function is unique, we see that 
        \( ( \overline{ p }^{ -1 } )^{ -1 } ( W ) = \overline{ p } ( W ) \).
        Thus, it suffices to check that \( \overline{ p } ( W ) \) is open
        in \( Y \) under the quotient topology induced by \( p \).

        \baseSkip 

        To do so, we shall utilize the fact that \( p \) is a quotient map;
        because \( p \) is a quotient map, we have that 
        \( \overline{ p } ( W ) \) is open in \( Y \) if and only if 
        \( p^{ -1 } ( \overline{ p } ( W ) ) \) is open in \( X \).
        Notice that
        \begin{equation*}
            p = \overline{ p } \circ \pi
            \iff 
            \pi = \overline{ p }^{ -1 } \circ p
            \implies 
            p^{ -1 } ( \overline{ p } ( W ) )
            =
            ( \overline{ p }^{ -1 } \circ p )^{ -1 } ( W )
            =
            \pi^{ -1 } ( W )
        \end{equation*}
        Since \( \pi \) itself is a quotient map (namely, continuous), we have 
        that \( W \) being open in \( X^{ * } \) implies that 
        \( \pi^{ -1 } ( W ) \) is open in \( X \).
        Thus, we have that \( p^{ -1 } ( \overline{ p } ( W ) ) \) is open in 
        \( X \), which results in \( \overline{ p } ( W ) \) to be
        open in \( Y \).
        With this, we see that \( \overline{ p }^{ -1 } \) is continuous.

        \baseSkip

        With all three conditions proved, we have that \( \overline{ p } \) is 
        a homeomorphism between \( X \) and \( Y \)
    \end{proofBox}
\end{thmBox}