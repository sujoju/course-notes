\begin{defBox}{Open Covering}[def:26_covering]
    A collection \( \mathcal{A} \) of subsets of a space \( X \) is said to 
    \textbf{cover} \( X \), or be a \textbf{covering} of \( X \), if the 
    union of the elements of \( \mathcal{A} \) is equal to \( X \).
    It is called an \textbf{open covering} of \( X \) if its elements are
    open subsets of \( X \).

    \baseSkip

    For subspaces, we have the following: if \( Y \) is a subspace of \( X \),
    a collection \( \mathcal{A} \) of subsets of \( X \) is said to 
    \textbf{cover} \( Y \) if the union of its elements \textit{contains} 
    \( Y \).
\end{defBox}

\begin{defBox}{Compact Spaces}[def:26_compact]
    A space \( X \) is said to be compact if every open covering 
    \( \mathcal{A} \) of \( X \) contains a finite subcollection that also 
    covers \( X \).
\end{defBox}

\begin{defBox}{Tubular Neighborhoods}[def:26_tube_nbhd]
    Let \( X \times Y \) be a product of two topological spaces.
    Then a \textbf{tubular neighborhood} of a \textbf{slice} \( X \times
    \{ y \} \) is a set of the form \( X \times V \), where \( V \) is an 
    open neighborhood of \( y \).
\end{defBox}

\begin{defBox}{Finite Intersection Property}[def:26_finite_intersection]
    A collection \( \mathcal{C} \) of subsets of \( X \) is said to have the 
    \textbf{finite intersection property} if for every finite subcollection
    \begin{equation*}
        \{ C_{ 1 } , \ldots , C_{ n } \}
    \end{equation*}
    of \( \mathcal{C} \), the intersection 
    \( C_{ 1 } \cap \ldots \cap C_{ n } \) is nonempty.
\end{defBox}