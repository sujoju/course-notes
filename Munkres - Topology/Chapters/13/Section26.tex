\section{Learning Objectives}

\begin{itemize}
    \item What is an open covering and how does it relate to the definition
        of compactness?
    \item What is a tubular neighborhood?
    \item What is the finite intersection property?
    \item How does compactness behave with subsets? 
        [\hyperlink{lem:26.1}{Lemma 26.1}]
    \item Are all subspaces of a compact set compact? If not, then what 
        condition can we put on these subspaces so that they always are?
        [\hyperlink{thm:26.2}{Theorem 26.2}]
    \item Are all compact subspaces closed? If not, then what condition can
        we put on the overall space so that they always are?
        [\hyperlink{thm:26.3}{Theorem 26.3}]
    \item Why can we always find disjoint neighborhoods of compact subspaces
        in a Hausdorff space? Why do we need these subspaces to be compact?
        [\hyperlink{thm:26.4}{Theorem 26.4}]
    \item Is the image of a compact space always compact?
        [\hyperlink{thm:26.5}{Theorem 26.5}]
    \item If the domain of a function is compact and the codomain is Hausdorff,
        then is it required to show that the function's inverse is continuous
        when showing whether or not the function is a homeomorphism.
        [\hyperlink{thm:26.6}{Theorem 26.6}]
    \item Is the product of finitely many compact spaces compact?
        How about for arbitrary products?
        [\hyperlink{thm:26.7}{Theorem 26.7}]
    \item What is the Tube Lemma? [\hyperlink{lem:26.8}{Lemma 26.8}]
\end{itemize}

\section{Definitions}

\subimport{}{Section26_Def.tex}

\section{Theorems}

\subimport{}{Section26_Thm.tex}

\section{Examples}

\subimport{}{Section26_Eg.tex}

\section{Discussion}

\subimport{}{Section26_Discussion.tex}