\begin{remarkBox}{[\hyperlink{lem:26.1}{Lemma 26.1}]}
    This lemma states \( A \) being compact in \( X \) means that \( A \) is 
    also compact in the subspace topology.
\end{remarkBox}

\begin{remarkBox}{Cofinite Topology}
    This topology is not Hausdorff, and we see that there are sets that are
    not closed, but compact.
\end{remarkBox}

\begin{remarkBox}{[\hyperlink{lem:26.4}{Lemma 26.4}]}
    This lemma tells us that Hausdorffness automatically gives us a stronger
    condition: we can separate points from compact sets.
    In fact, this lemma can be extended (as we did with 
    [\hyperlink{thm:26.4}{Theorem 26.4}]) to say that disjoint subsets of 
    a compact set can be separated.
\end{remarkBox}

\begin{remarkBox}{[\hyperlink{thm:26.7}{Theorem 26.7}]}
    We note that both spaces being compact was important: \( X \) being compact
    allowed for us to find an open tubular neighborhood around each slice
    \( X \times \{ y \} \) for all \( y \in Y \) via
    [\hyperlink{lem:26.8}{Lemma 26.8}].
    Notice that each of these tubular neighborhoods can be covered by a finite
    subcollection of \( \mathcal{A} \).
    Why does compactness of \( Y \) matter then?
    Notice that we have an infinite amount of tubular neighborhoods, and an
    infinite union of finite sets may not be finite.
    Thus, \( Y \) being compact allows for us to see that there are only a 
    finite amount of tubular neighborhoods needed to cover \( Y \) (really,
    \( X \times Y \)); these tubular neighborhoods are each covered by a finite
    subcollection of \( \mathcal{A} \), and a finite union of finite sets will
    remain finite.
    Hence, we are left with a finite subcollection of \( \mathcal{A} \) that
    also covers \( X \times Y \).

    \baseRule
    
    We have shown in this theorem that the product of finitely many compact 
    spaces is compact.
    It can be shown as well that arbitrary products of compact spaces are
    compact as well; however, the proof is much more difficult and has its 
    own name -- \textit{Tychonoff's Theorem}.
\end{remarkBox}

\begin{remarkBox}{[\hyperlink{thm:26.9}{Theorem 26.9}]}
    A special case of this theorem occurs when we have a \textbf{nested
    sequence}
    \begin{equation*}
        C_{ 1 } \supset C_{ 2 } \supset \ldots \supset 
        C_{ n } \supset C_{ n + 1 } \supset \ldots
    \end{equation*}
    of closed sets in a compact space \( X \).
    If each of the sets \( C_{ n } \) is nonempty, then the collection 
    \( \mathcal{C} = \{ C_{ n } \}_{ n \in \mathbb{Z}_{ + } } \) automatically
    has the finite intersection property.
    Then the intersection
    \begin{equation*}
        \bigcap_{ n \in \mathbb{Z}_{ + } } C_{ n }
    \end{equation*}
    is nonempty.
\end{remarkBox}

\begin{remarkBox}{Compactness in a Space vs. Period}
    If \( A \) is a subset of a topological space \( X \), then there are two
    ways we have talked about \( A \) being compact.

    \baseSkip

    \wrapBox{First}
    We can talk about \( A \) being \textit{compact in} \( X \).
    This has to do with open covers involving subsets open in \( X \).

    \baseSkip 
    \wrapBox{Second} 
    We can talk about \( A \) being \textit{compact as a topological space} in
    its own right equipped with the subspace topology.
    This has to do with open covers involving subsets open in \( A \).

    \baseSkip

    Importantly, these two notions are the \textit{same} (as shown in 
    [\hyperlink{lem:26.1}{Lemma 26.1}])!
    So, when we say that "\( A \) is compact in \( X \)", we might as well say
    that "\( A \) is compact with its subspace topology".
    The point is that the containing \( X \) does not really matter beyond the 
    subspace topology that it gives.

    \baseSkip

    However, the property of being closed \textit{does} depend on the containing
    space. For example, let's compare:
    the interval \( ( 0, 1 ) \) is closed in \( ( 0, 1 ) \), but it is 
    \textit{not} closed in \( \mathbb{R} \).
    Closedness in \( ( 0, 1 ) \) does not guarantee closedness in 
    \( \mathbb{R} \).
    The interval \( [ \frac{ 1 }{ 3 }, \frac{ 2 }{ 3 } ] \) is compact in 
    \( ( 0, 1 ) \), so it \textit{must} also be compact in \( \mathbb{R} \).
    Why? Because compactness is an \textit{inherent} property of 
    \( [ \frac{ 1 }{ 3 }, \frac{ 2 }{ 3 } ] \). 
    This is what makes compactness so special.
\end{remarkBox}

\begin{remarkBox}{More Examples}
    More examples can be found in \S 26 of Munkres.
\end{remarkBox}