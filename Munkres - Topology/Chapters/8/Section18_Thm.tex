\begin{thmBox}{Invertible Functions}[thm:18.0-1]
    Let \( f: X \rightarrow Y \) be a function between two sets.
    We say that \( f \) is a bijection if and only if it is invertible,
    meaning that it has an inverse.

    \baseRule

    \begin{proofBox}
        We shall prove this by first showing that \( f \) is injective if and 
        only if \( f \) admits a left inverse \( g: Y \rightarrow X \) where
        \( g \circ f = i_{ X } \). 

        \baseSkip

        Recall that \( f \) being injective tells us that we have for all 
        \( x, y \in X \)
        \begin{equation*}
            f ( x ) = f ( y ) \implies x = y
        \end{equation*}
        If \( g \) is the left inverse of \( f \), then we have that
        \begin{equation*}
            g ( f ( x ) ) = x
            \quad 
            \forall x \in X   
        \end{equation*}
        If we assume that \( f \) is injective, then we want to show that 
        \( g \) is indeed the left inverse of \( f \).
        To do so, it just amounts to showing that \( g \) is indeed a 
        well-defined function.
        We see that 
        \begin{equation*}
            f ( x_{ 1 } ) = f ( x_{ 2 } )
            \implies
            x_{ 1 } = x_{ 2 }
            \iff
            g ( f ( x_{ 1 } ) ) = g ( f ( x_{ 2 } ) )
        \end{equation*}
        meaning that \( g \) is indeed a left inverse of \( f \),
        given that \( f \) is injective.

        \baseSkip

        Now, if \( g \) is the left inverse of \( f \), then we want to show
        that \( f \) is injective.
        Since \( g \) is well defined, we see that
        \begin{equation*}
            f ( x_{ 1 } ) = f ( x_{ 2 } )
            \implies
            g ( f ( x_{ 1 } ) ) = g ( f ( x_{ 2 } ) )
            \iff 
            x_{ 1 } = x_{ 2 }
        \end{equation*}
        which tells us that \( f \) is indeed injective.

        \baseSkip

        We shall now prove that \( f \) is surjective if and 
        only if \( f \) admits a right inverse \( g: Y \rightarrow X \) where
        \( f \circ g = i_{ Y } \). 

        \baseSkip

        Recall that \( f \) being surjective tells us that we have for all 
        \( y \in Y \), there exists some \( x \in X \) such that
        \( f ( x ) = y \).
        Let us define \( g: Y \rightarrow X \) to be the function which maps
        each \( y \in Y \) to such \( x \in X \); if there is more than one
        \( x \), then the function \( g \) maps \( y \) to one of them
        chosen in an arbitrary way -- this excludes the possibility that 
        \( g \) maps \( y \) to two distinct values, in which it wouldn't be
        a function.
        It follows that
        \begin{equation*}
            f ( g ( y ) ) = f ( x ) = y
            \quad 
            \forall y \in Y   
        \end{equation*}
        which tells us that \( g \) is indeed the right inverse of \( f \).

        \baseSkip

        Now, if \( g \) is the right inverse of \( f \), then we want to show
        that \( f \) is surjective.
        Let \( y \in Y \) be given.
        We want to show that there exists some \( x \in X \) such that 
        \( y = f ( x ) \).
        Since \( g \) is well defined, we see that \( g ( y ) = x \) for some
        \( x \in X \).
        Furthermore, since \( g \) is the right inverse of \( f \), we have that
        \begin{equation*}
            y = f ( g ( y ) ) = f ( x )
        \end{equation*}
        which tells us that \( f \) is indeed surjective.

        \baseSkip

        Putting everything together, we see that \( f \) is bijective if and 
        only if \( f \) has both a left and right inverse -- that is, \( f \)
        is bijective if and only if \( f \) has an inverse.
    \end{proofBox}
\end{thmBox}

\begin{thmBox}{Equivalent Definition of Continuity}[thm:18.0-2]
    Let \( f: X \rightarrow Y \) be a function between topological spaces. 
    Then the following definitions are equivalent.
    When either definition is satisfied, we simply say that \( f \) is 
    \textbf{continuous}.
    \begin{enumerate}[label = (\alph*)]
        \item at each point: \( f \) is continuous at each point \( x \) in its 
            domain
        \item inverse images of open sets: \( f^{ -1 } ( V ) \) is open in 
            \( X \) whenever \( V \) is open in \( Y \).
    \end{enumerate}

    \baseRule

    \begin{proofBox}*
        \wrapBox{(a) \( \implies \) (b)}

        Let \( f \) be continuous at each \( x \in X \).
        Let \( V \subset Y \) be open.
        Our goal is to show that \( f^{ -1 } ( V ) \) is open in \( X \).
        This means that for each \( x \in f^{ -1 } ( V ) \), we can to find some open neighborhood \( U \) of \( x \) so that 
        \( U \subset f^{ -1 } ( V ) \).
        
        \baseSkip

        Let \( x \in f^{ -1 } ( V ) \) be given.
        Because we know that \( V \) is an open neighborhood of \( f ( x ) \) 
        and \( f \) is continuous at \( x \), there exists an open neighborhood 
        \( U \) so that \( f ( U ) \subset V \).
        By definition, we have that for all \( x \in U \), we get 
        \( f ( x ) \in V \).
        Notice that 
        \begin{equation*}
            f^{ -1 } ( V )
            =
            \{ x \in X \mid f ( x ) \in V \}
        \end{equation*}
        which tells us that all of \( U \) is included in \( f^{ -1 } ( V ) \)
        -- that is, we have that \( U \subset f^{ -1 } ( V ) \).
        This completes this direction.

        \baseSkip

        \wrapBox{(b) \( \implies \) (a)}

        Now, let \( f^{ -1 } ( V ) \) be open whenever \( V \) is open.
        Let \( x \in X \) be given.
        Our goal is to show that \( f \) is continuous at \( x \).

        \baseSkip

        Let \( V \) be an open neighborhood of \( f ( x ) \).
        Thus, it follows that \( f^{ -1 } ( V ) \) is open by assumption, and
        it also contains \( x \) -- that is, \( f^{ -1 } ( V ) \) is an
        open neighborhood of \( x \) in \( X \).
        If we pick \( U = f^{ -1 } ( V ) \), then we see that 
        \begin{equation*}
            f ( U )
            =
            \{ f ( x ) \in Y \mid x \in U \}
            =
            \{ f ( x ) \in Y \mid x \in f^{ -1 } ( V ) \}
        \end{equation*}
        which tells us that \( f ( x ) \in V \) for all \( x \in U \).
        Thus, we have that \( f ( U ) \subset V \).
    \end{proofBox}
\end{thmBox}

\begin{thmBox}{18.1}[thm:18.1]
    Let \( X \) and \( Y \) be topological spaces; let \( f: X \rightarrow Y \).
    Then the following are equivalent:
    \begin{enumerate}
        \item \( f \) is continuous.
        \item For every subset \( A \) of \( X \), we have that 
            \( f ( \overline{ A } ) \subset \overline{ f ( A ) } \).
        \item For every closed set \( B \) of \( Y \), the set \( f^{ -1 } 
            ( B ) \) is closed in \( X \).
    \end{enumerate}

    \baseRule

    \begin{proofBox}
        We shall show that 
        \( ( 1 ) \implies ( 2 ) \implies ( 3 ) \implies ( 1 ) \).

        \baseSkip 

        \wrapBox{\( ( 1 ) \implies ( 2 ) \)}

        Let \( x \in \mathrm{Cl} \ A \) be given.
        Then \( f ( x ) \) represents some arbitrary element of 
        \( f ( \mathrm{Cl} \ A ) \).
        Our goal is to show that \( f ( x ) \in \mathrm{Cl} \ f ( A ) \), which 
        is equivalent to showing that every open neighborhood that contains 
        \( f ( x ) \) intersects \( f ( A ) \) non-trivially.

        \baseSkip

        Let \( V \) be any open neighborhood of \( f ( x ) \).
        Since we know that \( f \) is continuous, \( V \) being open in \( Y \) 
        implies that \( f^{ -1 } ( V ) \) is also open in \( X \).
        Furthermore, we see that \( V \) containing \( f ( x ) \) implies as 
        well that \( f^{ -1 } ( V ) \) contains \( x \) -- thus, 
        \( f^{ -1 } ( V ) \) is an open neighborhood of \( x \).
        Since we assumed \( x \in \mathrm{Cl} \ A \), it follows that 
        \( f^{ -1 } ( V ) \cap A \neq \emptyset \).
        Notice that
        \begin{equation*}
            f^{ -1 } ( V ) \cap A
            =
            \{ x \in X \mid x \in A \text{ and } x \in f^{ -1 } ( V ) \}
        \end{equation*}
        which tells us that for any \( x \in f^{ -1 } ( V ) \cap A \), it must 
        be the case that \( f ( x ) \in V \cap f ( A ) \).
        Now since \( f^{ -1 } ( V ) \cap A \neq \emptyset \), we have that
        \( V \cap f ( A ) \neq \emptyset \).
        Hence, we can conclude that \( f ( x ) \in \mathrm{Cl} \ f ( A ) \),
        which results in \( f ( \mathrm{Cl} \ A ) \subset 
        \mathrm{Cl} \ f ( A ) \).

        \baseSkip 

        \wrapBox{\( ( 2 ) \implies ( 3 ) \)}

        Let \( B \) be a closed set of \( Y \) and let \( A = f^{ -1 } ( B ) \).
        We want to show that \( A \) is closed in \( X \).
        In this case, it suffices to show that \( A = \overline{ A } \).
        By elementary set theory, we have that 
        \begin{equation*}
            f ( A )
            =
            f ( f^{ -1 } ( B ) )
            \subset 
            B
        \end{equation*}
        Therefore, we see that if \( x \in \overline{ A } \), then 
        \begin{equation*}
            f ( x ) \in f ( \overline{ A } ) 
            \subset 
            \overline{ f ( A ) }
            \subset 
            \overline{ B } = B
        \end{equation*}
        Where the first inclusion results from our assumption, and the second 
        inclusion results from the fact that \( A \subset B \) implies 
        \( \overline{ A } \subset \overline{ B } \).
        From here, we note that \( f ( x ) \in B \) is equivalent to saying that
        \( x \in f^{ -1 } ( B ) = A \). 
        Thus, we are left with \( \overline{ A } \subset A \).
        Since \( A \subset \overline{ A } \) by definition, it follows that 
        \( A = \overline{ A } \), which tells us that \( f^{ -1 } ( B ) \) is
        closed.

        \baseSkip 

        \wrapBox{\( ( 3 ) \implies ( 1 ) \)}

        Let \( B \) be a closed set of \( Y \).
        We are given that \( f^{ -1 } ( B ) \) is closed in \( X \) and want to
        show that \( f \) is continuous -- that is, for every open set 
        \( V \) of \( Y \), we have \( f^{ -1 } ( V ) \) to be open in \( X \).

        \baseSkip

        It follows that \( Y \setminus V \) is closed in \( Y \), which results 
        in \( f^{ -1 } ( Y \setminus V ) \) by our assumption.
        Furthermore, we have that 
        \begin{equation*}
            f^{ -1 } ( Y \setminus V ) = X \setminus f^{ -1 } ( V )
        \end{equation*}
        which results in \( f^{ -1 } ( V ) \) to be open in \( X \).
        Thus, we have that \( f \) is continuous.

        \baseSkip 

        \wrapBox{BONUS: \( ( 1 ) \implies ( 3 ) \)}

        Let \( f \) be continuous and \( B \subset Y \) be a closed set.
        It follows that \( Y \setminus B \) is open.
        Since \( f \) is continuous, we have that 
        \( f^{ -1 } ( Y \setminus B ) \) is open as well.
        Notice that 
        \begin{equation*}
            X \setminus f^{ -1 } ( B ) = f^{ -1 } ( Y \setminus X )
        \end{equation*}
        which results in \( f^{ -1 } ( B ) \) is closed in \( X \).
    \end{proofBox}
\end{thmBox}

\begin{thmBox}{Equivalent Definition of Homeomorphism}[thm:18_equiv_hom_def]
    Let \( X \) and \( Y \) be topological spaces; let \( f : X \rightarrow Y \)
    be a bijection.
    Then \( f \) is a homeomorphism if and only if we have \( f ( U ) \) is open
    if and only if \( U \) is open.
    That is, another way to defined a homeomorphism is to say that it is a 
    bijective correspondence \( f: X \rightarrow Y \) such that \( f ( U ) \) 
    is open if and only if \( U \) is open.

    \baseRule

    \begin{proofBox}

    \end{proofBox}
\end{thmBox}

\begin{thmBox}{18.2 (Rules for Constructing Continuous Functions)}[thm:18.2]
    Let \( X, Y, \) and \( Z \) be topological spaces.
    \begin{enumerate}[label = (\alph*)]
        \item (Constant Function) If \( f: X \rightarrow Y \) maps all of 
            \( X \) into the single point \( b \) of \( Y \) -- that is, 
            \( f ( x ) = b \) for all \( x \in X \), then \( f \) is continuous.
        \item (Inclusion) If \( A \) is a subspace of \( X \) (equipped with 
            the subspace topology), the inclusion
            function \( i_{ A }: A \rightarrow X \), given by 
            \( i_{ A } ( x ) = x \) for all \( x \in A \), is continuous.
        \item (Composites) If \( f: X \rightarrow Y \) and 
            \( g: Y \rightarrow Z \) are continuous, then the map
            \( g \circ f: X \rightarrow Z \), defined by \( ( g \circ f ) ( x )
            = g ( f( x ) ) \) for all \( x \in X \), is continuous.
        \item (Restricting the Domain) If \( f: X \rightarrow Y \) is 
            continuous, and if \( A \) is a subspace of \( X \), then the 
            restricted function \( f \vert_{ A }: A \rightarrow Y \) is 
            continuous.
        \item (Restricting or Expanding the Range) Let \( f: X \rightarrow Y \)
            be continuous. If \( Z \) is a subspace of \( Y \) containing the 
            image set \( f ( X ) \), then the function \( g: X \rightarrow Z \)
            obtained by restricting the range of \( f \) is continuous.
            If \( Z \) is a space having \( Y \) as a subspace, then the 
            function \( h: X \rightarrow Z \) obtained by expanding the range of
            \( f \) is continuous.
        \item (Local Formulation of Continuity) The map \( f: X \rightarrow Y \)
            is continuous if \( X \) can be written as the union of open sets 
            \( U_{ i } \) such that \( f \vert_{ U_{ i } } \) is continuous for 
            each \( i \in I \), where \( I \) is some index set.
    \end{enumerate}

    \baseRule

    \begin{proofBox}*
        \wrapBox{a}

        Let \( V \subset Y \) be open. 
        We want to show that \( f^{ -1 } ( V ) \) is open in \( X \). 
        Notice that if \( V \) contains \( b \), then we end up getting that 
        \( f^{ -1 } ( V ) = X \); however, if \( V \) does not contain \( b \), 
        then \( f^{ -1 } ( V ) = \emptyset \).
        In either case, we have that \( f^{ -1 } ( V ) \) is open since both 
        \( X \) and \( \emptyset \) are trivially open.
        
        \baseSkip 

        \wrapBox{b}

        Let \( U \subset X \) be open. 
        We want to show that \( i_{ A }^{ -1 } ( U ) \) is open in \( A \) in
        the subspace topology.
        Notice that \( i_{ A }^{ -1 } ( U ) = U \cap A \), and since \( A \) is 
        equipped with the subspace topology, we see that \( U \cap A \) is open
        in \( A \) by definition since \( U \) is open in \( X \).
        Thus, \( i_{ A }^{ -1 } ( U ) \) is open in \( A \) in
        the subspace topology.

        \baseSkip

        \wrapBox{c}

        Let \( W \subset Z \) be open.
        Our goal is to show that \( ( g \circ f )^{ -1 } ( W ) \) is open in 
        \( X \).
        Notice that \( ( g \circ f )^{ -1 } ( W ) = 
        f^{ -1 } ( g^{ -1 } ( W ) ) \).
        Since we are given that \( W \) is open in \( Z \), we have that 
        \( g^{ -1 } ( W ) \) is open in \( Y \) because \( g \) is continuous.
        Furthermore, since \( f \) is continuous, we see that 
        \( f^{ -1 } ( g^{ -1 } ( W ) ) \) must be open as well.
        Therefore, we see that \( ( g \circ f ) \) is continuous.

        \baseSkip

        \wrapBox{d}

        Notice that \( f \vert A = f \circ i_{ A } \), and since both are 
        continuous, their composition is continuous as well.

        \baseSkip

        \wrapBox{e}

        \textbf{Restricting range}:
        Let \( V \cap Z \) be given where \( V \) is open in \( Y \); notice 
        that \( V \cap Z \) is an arbitrary open set in \( Z \) under the 
        subspace topology.
        We want to show that \( g \) is continuous, which means that we need to
        show that \( g^{ -1 } ( V \cap Z ) \) is open in \( X \).   
        Notice that \( Z \) is not necessarily open in \( Y \), which 
        could pose problems for showing that \( g^{ -1 } ( V \cap Z ) \) is 
        open in \( X \).

        \baseSkip

        Let us look at \( g^{ -1 } ( V \cap Z ) \).
        This is equivalent to saying that \( g^{ -1 } ( V ) \cap 
        g^{ -1 } ( Z ) \).
        Since \( Z \) is defined to be a subspace of \( Y \) containing the 
        image set \( f ( X ) \), it follows that \( g^{ -1 } ( Z ) = X \).
        Thus, we have that \( g^{ -1 } ( V ) \cap g^{ -1 } ( Z ) = 
        g^{ -1 } ( V ) \cap X = g^{ -1 } ( V ) \).
        From here, we see the following:
        \begin{equation*}
            \begin{aligned}
                g^{ -1 } ( V )
                &= 
                \{ x \in X \mid g ( x ) \in V \}
                \\
                f^{ -1 } ( V )
                &= 
                \{ x \in X \mid f ( x ) \in V \}
            \end{aligned}
        \end{equation*}
        Since \( g \) is obtained by restricting the range of \( f \), we see 
        that \( g ( x ) = f ( x ) \) for all \( x \in X \), which results in 
        \( g^{ -1 } ( V ) = f^{ -1 } ( V ) \).
        Since \( f \) is continuous, it follows that \( f^{ -1 } ( V ) \) is 
        open, which further results in \( g^{ -1 } ( V ) \) to be open as well.
        Thus, we have that \( g^{ -1 } ( V \cap Z ) \) is open in \( X \).
        Hence \( g \) is continuous.

        \baseSkip

        \textbf{Expanding range}:
        Now to show that \( h: X \rightarrow Z \) is continuous if \( Z \) has 
        \( Y \) as a subspace, we note that \( h \) is the composite of the 
        map \( f: X \rightarrow Y \) and \( i_{ Y }: Y \rightarrow Z \).
        We have already shown that the composition of continuous function are 
        continuous, which results in \( h \) to be continuous.

        \baseSkip

        \wrapBox{f}

        By hypothesis, we can write \( X \) as a union of open sets 
        \( U_{ i } \), such that \( f \vert_{ U_{ i } } \) is continuous for 
        each \( i \in I \).
        We now let \( V \) be an open set in \( Y \).
        We want to show that \( f^{ -1 } ( V ) \) is  open.
        Notice that \( f \vert_{ U_{ i } } \) is constructed by restricting the
        function \( f \) to \( U_{ i } \), meaning that 
        \begin{equation*}
            ( f \vert_{ U_{ i } } )^{ -1 } ( V )
            =
            \{ x \in U_{ i } \mid f ( x ) \in V \}
            =
            f^{ -1 } ( V ) \cap U_{ i }
        \end{equation*}
        since all the expression above represent the set of those points \( x \)
        lying in \( U_{ i } \) for which \( f ( x ) \in V \).
        Because we are given \( f \vert_{ U_{ i } } \) to be continuous,
        we have that \( f^{ -1 } ( V ) \cap U_{ i } \) is open in \( U_{ i } \),
        hence open in \( X \).
        Let us now take a look \( f^{ -1 } ( V ) \) more closely knowing that
        \( X \) can be written as a union of open set \( U_{ i } \):
        \begin{equation*}
            \begin{aligned}
                f^{ -1 } ( V )
                &=
                \{ x \in X \mid f ( x ) \in V \}
                \\
                &=
                \{ x \in \bigcup_{ i \in I } U_{ i } \mid f ( x ) \in V \}
                \\
                &=
                \{ x \in X \mid x \in U_{ i } \text{ and } f ( x ) \in V 
                \text{ for at least one } i \in I \}
                \\
                &=
                \bigcup_{ i \in I } ( f^{ -1 } ( V ) \cap U_{ i } )
            \end{aligned}
        \end{equation*}
        Since each \( f^{ -1 } ( V ) \cap U_{ i } \) is open in \( X \) for all
        \( i \in I \), and arbitrary unions of open sets are still open, we have
        that \( f^{ -1 } ( V ) \) is open.
        Thus, \( f \) is continuous.
    \end{proofBox}
\end{thmBox}

\begin{thmBox}{18.3 (The Pasting Lemma)}[thm:18.3]
    Let \( X = A \cup B \), where \( A \) and \( B \) are closed in \( X \).
    Let \( f: A \rightarrow Y \) and \( g: B \rightarrow Y \) be continuous.
    If \( f ( x ) = g ( x ) \) for every \( x \in A \cap B \), then \( f \)
    and \( g \) combine to give a continuous function \( h: X \rightarrow Y \),
    defined by setting \( h ( x ) = f ( x ) \) if \( x \in A \), and 
    \( h ( x ) = g ( x ) \) if \( x \in B \).

    \baseRule

    \begin{proofBox}
        Let \( f \) and \( g \) be continuous.
        Our goal is to show that \( h \) is continuous.
        Let \( D \subset Y \) be closed.
        We want to show that \( h^{ -1 } ( D ) \) is closed as well, which 
        will tell us \( h \) is continuous.
        Notice that 
        \begin{equation*}
            h^{ -1 } ( D )
            =
            [ h^{ -1 } ( D ) \cap A ] \cup [ h^{ -1 } ( D ) \cap B ]
            =
            f^{ -1 } ( D ) \cup g^{ -1 } ( D ) 
        \end{equation*}
        We have that \( f^{ -1 } ( D ) \) and \( g^{ -1 } ( D ) \) are closed in
        \( A \) and \( B \), respectively.
        Since \( A \) and \( B \) are closed in \( X \), it follows that 
        \( f^{ -1 } ( D ) \) and \( g^{ -1 } ( D ) \) are both closed in \( X \)
        as well by transitivity of closedness.
        Hence, it follows that \( h^{ -1 } ( D ) \) is closed as well since 
        we have a finite union of closed sets, which finishes our proof.

        \baseSkip

        This argument works for finite unions, but does not work for arbitrary unions since we cannot guarantee that arbitrary unions of closed sets 
        are closed.
        Furthermore, we have that this theorem can be rephrased in terms of open
        subsets.
    \end{proofBox}
\end{thmBox}

\begin{thmBox}{Maps into Products}[thm:18.4]
    Let \( Y = \prod_{ i \in I } Y_{ i } \) be a product of topological spaces.
    Let \( f: X \rightarrow Y \) be a function given by the equation:
    \begin{equation*}
        \mathbf{f} ( t )
        =
        ( f_{ 1 } ( t ), f_{ 2 } ( t ), f_{ 3 } ( t ), \ldots )
    \end{equation*}\
    which we can further write in terms of its \textbf{component functions}
    \( f_{ i } \):
    \begin{equation*}
        \mathbf{f} = ( f_{ i } )_{ i \in I }
        \quad \mathrm{where} \quad 
        f_{ i } = \mathrm{proj}_{ i } \circ \mathbf{f}
    \end{equation*}
    We say that \( \mathbf{f} \) is continuous if and only if its component 
    functions are all continuous, assuming that the codomain has been equipped 
    with the product topology.

    \baseRule

    \begin{proofBox}*
        \wrapBox{\( \implies \)} 

        Assume that \( \mathbf{f} \) is continuous.
        We want to show that each of their component functions \( f_{ i } \) are
        continuous for all \( i \in I \).
        Indeed, we see that they are continuous since such functions are defined
        as \( f_{ i } = \mathrm{proj}_{ i } \circ \mathbf{f} \), which is the 
        composition of two continuous functions.

        \baseSkip

        \wrapBox{\( \impliedby \)}

        Assume \( f_{ i } : X \rightarrow Y \) is continuous for all 
        \( i \in I \).
        Let \( U = \prod_{ i \in I } U_{ i } \) be a basic open set in the 
        product topology.
        Our goal is to show that \( \mathbf{f} \) is continuous, which we can 
        do by showing that \( \mathbf{f}^{ -1 } ( U ) \) is open.

        \baseSkip 

        We start by noting that 
        \begin{equation*}
            f_{ i }^{ -1 }
            =
            ( \mathrm{proj}_{ i } \circ \mathbf{f} )^{ -1 }
            =
            \mathbf{f}^{ -1 } \circ \mathrm{proj}_{ i }^{ -1 }
        \end{equation*}
        Thus, it follows that 
        \begin{equation*}
            \mathbf{f}^{ -1 } ( U )
            =
            \mathbf{f}^{ -1 } \left( \prod_{ i \in I } U_{ i } \right)
            =
            \mathbf{f}^{ -1 } 
            \left( 
                \bigcap_{ i \in I } \mathrm{proj}_{ i }^{ -1 } ( U_{ i } ) 
            \right)
            =
            \bigcap_{ i \in I } 
            ( \mathbf{f}^{ -1 } \circ \mathrm{proj}_{ i }^{ -1 } ) ( U_{ i } ) 
            =
            \bigcap_{ i \in I }  f_{ i }^{ -1 } ( U_{ i } )
        \end{equation*}
        Since we are working in a product topology, it follows that 
        \begin{equation*}
            \bigcap_{ i \in I }  f_{ i }^{ -1 } ( U_{ i } )
            =
            \bigcap_{ \substack{ i \in I \\ U_{ i } \neq Y } } 
            f_{ i }^{ -1 } ( U_{ i } )
        \end{equation*}
        Since each \( f_{ i } \) is continuous for all \( i \in I \), we have
        that \( f_{ i }^{ -1 } ( U_{ i } ) \) is open for all \( i \in I \).
        Furthermore, since we only have finitely many \( U_{ i } \) such that
        \( U_{ i } \neq Y_{ i } \), it follows that 
        \( \mathbf{f}^{ -1 } ( U ) \) is 
        open since a finite intersection of open sets is still open.
        Thus, we have that \( \mathbf{f} \) is continuous.
    \end{proofBox}
\end{thmBox}

\begin{thmBox}{Continuity and Limits}[thm:cont_lim]
    Let \( f: X \rightarrow Y \) be a continuous function between topological
    spaces.
    If \( x_{ n } \rightarrow x \) is a convergent sequence in \( X \), then 
    \( f ( x_{ n } ) \rightarrow f ( x ) \) in \( Y \).

    \baseRule

    \begin{proofBox}
        Let \( x_{ n } \rightarrow x \).
        Our goal is to show that \( f ( x_{ n } ) \rightarrow f ( x ) \).
        Let \( V \) be an open neighborhood of \( f ( x ) \).
        We want to show that \( f ( x_{ n } ) \in V \) for all \( n \geq N \).
        Notice that \( f^{ -1 } ( V ) \) is an open neighborhood of \( x \)
        because \( f \) is continuous.
        Because \( x_{ n } \rightarrow x \), we have that there exists some 
        positive integer \( N \) such that \( x_{ n } \in f^{ -1 } ( V ) \) for all \( n \geq N \).
        Notice that \( x_{ n } \in f^{ -1 } ( V ) \iff f ( x_{ n } ) \in V \).
    \end{proofBox}
\end{thmBox}

\begin{thmBox}{Hausdorffness and Continuity}[thm:haus_cont]
    Let \( X \) be a topological space and let \( A \) be a subset of \( X \) so
    that \( \mathrm{Cl} \ A = X \).
    Let \( f: X \rightarrow Y \) and \( g: X \rightarrow Y \) be a pair of 
    continuous functions into a topological space \( Y \) that 
    \textbf{agree} on \( A \), meaning that \( f ( x ) = g ( x ) \) for all 
    \( x \in A \).

    \baseSkip 

    If \( Y \) is Hausdorff, then \( f \) and \( g \) must agree on all of
    \( X \).

    \baseRule

    \begin{proofBox}
        Towards a contradiction (TAC), assume there exists some \( p \in X \) so
        that \( f ( p ) \neq g ( p ) \).
        Because \( Y \) is Hausdorff, there exists disjoint open neighborhoods,
        say \( U  \) containing \( f ( p ) \) and \( V \) containing \( g ( p ) \).
        Notice that both \( f^{ -1 } ( U ) \) and \( g^{ -1 } ( V ) \) are  open neighborhoods of \( p \).

        \baseSkip 

        If we take their intersection, then we still end up with an open 
        neighborhood of \( p \) -- that is, 
        \( f^{ -1 } ( U ) \cap g^{ -1 } ( V ) \) is still an open neighborhood
        of \( p \).
        Since \( p \in \mathrm{Cl} \ A \), this implies that 
        \( A \cap ( f^{ -1 } ( U ) \cap g^{ -1 } ( V ) ) \neq \emptyset \).
        Let us pick any 
        \( x \in A \cap ( f^{ -1 } ( U ) \cap g^{ -1 } ( V ) ) \).
        This means that \( f ( x ) \in U \) and \( g ( x ) \in V \).
        Because \( x \in A \), we have by assumption that 
        \( f ( x ) = g( x ) \), which results in \( U \cap V \neq \emptyset \).
        However, we have that \( U \cap V = \emptyset \), so we reach a 
        contradiction.

        \baseSkip
        
        Therefore, \( f \) and \( g \) must agree on all of \( X \).
    \end{proofBox}
\end{thmBox}