\begin{defBox}{Image of a Function}[def:18_img_func]
    If \( A \) is a subset of the domain \( X \), then its image (denoted as 
    \( f ( A ) \)) is 
    \begin{equation*}
        f ( A )
        =
        \{ f ( x ) \in Y \mid x \in A \}
    \end{equation*}
    In words, \( f ( A ) \) is the set of all outputs of points from the 
    specified subset \( A \).
\end{defBox}

\begin{defBox}{Inverse Image of a Function}[def:18_inv_img_func]
    If \( f: X \rightarrow Y \) is a function and \( B \) is a subset of the 
    codomain \( Y \), then its \textbf{inverse image} is 
    \begin{equation*}
        f^{ -1 } ( B )
        =
        \{ x \in X \mid f ( x ) \in B \}
    \end{equation*}
    In words, \( f^{ -1 } ( B ) \) is the set of all points in the domain 
    \( X \) whose outputs lie in the specified subset \( B \) of the codomain.
\end{defBox}

\begin{defBox}{Injection, Surjection, Bijection}[def:18_inj_sur_bij]
    Let \( f: X \rightarrow Y \) be a function between two sets \( X \) and 
    \( Y \).
    We say that \( f \) is an \textbf{injection} if each output corresponds
    to \textit{at most} one input. Symbolically, 
    \begin{equation*}
        \forall x, y \in X,
        f ( a ) = f ( b ) \implies a = b
        \iff 
        \forall x, y \in X,
        a \neq b \implies f ( a ) \neq f ( b )
    \end{equation*}
    We say that \( f \) is a \textbf{surjection} if each element in the codomain
    is an output of the function. Symbolically, 
    \begin{equation*}
        \forall y \in Y, 
        \exists x \in X
        \text{ such that }
        f ( x ) = y
    \end{equation*}
    We say that \( f \) is a \textbf{bijection} if it is both an injection and 
    surjection.
\end{defBox}

\begin{defBox}{Inverse Function}[def:18_inv_func]
    An inverse to a function \( f: X \rightarrow Y \) is a function
    \( f^{ -1 }: Y \rightarrow X \) so that \( f \) and \( f^{ -1 } \) undo
    one another in the sense that 
    \begin{equation*}
        f \circ f^{ -1 } = i_{ Y } 
        \quad \mathrm{and} \quad 
        f^{ -1 } \circ f = i_{ X }
    \end{equation*}
    Where \( i_{ Y }: Y \rightarrow Y \) and \( i_{ X }: X \rightarrow X \) are the identity functions on \( Y \) and \( X \), respectively.
    On the level of elements, the two equalities above are the same as saying 
    that
    \begin{equation*}
        f ( f^{ -1 } ( y ) ) = y
        \quad \forall y \in Y
        \quad \mathrm{and} \quad 
        f^{ -1 } ( f ( x ) ) = x   
        \quad \forall x \in X   
    \end{equation*}
\end{defBox}

\begin{defBox}{Continuity at a Point}[def:18_cts_pt]
    Let \( f: X \rightarrow Y \) be a function between topological spaces.
    Then \( f \) is \textbf{continuous at a given} \( x \in X \) if and only if,
    for each open neighborhood \( V \) of \( f ( x ) \), there exists an open neighborhood \( U \) of \( x \) so that \( f ( U ) \subset V \).

    \baseSkip

    If the topologies of \( X \) and \( Y \) happen to be generated by bases, then we obtain an equivalent definition by replacing "open neighborhood" 
    with "basic open neighborhood" in each occurrence above.
\end{defBox}

\begin{defBox}{Homeomorphisms}[def:18_homeomorphism]
    Let \( X \) and \( Y \) be topological spaces; let \( f: X \rightarrow Y \)
    be a bijection.
    If both the function \( f \) and the inverse function \( f^{ -1 }: Y 
    \rightarrow X \) are continuous, then \( f \) is called a 
    \textbf{homeomorphism}.

    \baseSkip

    If there is a homeomorphism between topological spaces \( X \) and \( Y \),
    then we say that \( X \) and \( Y \) are \textbf{homeomorphic}.
\end{defBox}

\begin{defBox}{Homeomorphism Invariant}[def:18_homeomorphism_invariant]
    A \textbf{homeomorphism invariant} is a property of topological spaces that 
    remains unchanged among homeomorphic spaces.
\end{defBox}

\begin{defBox}{Topological Property}[def:18_top_prop]
    Any property of \( X \) that is entirely expressed in terms of the 
    topology of \( X \) (that is, in terms of the open sets of \( X \)) yields,
    via the bijective correspondence \( f: X \rightarrow Y \), the corresponding
    property for the space \( Y \).
    Such a property of \( X \) is called the \textbf{topological property} of 
    \( X \).
\end{defBox}

\begin{defBox}{Topological Imbedding}[def:18_embedding]
    Suppose that \( f: X \rightarrow Y \) is an injective continuous map, where
    \( X \) and \( Y \) are topological spaces.
    Let \( Z \) be the image set \( f ( X ) \), considered as a subspace of 
    \( Y \); then the function \( f': X \rightarrow Z \) obtained by restricting
    the range of \( f \) is bijective.
    If \( f' \) happens to be a homeomorphism of \( X \) with \( Z \), we say
    that the map \( f: X \rightarrow Y \) is a \textbf{topological imbedding}, 
    or simply an \textbf{imbedding}, of \( X \) in \( Y \).
\end{defBox}