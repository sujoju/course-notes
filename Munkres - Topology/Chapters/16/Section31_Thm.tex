\begin{thmBox}[Lemma]{31.1}[lem:31.1]
    Let \( X \) be a topological space. 
    Let one-point sets in \( X \) be closed -- that is, let \( X \) be 
    \( T_{ 1 } \).

    \begin{enumerate}[label = (\alph*)]
        \item \( X \) is regular if and only if given a point \( x \in X \) and 
            an open neighborhood \( U \) of \( x \), there is a closed 
            neighborhood \( V \) of \( x \) such that \( V \subset U \) --
            that is, there is an open neighborhood \( V \) of \( x \) such
            that \( \overline{ V } \subset U \).
        \item \( X \) is normal if and only if given a closed set \( A \) and 
            an open set \( U \) containing \( A \), there is a closed 
            neighborhood \( V \) of \( A \) such that \( V \subset U \) -- 
            that is, there is an open set \( V \)
            containing \( A \) such that \( \overline{ V } \subset U \).
    \end{enumerate}

    \baseRule

    \begin{proofBox}
        \wrapBox{a}

        Suppose that \( X \) is regular.
        Let \( x \) and the neighborhood \( U \) of \( x \) be given.
        Let \( B = X \setminus U \); we can see that \( B \) is a closed set
        of \( X \) that is disjoint from \( x \).
        Since \( X \) is regular, we see that there exists disjoint open sets
        \( V \) and \( W \) containing \( x \) and \( B \), respectively.
        Notice that \( \overline{ V } \) is a closed neighborhood of \( x \) 
        as \( \overline{ V } \) is closed in \( X \) and contains the open 
        neighborhood \( V \) of \( x \).
        Further notice that the set \( \overline{ V } \) is disjoint from
        \( B \); indeed if any \( y \in B \) is given, then we see that 
        \( W \) is an open neighborhood of \( y \). 
        Since \( W \) and \( V \) are disjoint from each other, we find that 
        \( y \) cannot be in \( \overline{ V } \) by 
        [\hyperlink{thm:17.5}{Theorem 17.5}].
        Thus, it must be the case that \( \overline{ V } \subset U \), as 
        desired.

        \baseSkip

        Notice that \( X \setminus W \) would also work as well; indeed,
        \( X \setminus W \) is closed and contains the open neighborhood \( V \)
        of \( x \). Thus, \( X \setminus W \) is a closed neighborhood of 
        \( x \). 
        Furthermore, we see that \( X \setminus W \) is contained in \( U \)
        since \( W \) was an open neighborhood of \( X \setminus U \).

        \baseSkip

        Conversely, suppose that the point \( x \) and some closed set \( B \)
        not containing \( x \) be given.
        Our goal is to show that there exists disjoint open sets containing 
        \( x \) and \( B \), respectively.
        Let \( U = X \setminus B \); this is an open neighborhood of \( x \) 
        since \( B \) is closed and disjoint from \( x \).
        By hypothesis, we have that there is an open neighborhood \( V \) of 
        \( x \) such that \( \overline{ V } \subset U \); notice that 
        \( \overline{ V } \) is our closed neighborhood of \( x \).
        Since \( \overline{ V } = V \cup V' \), it follows that the open sets
        \( V \) and \( X \setminus \overline{ V } \) are disjoint open sets
        containing \( x \) and \( B \), respectively.
        Therefore, we have that \( X \) is regular.

        \baseSkip
        \wrapBox{b}

        This proof uses the exact same argument; we just need to replace the 
        point \( x \) with a closed set \( A \) of \( X \) throughout.
    \end{proofBox}
\end{thmBox}

\begin{thmBox}{31.2}[thm:31.2]
    \begin{enumerate}[label = (\alph*)]
        \item A subspace of a Hausdorff space is Hausdorff; a product of 
            Hausdorff spaces is Hausdorff.
        \item A subspace of a regular space is regular; a product of regular
            spaces is regular.
    \end{enumerate}

    \baseRule

    \begin{proofBox}*
        \wrapBox{a}
        This was already proved in [\hyperlink{thm:17.11}{Theorem 17.11}].

        \baseSkip
        \wrapBox{b}

        Let \( Y \) be a subspace of the regular space \( X \).
        Since \( X \) is \( T_{ 1 } \), we find that \( Y \) must be \( T_{ 1 } \)
        as well since singleton point sets of \( Y \) are still closed in \( Y \).
        Let \( x \) be a point of \( Y \) and let \( B \) be a closed subset of \( Y \) 
        disjoint from \( x \). 
        By [\hyperlink{thm:17.4}{Theorem 17.4}], we have that \( \mathrm{Cl} \ B \) in
        \( Y \) is equal to \( \overline{ B } \cap Y \), where \( \overline{ B } \)
        denotes the closure of \( B \) in \( X \).
        Now since \( B \) is closed we see that \( \mathrm{Cl} \ B = B \), which tells 
        us that 
        \begin{equation*}
            B = \overline{ B } \cap Y
        \end{equation*}
        Thus, it follows that \( x \notin \overline{ B } \) as \( B \) is disjoint from
        \( x \).
        Using the regularity of \( X \), we can choose disjoint open sets \( U \) and 
        \( V \) of \( X \) containing \( x \) and \( B \), respectively.
        Then \( U \cap Y \) and \( V \cap Y \) are disjoint open sets in \( Y \)
        containing \( x \) and \( B \), respectively.

        \baseSkip

        Alternatively, we can prove that \( Y \) is regular using 
        [\hyperlink{lem:31.1}{Lemma 31.1}]: Let \( Y \) be a subspace of the
        regular space \( X \); let \( x \in Y \) and an open neighborhood \( U \) of
        \( x \) in \( Y \) be given.
        By definition of the subspace topology, we see that \( U \) being open in
        \( Y \) means that there exists some open set \( V \) in \( X \) such that
        \( U = V \cap Y \).
        Since \( X \) is regular, we see that there exists an open neighborhood \( W \)
        of \( x \) in \( X \) such that \( \mathrm{Cl} \ W \subset V \) 
        (this is where argument
        for normality fails since a closed subset of \( Y \) may not be closed in 
        \( X \)).
        Thus, we see that \( W \cap Y \) is an open neighborhood of \( x \) in \( Y \) 
        such that
        its closure \( \mathrm{Cl} \ W \cap Y \) in \( Y \) is contained in the open 
        neighborhood \( V \cap Y \) of \( x \) in \( Y \) -- that is, 
        \( ( \mathrm{Cl} \ W \cap Y ) \subset ( V \cap Y ) \) 
        Thus, we are done.

        \baseSkip 
        
        Let \( X \) and \( Y \) be regular topological spaces.
        We want to show that \( X \times Y \) is regular.
        Notice that both \( X \) and \( Y \) being \( T_{ 1 } \) results in
        \( X \times Y \) to be \( T_{ 1 } \).
        We now let a point \( ( a, b ) \in X \times Y \) and an open neighborhood \( U \) of \( ( a, b ) \) be given.
        By definition of open neighborhood, we see that there exists a basic
        open neighborhood \( V \times W \) of \( ( a, b ) \) that is contained
        in \( U \).
        \( X \) being regular tells us that there exists a closed neighborhood
        \( C \) of \( a \) that is contained in \( V \).
        \( Y \) being regular tells us that there exists a closed neighborhood
        \( D \) of \( b \) that is contained in \( W \).
        We find that \( C \times D \) is a closed neighborhood of \( ( a, b ) \)
        that is contained in \( U \).
    \end{proofBox}
\end{thmBox}