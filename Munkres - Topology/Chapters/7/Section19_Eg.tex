\begin{egBox}{Box Topology}[eg:19.1]
    We want to show that the basis given in [\hyperlink{box_top}{box topology}]
    definition is indeed a basis.

    \baseSkip

    Let \( \{ X_{ i } \}_{ i \in I } \) be an indexed set of topological spaces
    and \( X = \prod_{ i \in I } X_{ i } \).
    We see that for each \( x \in X \), there is indeed at least one basis 
    element containing \( x \) -- \( X \) itself is a basis element by 
    definition since \( X_{ i } \) is open trivially for all \( i \in I \).
    Thus, the first condition for a basis is met.

    \baseSkip 

    As for the second condition, we want to show that if \( x \) belongs to the
    intersection of two basis elements, say \( B_{ 1 } = \prod_{ i \in I }
    U_{ i } \) and \( B_{ 2 } = \prod_{ i \in I } V_{ 1 } \), then there is a
    basis element \( B_{ 3 } \) containing \( x \) such that 
    \( B_{ 3 } \subset B_{ 1 } \cap B_{ 2 } \).
    In fact, we see that this condition follows immediately as the intersection
    of \( B_{ 1 } \) and \( B_{ 2 } \) is itself another basis element:
    \begin{equation*}
        B_{ 3 }
        =
        \left( \prod_{ i \in I } U_{ i } \right)
        \cap 
        \left( \prod_{ i \in I } V_{ i } \right)
        =
        \prod_{ i \in I } ( U_{ i } \cap V_{ i } )
    \end{equation*}
    Where \( U_{ i } \cap V_{ i } \) are both open in \( X_{ i } \) due to 
    then being a finite intersection of open sets.
\end{egBox}

\begin{egBox}{Sequences in the Box Topology}[eg:19.2]
    Let's consider the sequence \( \mathbf{x}^{ n } \) of points in
    \( \mathbb{R}^{ \omega } = \prod_{ i = 0 }^{ \infty } \mathbb{R} \) shown
    below:
    \begin{equation*}
        \mathbf{x}^{ 1 }
        =
        ( 1, 1, 1, \ldots )
        \quad
        \mathbf{x}^{ 2 }
        =
        \left( \frac{ 1 }{ 2 }, \frac{ 1 }{ 2 }, \frac{ 1 }{ 2 }, \ldots \right)
        \quad
        \mathbf{x}^{ 3 }
        =
        \left( \frac{ 1 }{ 3 }, \frac{ 1 }{ 3 }, \frac{ 1 }{ 3 }, \ldots \right)
        \quad
        \ldots
    \end{equation*}
    Seems natural that we have the convergence \( \mathbf{x}^{ n } \rightarrow
    ( 0, 0, 0, \ldots ) \) because, for each fixed index, we have a sequence 
    that converges to \( 0 \) ... Right?

    \baseSkip

    Unfortunately, \textbf{not} in the box topology.
    We can find a basic open neighborhood of \( ( 0, 0, 0, \ldots ) \) that 
    contain \textbf{none} of the \( \mathbf{x}^{ n } \) above:
    \begin{equation*}
        \prod_{ n = 1 }^{ \infty } 
        \left( - \frac{ 1 }{ n }, \frac{ 1 }{ n } \right)
    \end{equation*}
\end{egBox}

\begin{egBox}{Closure of \( \mathbb{R}^{ \infty } \)}[eg:19.3]
    We define \( \mathbb{R}^{ \infty } \subset \mathbb{R}^{ \omega } \) to 
    consist of all elements whose coordinates are eventually zero.
    Precisely,
    \begin{equation*}
        \mathbb{R}^{ \infty }
        =
        \{
            \mathbf{x} \in \mathbb{R}^{ \omega }
            \mid 
            \text{there exists index } K \text{ so that } 
            x_{ k } = 0 \text{ for all } k \geq K
        \}
    \end{equation*}
    The closure of \( \mathbb{R}^{ \infty } \) in the product topology on 
    \( \mathbb{R}^{ \omega } \), denoted as \( \mathrm{Cl} \ 
    \mathbb{R}^{ \infty } \), is all of \( \mathbb{R}^{ \omega } \).

    \baseSkip

    To show why, we start by letting \( \mathbf{x} = ( x_{ 1 }, x_{ 2 }, 
    x_{ 3 }, \ldots ) \) be a given element of \( \mathbb{R}^{ \omega } \).
    Our goal is to show that \( \mathbf{x} \in \mathrm{Cl} \
    \mathbb{R}^{ \infty } \).
    We know that [\hyperlink{thm:17.5}{Theorem 17.5}] tells us that we must 
    equivalently show that every basic open set \( U \) containing 
    \( \mathbf{x} \) intersects \( \mathbb{R}^{ \infty } \).

    \baseSkip 

    Let \( U = \prod_{ i \in I } U_{ i } \) be any basic open neighborhood of 
    \( \mathbf{x} \) -- again, we can always interchange "open" with "basic".
    We know by definition of the product topology that there can only be 
    finitely many \( i \in I \) such that \( U_{ i } \neq \mathbb{R} \);
    let \( i_{ 1 } , \ldots , i_{ n } \) be all such \( i \) and define 
    \( N = \max \{ i_{ 1 } , \ldots , i_{ n } \} \).
    From this, it follows that for all \( i > N \), we have 
    \( U_{ i } = \mathbb{R} \).

    \baseSkip 

    We now define a sequence \( \mathbf{x}' \) of elements
    in \( \mathbb{R}^{ \infty } \) as follows:
    \begin{equation*}
        x_{ i }'
        \equiv 
        \begin{cases} 
            x_{ i } &\quad i \leq N
            \\
            0 &\quad i > N
        \end{cases}
    \end{equation*}
    We can see that \( \mathbf{x}' \in U \cap \mathbb{R}^{ \infty } \),
    which means that \( U \cap \mathbb{R}^{ \infty } \) is non-empty.
    Since our choice of \( U \) was arbitrary, we have that every basic open 
    set in \( \mathbb{R}^{ \omega } \) containing \( \mathbf{x} \) intersects
    \( \mathbb{R}^{ \infty } \).
    Thus, \( \mathbf{x} \in \mathrm{Cl} \ \mathbb{R}^{ \infty } \).

    \baseSkip

    Furthermore, since our choice of \( \mathbf{x} \) was arbitrary, we have 
    that every element of \( \mathbb{R}^{ \omega } \) is in the closure of 
    \( \mathbb{R}^{ \infty } \).
    Therefore, we see that \( \mathbb{R}^{ \omega } = 
    \mathrm{Cl} \ \mathbb{R}^{ \infty } \), as desired.

    \baseRule

    Alternatively, for each element \( \mathbf{x} = ( x_{ 1 }, x_{ 2 }, 
    x_{ 3 }, \ldots ) \) of \( \mathbb{R}^{ \omega } \), we can identify a 
    sequence 
    \( \mathbf{x}^{ n } \) of elements in \( \mathbb{R}^{ \infty } \) so that 
    \( \mathbf{x}^{ n } \rightarrow \mathbf{x} \) in the product topology of 
    \( \mathbb{R}^{ \omega } \):
    \begin{equation*}
        \begin{aligned}
            \mathbf{x}^{ 1 } &= ( x_{ 1 }, 0, \ldots )
            \\
            \mathbf{x}^{ 2 } &= ( x_{ 1 }, x_{ 2 }, 0, \ldots )
            \\
            \mathbf{x}^{ 3 } &= ( x_{ 1 }, x_{ 2 }, x_{ 3 }, 0, \ldots )
            \\
            &\vdots 
            \\
            \mathbf{x} &= ( x_{ 1 }, x_{ 2 }, x_{ 3 }, \ldots )
        \end{aligned}
    \end{equation*}
    We can see that this sequence converges in the product topology of 
    \( \mathbb{R}^{ \omega } \) since we have that each \( x_{ i }^{ n } 
    \rightarrow x_{ i } \) (recall 
    [\hyperlink{thm:15_convergence_product}{convergence}] of sequences in 
    products).

    \baseSkip

    To show why \( \mathbb{R}^{ \omega } \) is equal to 
    \( \mathrm{Cl} \ ( \mathbb{R}^{ \infty } ) \), we note that 
    every open neighborhood of \( \mathbf{x} \) contains points in this 
    sequence (by construction, essentially), which are also points in 
    \( \mathbb{R}^{ \infty } \).
    So \( \mathbf{x} \) is in \( \mathrm{Cl} \ ( \mathbb{R}^{ \infty } ) \),
    and since this choice of \( \mathbf{x} \) was arbitrary, we have that 
    \( \mathbb{R}^{ \omega } \) is equal to 
    \( \mathrm{Cl} \ ( \mathbb{R}^{ \infty } ) \).
\end{egBox}