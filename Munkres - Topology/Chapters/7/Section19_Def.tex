\begin{defBox}{\( I \)-tuple and Coorindate}[def:19_Ituple_coord]
    Let \( I \) be an index set.
    Given a set \( X \), we define a \( I \)\textbf{-tuple} of elements of 
    \( X \) to be a function \( \mathbf{x}: I \rightarrow X \).
    If \( i \) is an element of \( I \), we denote the value of \( \mathbf{x} \)
    at \( i \) by \( x_{ i } \) rather than \( \mathbf{x} ( i ) \);
    we call it the \( i \)th \textbf{coordinate} of \( \mathbf{x} \).
    We often denote the function \( \mathbf{x} \) itself by the symbol
    \begin{equation*}
        ( x_{ i } )_{ i \in I }
    \end{equation*}
    which is as close as we can come to a "tuple notation" for an arbitrary
    index set \( I \).
    We denote the set of all \( I \)-tuples of elements of \( X \) by 
    \( X^{ I } \).
\end{defBox}

\begin{defBox}{Cartesian Product}[def:19_cartesian_product]
    Let \( \{ X_{ i } \}_{ i \in I } \) be an indexed family of sets;
    let \( X = \bigcup_{ i \in I } X_{ i } \).
    The \textbf{cartesian product} of this indexed family, denoted by 
    \begin{equation*}
        \prod_{ i \in I } X_{ i }
    \end{equation*}
    is defined to be the set of all \( I \)-tuples \( ( x_{ i } )_{ i \in I } \)
    of elements of \( X \) such that \( x_{ i } \in X_{ i } \) for each 
    \( i \in I \). That is, it is the set of all functions
    \begin{equation*}
        \mathbf{x}: I \rightarrow \bigcup_{ i \in I } X_{ i }
    \end{equation*}
    such that \( \mathbf{x} ( i ) \in X_{ i } \) for each \( i \in I \).
\end{defBox}

\begin{defBox}{Arbitrary Products}[def:19_arb_products]
    Let \( I \) be an index set.
    If \( \{ X_{ i } \}_{ i \in I } \) is an indexed collection of sets,
    then we can define their product (also known as the \textbf{cartesian 
    product}) as follows :
    \begin{equation*}
        \prod_{ i \in I } X_{ i }
        =
        X_{ 1 } \times X_{ 2 } \times X_{ 3 } \times \ldots
        =
        \left\{
            ( x_{ 1 } , x_{ 2 }, x_{ 3 }, \ldots )
            \ \middle\vert \ 
            x_{ 1 } \in X_{ 1 }, x_{ 2 } \in X_{ 2 }, x_{ 3 } \in X_{ 3 },
            \ldots
        \right\}
    \end{equation*}
    The elements of this product, say \( \mathbf{x} \), are entirely 
    determined by their coordinates \( x_{ i } \) where \( i \in I \).
    We notate this by writing these elements in terms of their coordinates:
    \begin{equation*}
        \mathbf{x}
        =
        ( x_{ i } )_{ i \in I }
        =
        ( x_{ 1 } , x_{ 2 }, x_{ 3 }, \ldots )
    \end{equation*}
\end{defBox}

\begin{defBox}{Box Topology}[def:19_box_top]
    Let \( \{ X_{ i } \}_{ i \in I } \) be an indexed set of topological spaces
    and \( X = \prod_{ i \in I } X_{ i } \).
    The \textbf{box topology} on \( X \) is the topology generated by the basis
    of sets of the form: 
    \begin{equation*}
        \prod_{ i \in I } U_{ i }
        \quad \text{where }
        U_{ i }
        \text{ is open in }
        X_{ i } 
        \text{ for each }
        i \in I
    \end{equation*}
\end{defBox}

\begin{defBox}{Product Topology (Arbitrary Products)}[def:19_product_top]
    Let \( \{ X_{ i } \}_{ i \in I } \) be an indexed set of topological spaces
    and \( X = \prod_{ i \in I } X_{ i } \).
    The \textbf{product topology} on \( X \) is the topology generated by the 
    basis of the following form:
    \begin{equation*}
        \prod_{ i \in I } U_{ i }
        \quad \text{where }
        U_{ i }
        \text{ is open in }
        X_{ i } 
        \text{ for each }
        i \in I
    \end{equation*}
    \textbf{and}
    \begin{equation*}
        U_{ i } = X_{ i }
        \text{ for all but finitely many } i
    \end{equation*}
\end{defBox}

\begin{defBox}{Alternative Definition of the Product Topology}[def:19_alt_def_product_topology]
    Let \( \mathcal{S}_{ i } \) denote the collection
    \begin{equation*}
        \mathcal{S}_{ i } 
        =
        \{ \mathrm{proj}_{ i }^{ -1 } ( U_{ i } ) \mid U_{ i } 
        \text{ is open in } X_{ i } \}
    \end{equation*}
    and let \( \mathcal{S} \) denote the union of these collections,
    \begin{equation*}
        \mathcal{S}
        =
        \bigcup_{ i \in I } \mathcal{S}_{ i }
    \end{equation*}
    The topology generated by the subbasis \( \mathcal{S} \) is called the 
    \textbf{product topology}.
    In this topology, \( \prod_{ i \in I } X_{ i } \) is called a 
    \textbf{product space}.
\end{defBox}

\begin{defBox}{Projections}[def:19_projections]
    Let \( \{ X_{ i } \}_{ i \in I } \) be an indexed set of topological spaces
    and \( X = \prod_{ i \in I } X_{ i } \).
    For each fixed index \( k \in I \), the \textbf{projection onto the 
    \textit{k}th component} is the function \( \mathrm{proj}_{ k } : X 
    \rightarrow X_{ k } \), with the formula:
    \begin{equation*}
        \mathrm{proj}_{ k } ( \mathbf{x} )
        =
        \mathrm{proj}_{ k } ( x_{ 1 }, x_{ 2 }, x_{ 3 }, \ldots )
        =
        x_{ k }
    \end{equation*}
\end{defBox}