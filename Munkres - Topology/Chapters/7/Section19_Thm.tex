\begin{thmBox}{Comparison of the Box and Product Topologies}[thm:19.1]
    Let \( \{ X_{ i } \}_{ i \in I } \) be an indexed set of topological spaces
    and \( X = \prod_{ i \in I } X_{ i } \).
    The box topology on \( X \) has as basis all sets of the form 
    \( \prod_{ i \in I } U_{ i } \), where \( U_{ i } \) is open in 
    \( X_{ i } \) for each \( i \in I \).
    The product topology on \( X \) has as basis all sets of the form 
    \( \prod_{ i \in I } U_{ i } \), where \( U_{ i } \) is open in 
    \( X_{ i } \) for each \( i \in I \) \textit{and} \( U_{ i } = X_{ i } \)
    except for finitely many values of \( i \in I \).

    \baseRule

    \begin{proofBox}
        To compare the box and product topologies (using the alternative 
        definition), we consider the basis \( \mathcal{B} \) that 
        \( \mathcal{S} \) generates.
        Using what we found in [\hyperlink{thm:15.2}{Theorem 15.2}], we can 
        gather that the collection \( \mathcal{B} \) consist of all finite 
        intersections of elements of \( \mathcal{S} \).
        If we intersect elements belonging to the same one of the sets 
        \( \mathcal{S}_{ i } \), we do not get anything new, because 
        \begin{equation*}
            \mathrm{proj}_{ i }^{ -1 } ( U_{ i } )
            \cap 
            \mathrm{proj}_{ i }^{ -1 } ( V_{ i } )
            =
            \mathrm{proj}_{ i }^{ -1 } ( U_{ i } \cap V_{ i } )
        \end{equation*}
        that is, the intersection of two elements of \( \mathcal{S}_{ i } \), 
        or of finitely many such elements, is again an element of 
        \( \mathcal{S}_{ i } \).

        \baseSkip 

        We get something new only when we intersect elements from 
        \textit{different} sets \( \mathcal{S}_{ i } \).
        The typical element of the basis \( \mathcal{B} \) can thus be 
        described as follows: Let \( i_{ 1 } , \ldots , i_{ n } \) be a finite
        set of distinct indices from the index set \( I \), and let 
        \( U_{ i_{ j } } \) be an open set in \( X_{ i_{ j } } \) for 
        \( j = 1 , \ldots , n \).
        Then 
        \begin{equation*}
            B
            =
            \mathrm{proj}_{ i_{ 1 } }^{ -1 } ( U_{ i_{ 1 } } )
            \cap 
            \mathrm{proj}_{ i_{ 2 } }^{ -1 } ( U_{ i_{ 2 } } )
            \cap \ldots \cap 
            \mathrm{proj}_{ i_{ n } }^{ -1 } ( U_{ i_{ n } } )
        \end{equation*}
        is the typical element of \( \mathcal{B} \).

        \baseSkip

        Now a point \( \mathbf{x} = ( x_{ i } )_{ i \in I } \) is in \( B \)
        if and only if its \( i_{ 1 } \)th coordinate is in \( U_{ i_{ 1 } } \),
        its \( i_{ 2 } \)th coordinate is in \( U_{ i_{ 2 } } \), and so on.
        There is no restriction whatever on the \( ith \) coordinate of 
        \( \mathbf{x} \) it \( i \) is not one of the indices \( i_{ 1 }
        , \ldots , i_{ n } \). 
        As a result, we can write \( B \) as the product where \( U_{ i } \)
        denotes the entire space \( X_{ i } \) if \( i \neq i_{ 1 } , \ldots , 
        i_{ n } \).
    \end{proofBox}
\end{thmBox}

\begin{thmBox}{19.2}[thm:19.2]
    Suppose the topology on each space \( X_{ i } \) is given by a basis 
    \( \mathcal{B}_{ i } \).
    The collection of all sets of the form 
    \begin{equation*}
        \prod_{ i \in I } B_{ i }
    \end{equation*}
    where \( B_{ i } \in \mathcal{B}_{ i } \) for each \( i \in I \), will 
    serve as a basis for the box topology on \( \prod_{ i \in I } X_{ i } \).

    \baseSkip 

    The collection of all sets of the same form, where \( B_{ i } \in 
    \mathcal{B}_{ i } \) for finitely many indices \( i \) and \( B_{ i } = 
    X_{ i } \) for the remaining indices, will serve as a basis for the 
    product topology on \( \prod_{ i \in I } X_{ i } \).

    \baseRule

    \begin{proofBox}

    \end{proofBox}
\end{thmBox}

\begin{thmBox}{19.3}[thm:19.3]
    Let \( A_{ i } \) be a subspace of \( X_{ i } \), for each \( i \in I \).
    Then \( \prod_{ i \in I } A_{ i } \) is a subspace of \( \prod_{ i \in I }
    X_{ i } \) if both products are given in the box topology, or if both
    products are given in the product topology.

    \baseRule

    \begin{proofBox}

    \end{proofBox}
\end{thmBox}

\begin{thmBox}{19.4}[thm:19.4]
    If each space \( X_{ i } \) is Hausdorff, then \( \prod_{ i \in I } 
    X_{ i } \) is a Hausdorff space in both the box and product topologies.

    \baseRule

    \begin{proofBox}

    \end{proofBox}
\end{thmBox}

\begin{thmBox}{19.5}[thm:19.5]
    Let \( \{ X_{ i } \}_{ i \in I } \) be an indexed set of topological spaces
    and \( X = \prod_{ i \in I } X_{ i } \).
    Let \( A_{ i } \subset X_{ i } \) for each \( i \in I \).
    If \( X \) is given either the product or the box topology, then 
    \begin{equation*}
        \prod_{ i \in I } \overline{ A }_{ i }
        =
        \overline{ \prod_{ i \in I } A_{ i } }
    \end{equation*}

    \baseRule

    \begin{proofBox}

    \end{proofBox}
\end{thmBox}

\begin{thmBox}{19.6}[thm:19.6]
    Let \( f: A \rightarrow \prod_{ i \in I } X_{ i } \) be given by the 
    equation
    \begin{equation*}
        f ( a ) 
        =
        ( f_{ i } ( a ) )_{ i \in I }
    \end{equation*}
    where \( f_{ i }: A \rightarrow X_{ i } \) for each \( i \).
    Let \( X = \prod_{ i \in I } X_{ i } \) have the product topology.
    Then the function \( f \) is continuous if and only if each function
    \( f_{ i } \) is continuous.

    \baseRule

    \begin{proofBox}

    \end{proofBox}
\end{thmBox}

\begin{thmBox}{Inverse Image Under a Projection}[thm:19_inv_img]
    Let \( \{ X_{ i } \}_{ i \in I } \) be an indexed set of topological spaces
    and \( X = \prod_{ i \in I } X_{ i } \).
    The inverse image of an open subset of \( X_{ k } \) under 
    \( \mathrm{proj}_{ k } \) is open in the product topology (and therefore
    also in the at least as fine box topology).

    \baseRule

    \begin{proofBox}
        Let us start by fixing \( k \in I \) and having \( U_{ k } \subset 
        X_{ k } \) be an open subset.
        The inverse image of \( U_{ k } \) under \( \mathrm{proj}_{ k } \)
        is defined to be the set of all point in the domain \( X \) whose 
        output lies in \( U_{ k } \); i.e., it is the set of all points 
        whose \( k \)th coordinate is in \( U_{ k } \).
        We can write this as follows:
        \begin{equation*}
            \mathrm{proj}_{ k }^{ -1 } ( U_{ k } )
            =
            \prod_{ i \in I } U_{ i }
            \quad \mathrm{where} \quad 
            U_{ i } = X_{ i }
            \text{ when }
            i \neq k    
        \end{equation*}
        For \( i = k \), we already have that \( U_{ k } \subset X_{ k } \).

        \baseSkip

        Notice that \( \mathrm{proj}_{ k }^{ -1 } ( U_{ k } ) \) is an arbitrary
        product of open sets where \( U_{ i } = X_{ i } \) for all but 
        \( i = k \). 
        Thus, \( \mathrm{proj}_{ k }^{ -1 } ( U_{ k } ) \) is a basic open set
        in the product topology (for arbitrary products), which means that it 
        is open.
    \end{proofBox}
\end{thmBox}

\begin{thmBox}{Convergence of Sequences in Products}[thm:19_convergence_product]
    Let \( \{ X_{ i } \}_{ i \in I } \) be an indexed set of topological spaces
    and \( X = \prod_{ i \in I } X_{ i } \).
    Consider a sequence 
    \begin{equation*}
        \mathbf{x}^{ 0 } = ( x_{ i }^{ 0 } )_{ i \in I }
        \quad 
        \mathbf{x}^{ 1 } = ( x_{ i }^{ 1 } )_{ i \in I }
        \quad 
        \mathbf{x}^{ 2 } = ( x_{ i }^{ 2 } )_{ i \in I }
        \quad 
        \ldots
    \end{equation*}
    and a point \( \mathbf{x} = ( x_{ i } )_{ i \in I } \) to which we would
    like to determine whether the sequence converges.
    
    \baseSkip 

    Then \( \mathbf{x}^{ n } \rightarrow \mathbf{x} \) in the product topology
    if and only if \( x_{ i }^{ n } \rightarrow x_{ i } \) for each index 
    \( i \in I \).

    \baseSkip

    Note that this theorem holds for when \( I \) is a finite indexing set.

    \baseRule

    \begin{proofBox}*
        \wrapBox{\( \implies \)}
        Let \( \mathbf{x}^{ n } \rightarrow \mathbf{x} \) in the product 
        topology.
        Let \( i \in I \) be given, and also let an open neighborhood 
        \( U_{ i } \subset X_{ i } \) of \( x_{ i } \) be given.
        Our goal is to show that \( x_{ i }^{ n } \rightarrow x_{ i } \).

        \baseSkip 

        In order for our sequence to converge to \( x_{ i } \), it follows by 
        definition that 
        there exists some positive integer \( N \) such that \( x_{ i }^{ n } 
        \in U_{ i } \) for all \( n \geq N \).
        From here, we note that the [\hyperlink{thm:15_inv_img}{inverse image}] 
        of \( U_{ i } \) under \( \mathrm{proj}_{ i } \), i.e.
        \( \mathrm{proj}_{ i }^{ -1 } ( U_{ i } ) \), is open in \( X \).
        Further note that \( \mathrm{proj}_{ i }^{ -1 } ( U_{ i } ) \) 
        consists of all points in \( X \) such that their \( i \)th coordinate
        is in \( U_{ i } \). 
        Since \( U_{ i } \) was given to be an open neighborhood of 
        \( x_{ i } \), we see that \( x_{ i } \in U_{ i } \) by construction.
        Thus, we end up getting that \( \mathbf{x} \in 
        \mathrm{proj}_{ i }^{ -1 } ( U_{ i } ) \) -- that is, 
        \( \mathrm{proj}_{ i }^{ -1 } ( U_{ i } ) \) is an open neighborhood 
        of \( \mathbf{x} \).

        \baseSkip 

        Now because we were given \( \mathbf{x}^{ n } \rightarrow \mathbf{x} \),
        it follows that there exists some positive integer \( N \) such that 
        \( \mathbf{x}^{ n } \in \mathrm{proj}_{ i }^{ -1 } ( U_{ i } ) \) for 
        all \( n \geq N \).
        Notice that this implies that \( x_{ i }^{ n } \in U_{ i } \) for all 
        \( n \geq N \), which means that \( x_{ i }^{ n } \rightarrow 
        x_{ i } \).

        \baseSkip

        \wrapBox{\( \impliedby \)}
        Now let's assume that \( x_{ i }^{ n } \rightarrow x_{ i } \) for all 
        \( i \in I \).
        Our goal is to show that \( \mathbf{x}^{ n } \rightarrow \mathbf{x} \).

        \baseSkip

        Let us start by letting \( U = \prod_{ i \in I } U_{ i } \) be a 
        basic open neighborhood of \( \mathbf{x} \) -- always note that we 
        can interchange "open" and "basic".
        Because \( U_{ i } \) is an open neighborhood of \( x_{ i } \) and 
        because \( x_{ i }^{ n } \rightarrow x_{ i } \), we have that for all
        \( i \in I \), there exists \( N_{ i } \) so \( x_{ i }^{ n } \in 
        U_{ i } \) for all \( n \geq N_{ i } \).
        This is a little worrisome since we eventually want to take the maximum
        of all these \( N_{ i } \) so that we can use it for \( U \); if there 
        are infinitely many \( N_{ i } \), then we might not have a maximum!

        \baseSkip

        Thankfully, we have by definition of the product topology that 
        there only exists a finite amount of situations where \( U_{ i } \neq 
        X_{ i } \).
        In the case that \( U_{ i } = X_{ i } \), we can take \( N_{ i } = 0 \)
        since any sequence \( x_{ i }^{ n } \) is contained in \( X_{ i } \).
        Thus, we can pick \( N_{ i } = 0 \) 
        for all but finitely many \( i \) (those \( i \) where \( U_{ i } \neq
        X_{ i } \)).
        We can then set \( N = \max \{ \text{remaining } N_{ i } \} \), which
        results in us having that \( \mathbf{x}^{ n } \in U \) for all
        \( n \geq N \).
        Hence, we get that \( \mathbf{x}^{ n } \rightarrow \mathbf{x} \).
    \end{proofBox}
\end{thmBox}