\begin{thmBox}{25.1}[thm:25.1]
    The components of \( X \) are connected disjoint subspaces of \( X \) whose
    union is \( X \), such that each nonempty connected subspace of \( X \) 
    intersects only one of them.

    \baseRule

    \begin{proofBox}
        Since we know that the components of \( X \) are equivalence classes,
        it follows that these components are disjoint and their union is \( X \)
        by construction.
        Furthermore, we see that each connected subspace \( A \) of \( X \) 
        intersects with only one of these components: if \( A \) 
        intersects the components \( C_{ 1 } \) and \( C_{ 2 } \) of 
        \( X \), say in points \( x_{ 1 } \) and \( x_{ 2 } \), respectively, 
        then it follows that \( x_{ 1 } \sim x_{ 2 } \) by definition as they
        are both contained in the connected subspace \( A \); by the 
        transitivity property of our equivalence relation, we see that 
        all the points in \( C_{ 1 } \) are equivalent to \( C_{ 2 } \), which
        means that \( C_{ 1 } = C_{ 2 } \).

        \baseSkip 

        The proof for why components are connected is as follows: for each 
        point \( q \) of \( C ( p ) \), we know that \( p \sim q \).
        Thus, there is a connected subspace \( A_{ q } \) containing 
        both \( p \) and \( q \). 
        By the result that we just proved, it follows that 
        \( A_{ q } \subset C ( p ) \).
        Thus, it follows that 
        \begin{equation*}
            \bigcup_{ q \in C ( p ) } A_{ q }
            \subset 
            C ( p )
        \end{equation*}
        However, we know that for any point \( x \in C ( p ) \), it must be the 
        case that there exists at least one \( q \in C ( p ) \) such that 
        \( x \in A_{ q } \); thus, we have that
        \begin{equation*}
            C ( p )
            \subset
            \bigcup_{ q \in C ( p ) } A_{ q }
        \end{equation*}
        Hence, we see that 
        \begin{equation*}
            C ( p )
            =
            \bigcup_{ q \in C ( p ) } A_{ q }
        \end{equation*}
        and since the subspaces \( A_{ q } \) are connected and have the point
        \( p \) in common, we get that their union (i.e. \( C ( p ) \)) is 
        connected.
    \end{proofBox}
\end{thmBox}

\begin{thmBox}{25.2}[thm:25.2]
    The path components of \( X \) are path-connected disjoint subspaces of
    \( X \) whose union is \( X \), such that each nonempty path-connected
    subspace of \( X \) intersects only one of them.

    \baseRule

    \begin{proofBox}
        The proof for this theorem is almost the exact same as the proof given
        in [\hyperlink{thm:25.1}{Theorem 25.1}] -- just replace "component" with
        "path-component" and "connected" with "path-connected".
    \end{proofBox}
\end{thmBox}

\begin{thmBox}{25.3}[thm:25.3]
    A space \( X \) is locally connected if and only if for every open set 
    \( U \) of \( X \), each component of \( U \) is open in \( X \).

    \baseRule

    \begin{proofBox}
        Let \( X \) be locally connected and \( U \subset X \) be open.
        Let \( p \in U \) be given.
        Our goal is to show that \( C ( p ) \) is open in \( X \), where
        \( C ( p ) \) is the connected component of \( U \) containing \( p \).
        This amounts to showing that for all \( q \in C ( p ) \), we have 
        some open neighborhood of \( q \) that is contained in \( C ( p ) \).

        \baseSkip 

        Let \( q \in C ( p ) \) be given.
        Because \( X \) is locally connected, we see that there exists a 
        connected open neighborhood \( V \) of \( q \) such that 
        \( V \subset U \).
        By [\hyperlink{thm:25.1}{Theorem 25.1}], we see that \( V \) being 
        connected and containing \( q \) means that \( V \) is entirely
        contained in \( C ( p ) \); that is, \( V \) is in the union of all 
        connected subspaces of \( U \) that contain \( p \).
        Thus, \( V \subset C ( p ) \).
        Notice that another argument we can make is by looking at 
        \( C ( p ) \cup V \); this is a union of two connected spaces -- 
        both of which contains \( q \).
        Thus, by [\hyperlink{thm:23.3}{Theorem 23.3}], we see that 
        \( C ( p ) \cup V \) is also a connected subspace of \( U \) containing
        \( p \).
        By maximality, we see that \( V \subset C ( p ) \).
        Either way, we see that \( C ( p ) \) is open.

        \baseSkip

        Conversely, let's suppose that components of open sets in \( X \) are 
        open.
        Given a point \( x \) of \( X \) and a neighborhood \( U \) of \( x \),
        we let \( C \) to be the component of \( U \) containing \( x \).
        Since every component is connected, we have that \( C \) is a connected
        open neighborhood of \( x \) that is contained in \( U \).
        Hence, \( X \) is locally connected.
    \end{proofBox}
\end{thmBox}

\begin{thmBox}{25.4}[thm:25.4]
    A space \( X \) is locally path-connected if and only if for every open set
    \( U \) of \( X \), each path component of \( U \) is open in \( X \).

    \baseRule

    \begin{proofBox}
        Let \( X \) be locally path-connected and \( U \subset X \) be open.
        Let \( p \in U \) be given.
        For the sake of this proof, we shall denote \( C ( p ) \) to be the
        path-component of \( U \) containing \( p \).
        Our goal is to show that \( C ( p ) \) is open in \( X \).
        This amounts to showing that for all \( q \in C ( p ) \), we have 
        some open neighborhood of \( q \) that is contained in \( C ( p ) \).

        \baseSkip 

        Let \( q \in C ( p ) \) be given.
        Because \( X \) is locally path-connected, we see that there exists a 
        path-connected open neighborhood \( V \) of \( q \) such that 
        \( V \subset U \).
        By [\hyperlink{thm:25.2}{Theorem 25.2}], we see that \( V \) being 
        path-connected and containing \( q \) means that \( V \) is entirely
        contained in \( C ( p ) \); that is, \( V \) is in the union of all 
        connected subspaces of \( U \) that contain \( p \).
        Thus, \( V \subset C ( p ) \).

        \baseSkip

        Conversely, let's suppose that path-components of open sets in \( X \) 
        are open.
        Given a point \( x \) of \( X \) and a neighborhood \( U \) of \( x \),
        we let \( C \) to be the path-component of \( U \) containing \( x \).
        Since every path-component is path-connected, we have that \( C \) is a 
        connected open neighborhood of \( x \) that is contained in \( U \).
        Hence, \( X \) is locally connected.
    \end{proofBox}
\end{thmBox}

\begin{thmBox}{25.5}[thm:25.5]
    If \( X \) is a topological space, each path component of \( X \) lies in a
    component of \( X \).
    If \( X \) is locally path-connected, then the components and the 
    path-components of \( X \) are the same.

    \baseRule

    \begin{proofBox}
        Let \( C \) be a component of \( X \); let \( x \) be a point of 
        \( C \); let \( P \) be the path component of \( X \) containing
        \( x \).
        Since all path components are path-connected, and path connected sets
        are connected, it follows that \( P \) must be connected;
        thus, it follows that \( P \) is in the union of connected components
        of \( X \) that contain \( x \) -- that is, \( P \subset C \).
        Our goal is to show that if \( X \) is locally path-connected, then 
        \( P = C \).

        \baseSkip 

        Towards a contradiction, let's suppose that \( P \subsetneq C \)
        (i.e., \( P \) is a proper subset of \( C \)).
        This means that there exists path components of \( X \) that are
        different from \( P \) and intersect \( C \); let \( Q \) denote
        the union of all such path components.
        Using a similar argument that we did with \( P \), we see that 
        each of the path components that make up \( Q \) all are contained
        in \( C \).
        Thus, we see that 
        \begin{equation*}
            C = P \cup Q
        \end{equation*}
        Now, because \( X \) is locally path-connected, each path component of
        \( X \) is open in \( X \).
        Thus, we see that both \( P \) (which is a path-component) and \( Q \)
        (which is a union of path-components) are open in \( X \).
        As a result, we see that \( P \) and \( Q \) constitute a separation of
        \( C \), which contradicts the fact that \( C \) is connected.
    \end{proofBox}
\end{thmBox}

\begin{thmBox}{Connected Components}[thm:25.6]
    \begin{enumerate}[label = (\alph*)]
        \item Connected components are connected.
        \item Every connected component of \( X \) is closed in \( X \).
        \item Any pair of connected components must either be equal \textbf{or}
            disjoint.
    \end{enumerate}

    \baseRule

    \begin{proofBox}*
        \wrapBox{a}

        This is easily proved using [\hyperlink{thm:23.3}{Theorem 23.3}];
        since a connected component \( C ( p ) \) of \( X \) containing \( p \) 
        is the union of all connected subspaces of \( X \) that contain \( p \),
        we have that all of the connected subspaces in this union to contain the
        common point \( p \). 
        Thus, by [\hyperlink{thm:23.3}{Theorem 23.3}], we have that their union
        must also be connected.

        \baseSkip 
        \wrapBox{b}

        In general, we know that any set is contained in its closure; that is,
        \begin{equation*}
            C ( p ) \subset \overline{ C ( p ) }
        \end{equation*}
        Now, because \( C ( p ) \) is connected, we have that its closure 
        \( \overline{ C ( p ) } \) must be connected as well by 
        [\hyperlink{thm:23.4}{Theorem 23.4}].

        \baseSkip 

        Notice that \( \overline{ C ( p ) } \) is connected and contains the 
        point \( p \); thus, it follows by definition that
        \begin{equation*}
            \overline{ C ( p ) } \subset C ( p )
        \end{equation*}
        Hence, we get that \( C ( p ) = \overline{ C ( p ) } \), which results
        in \( C ( p ) \) to be closed.

        \baseSkip
        \wrapBox{c}

        It suffices to prove that if two connected components share a point in
        common, then the two connected components must be equal (else, they
        have no other choice but to be disjoint from each other).

        \baseSkip

        Let's say that we have two connected components \( C ( p ) \) and 
        \( C ( q ) \) that share a common point, say \( r \).
        It follows by [\hyperlink{thm:23.3}{Theorem 23.3}] that their union
        must be connected as well.
        Since \( C ( q ) \cup C ( p ) \) contains both \( p \) and \( q \), 
        it follows by the definition of a connected component that 
        \begin{equation*}
            C ( p ) \cup C ( q )
            \subset
            C ( p )
            \quad \mathrm{and} \quad 
            C ( p ) \cup C ( q )
            \subset
            C ( q )
        \end{equation*}
        This results in the following
        \begin{equation*}
            C ( p ) \cup C ( q )
            \subset 
            C ( p ) \cap C ( q )
            \implies 
            C ( p ) = C ( q )
        \end{equation*}
        Hence, we find that two connected components that share a common point
        must be equal.
    \end{proofBox}
\end{thmBox}

\begin{thmBox}{}[thm:25_homeomorphism_invariant]
    The number of connected components is a homeomorphism invariant.

    \baseRule

    \begin{proofBox}
        Let \( X \) be a topological space.
        We have shown in [\hyperlink{thm:25.1}{Theorem 25.1}] that the set of 
        all components of \( X \) is equivalent to the set of all equivalence 
        classes under the following equivalence relation: \( x \sim y \) if
        there is a connected subspace of \( X \) containing both \( x \) and
        \( y \).
        Thus, we have that the number of connected components is equivalent to
        the number of equivalence classes of \( X \).
        
        \baseSkip

        Now, let \( X \) be homeomorphic to \( Y \).
        This means that there exists a homeomorphism \( f: X \rightarrow Y \).
        Furthermore, we see that there exists a reduced homeomorphism 
        \( g: X^{ * } \rightarrow Y^{ * } \), where \( X^{ * } \) and 
        \( Y^{ * } \) correspond to the set of equivalence classes of \( X \) 
        and \( Y \) under the equivalence relation mentioned earlier;
        we indeed get a homeomorphism since restricting the domain/codomain
        of a continuous function still remains continuous.

        \baseSkip

        Now since \( g \) is a homeomorphism, it follows that it is bijective.
        Thus, we get that \( X^{ * } \) and \( Y^{ * } \) must have the 
        same cardinality: injective implies that \( \lvert X^{ * } \rvert \leq 
        \lvert Y^{ * } \rvert \), and surjective implies that 
        \( \lvert X^{ * } \rvert \geq \lvert Y^{ * } \rvert \).
        I.e., we see that the number of equivalence classes (equivalently, 
        the number of connected components) for both spaces must be the same.
    \end{proofBox}
\end{thmBox}