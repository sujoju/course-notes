\begin{defBox}{Components}[def:25_components]
    Given \( X \), define an equivalence relation on \( X \) by setting 
    \( x \sim y \) if there is a connected subspace of \( X \) containing both
    \( x \) and \( y \).
    The equivalence classes are called the \textbf{components} (or the 
    "connected components") of \( X \).

    \baseSkip

    Equivalent definition: let \( X \) be a topological space and \( p \) be a 
    point in \( X \). The \textbf{connected component} \( C ( p ) \) of \( X \)
    containing \( p \) is the union of all connected subspaces of \( X \) that
    contain \( p \).
\end{defBox}

\begin{defBox}{Path Components}[def:25_path_components]
    We define another equivalence relation on the space \( X \) by defining
    \( x \sim y \) if there is a path in \( X \) from \( x \) to \( y \).
    The equivalence classes are called \textbf{path components} of \( X \).

    \baseSkip

    Equivalent definition: let \( X \) be a topological space and \( p \) be a 
    point in \( X \). The \textbf{path components} of \( X \)
    containing \( p \) is the union of all path-connected subspaces of \( X \) 
    that contain \( p \).
\end{defBox}

\begin{defBox}{Locally (Path) Connected}[def:25_locally_connected]
    A space \( X \) is said to be \textbf{locally connected at} \( x \in X \) 
    if for every open neighborhood \( U \) of \( x \), there is a connected 
    open neighborhood \( V \) of \( x \) contained in \( U \).
    If \( X \) is locally connected at each of its points, \( X \) is said to be
    \textbf{locally connected}.

    \baseSkip

    Similarly, a space \( X \) is said to be \textbf{locally path connected at} 
    \( x \in X \) if for every open neighborhood \( U \) of \( x \), there is a 
    path-connected open neighborhood \( V \) of \( x \) contained in \( U \).
    If \( X \) is locally path-connected at each of its points, \( X \) is said 
    to be \textbf{locally path connected}.
\end{defBox}