\begin{egBox}{Trivial Topology}[eg:23.1]
    Let \( X \) be a topological space under the trivial topology.
    We have that \( X \) is connected.
    In fact, we have that \( X \) is path-connected.
\end{egBox}

\begin{egBox}{Discrete Topology}[eg:23.2]
    Let \( X \) be a topological space under the discrete topology (with 
    \( X \) having at least two elements).
    We have that \( X \) is disconnected (i.e., not connected) since every 
    subset of \( X \) is open; thus, it follows that every set is closed 
    since its complement is equal to some open set in \( X \).
\end{egBox}

\begin{egBox}{Non-Connected Spaces}[eg:23.3]
    Let \( X = [ 0, 1 ] \cup [ 2, 3 ] \) be a subset of \( \mathbb{R} \) with
    the subspace topology.
    Under this topology, we can see that both \( [ 0, 1 ] \) and 
    \( [ 2, 3 ] \) are clopen in \( X \).
    This tells us that \( X \) is not connected.
\end{egBox}

\begin{egBox}{Closed intervals in \( \mathbb{R} \) are Connected}[eg:23.4]
    Every closed interval in \( \mathbb{R} \) is connected.
    We shall not prove this very rigorously, but will provide the main 
    reasoning behind the proof.

    \baseSkip
    Take \( X = [ 0, 3 ] \).
    We want to show that a non-empty clopen subset \( A \) must be 
    \( [ 0, 3 ] \).
    Let's say that \( A \) contains \( 1 \).
    Since \( A \) is clopen (hence, open), we see that there must be some open 
    interval, say \( I \), that contains \( 1 \) and is contained in \( A \).
    Now, since \( A \) is clopen (hence, closed), we know that \( I \) must 
    include its limits points (since it must equal its closure), which means
    that its endpoints are included in \( I \).

    \baseSkip

    From here, we can apply a similar argument to the endpoints of \( I \); if 
    we focus our attention to the right endpoint, then we see that 
    since \( A \) is clopen (hence, open), there must be some open 
    interval, say \( I' \), that contains the endpoint and is contained in 
    \( A \).
    Now, since \( A \) is clopen (hence, closed), we know that \( I' \) must 
    include its limits points (since it must equal its closure), which means
    that its endpoints are included in \( I' \) as well.

    \baseSkip

    Continuing this process for both endpoints results in us eventually getting 
    that \( A = [ 0, 3 ] \) -- thus, showing that every closed interval in
    \( \mathbb{R} \) is connected.
\end{egBox}

\begin{egBox}{Connectedness in the Cofinite Topology}[eg:23.5]
    Let \( X \) be a set equipped with the cofinite topology.
    We claim that \( X \) is connected if \( X \) is infinite or if \( X \) 
    has \( \leq 1 \) element.
    Furthermore, we claim that \( X \) is \textbf{disconnected} (i.e., not 
    connected) otherwise.

    \baseSkip

    Starting with the first claim, let us suppose that \( X \) is infinite.
    We want to show that \( X \) has no non-trivial clopen subsets.
    Towards a contradiction, let's assume that there exists a non-trivial 
    clopen subset \( A \) of \( X \).
    Since \( A \) is open in \( X \) under the cofinite topology, it follows
    that \( X \setminus A \) is finite (it cannot be all of \( X \) since 
    \( A \) is non-trivial).
    Also, since we know that \( A \) is also closed in \( X \) under the 
    cofinite topology, it follows that \( X \setminus A \) must be open in 
    \( X \).
    However, \( X \setminus A \) being open in \( X \) means that 
    \( X \setminus ( X \setminus A ) = A \) must be finite (it cannot be all of 
    \( X \) since \( A \) is non-trivial).
    This results in a contradiction since \( X \) is assumed to be infinite,
    meaning that both \( X \setminus A \) and \( A \) cannot be finite.
    Thus, \( X \) has no non-trivial clopen subsets, meaning that \( X \) is 
    connected.

    \baseSkip

    Let us now consider the case when \( X \) has \( \leq 1 \) element.
    If \( X \) has no elements (that is, \( X = \emptyset \)), then we are 
    done since \( \emptyset \) is trivially connected.
    Now if \( X \) has \( 1 \) element, then we see that the only clopen sets of
    \( X \) under the cofinite topology are the trivial clopen sets.
    Thus, we have that \( X \) is connected as well.

    \baseSkip

    We now want to show that \( X \) is disconnected otherwise.
    Suppose that \( X \) is a finite set that has \( > 1 \) elements.
    Because \( X \) is finite, we end up getting that the discrete topology and
    cofinite topology are equivalent to each other.
    Since we know that the discrete topology is disconnected, we have that
    \( X \) is disconnected as well.
\end{egBox}

\begin{egBox}{Connectedness of Boundary and Interior}[eg:23.6]
    Suppose that \( A \) is a connected subspace of \( X \).
    Does it follows that \( \mathrm{Int} \ A \) and \( \partial A \) are
    connected as well?
    Does the converse hold?

    \baseSkip

    Let \( X = \mathbb{R} \) under the standard topology;
    let \( A = [ -1, 0 ] \cup [ 0, 1 ] \).
    We can see that \( A \) is connected as it is the union of two connected 
    sets that share a point in common.
    However, we see that \( \mathrm{Int} \ A = ( -1, 0 ) \cup ( 0, 1 ) \),
    which is not connected. 
    Now, let \( A = ( 0, 1 ) \).
    Be can see that \( \partial A = \{ 0, 1 \} \), which is also not connected
    since we see that both \( \{ 0 \} \) and \( \{ 1 \} \) are nontrivial 
    clopen subsets of \( A \).
    Hence, we end up seeing that \( \mathrm{Int} \ A \) and \( \partial A \) 
    need not be connected if \( A \) is connected.

    \baseSkip

    As for the converse, let us assume that \( \mathrm{Int} \ A \) and
    \( \partial A \) are connected.
    Towards a contradiction, let us assume that \( A \) is connected.
    It follows that \( \mathrm{Cl} \ A \) must be connected as well.
    If we let \( A = \mathbb{Q} \), then we see that 
    \( \mathrm{Int} \ \mathbb{Q} = \emptyset \); indeed, there is no way
    that we can have an open neighborhood of any rational number such that
    this neighborhood is fully contained in \( \mathbb{Q} \) -- we can always 
    another real number within this neighborhood . 
    Thus, it follows that
    \begin{equation*}
        \partial \mathbb{Q} 
        =
        \mathrm{Cl} \ \mathbb{Q} \setminus \mathrm{Int} \ \mathbb{Q}
        =
        \mathbb{R} \setminus \emptyset
        =
        \mathbb{R}
    \end{equation*}
    Notice that both \( \emptyset \) and \( \mathbb{R} \) are connected, which
    tells us that \( \mathrm{Int} \ \mathbb{Q} \) and \( \partial \mathbb{Q} \)
    are both connected.
    However, we know that \( \mathbb{Q} \) is not connected since
    \( \mathbb{Q} \cap ( - \infty, a ) \) and
    \( \mathbb{Q} \cap ( a, \infty ) \) form a separation of \( \mathbb{Q} \).
    Notice that \( \mathrm{Cl} \ A = \mathrm{Int} \ A \cup \partial A \).
\end{egBox}