\begin{thmBox}{17.1}[thm:17.1]
    Let \( X \) be a topological space. 
    Then the following conditions hold: 
    \begin{enumerate}
        \item \( \emptyset \) and \( X \) are closed.
        \item Arbitrary intersections of closed sets are closed. 
        \item Finite union of closed sets are closed. 
    \end{enumerate}

    \baseRule

    \begin{proofBox}
        \baseSkip

        \wrapBox{1}
        \( \emptyset \) and \( X \) are closed because they are the complements 
        of the open sets \( X \) and \( \emptyset \), respectively.

        \baseSkip
        \wrapBox{2}
        Given a collection of closed sets \( \{ A_{ i } \}_{ i \in I } \), where
        \( I \) is some index set, we want to show that 
        \( \bigcap_{ i \in I } A_{ i } \) is closed.
        Equivalently, we want to show that its complement 
        \( X \setminus \bigcap_{ i \in I } A_{ i } \) is open.
        We see by DeMorgan's law that 
        \begin{equation*}
            X \setminus \bigcap_{ i \in I } A_{ i }
            =
            \bigcup_{ i \in I } ( X \setminus A_{ i } )
        \end{equation*}
        Since the sets \( X \setminus A_{ i } \) are open by definition, the 
        right side of this equation represents an arbitrary union of open sets,
        which we know to be open. 
        Therefore, we have that arbitrary intersections of closed sets are 
        closed. 

        \baseSkip
        \wrapBox{3}
        Similarly, if we have a finite collection of sets 
        \( A_{ 1 } , \ldots , A_{ n } \), then we want to show that 
        \( \bigcup_{ i = 1 }^{ n } A_{ i } \) is closed.
        Equivalently, we want to show that its complement 
        \( X \setminus \bigcup_{ i = 1 }^{ n } A_{ i } \) is open.
        We see by DeMorgan's law again that 
        \begin{equation*}
            X \setminus \bigcup_{ i = 1 }^{ n } A_{ i }
            =
            \bigcap_{ i = 1 }^{ n } ( X \setminus A_{ i } )
        \end{equation*}
        Since the sets \( X \setminus A_{ i } \) are open by definition, the 
        right side of this equation represents a finite intersection of open 
        sets, which we know to be open. 
        Therefore, we have that finite unions of closed sets are 
        closed. 
    \end{proofBox}
\end{thmBox}

\begin{thmBox}{Alternative Characterizations}[thm:17_alt_characterization]
    Let \( A \) be a subset of a topological space \( X \).
    \begin{enumerate}[label = (\alph*)]
        \item A point \( x \in X \) is in the interior of \( A \) if and only 
            if there is an open neighborhood of \( x \) that is entirely 
            contained in \( A \).
        \item A point \( x \in X \) is in the closure of \( A \) if and only 
            if every open neighborhood of \( x \) has a nonempty intersection 
            with \( A \).
        \item A point \( x \in X \) is in the boundary of \( A \) if and only 
            if every open neighborhood of \( x \) has a nonempty intersection 
            with \( A \) and with \( X \setminus A \).
    \end{enumerate}

    \baseRule

    \begin{proofBox}

    \end{proofBox}
\end{thmBox}

\begin{thmBox}{17.2}[thm:17.2]
    Let \( Y \) be a subspace of \( X \). Then a set \( A \) is closed in 
    \( Y \) if and only if it equals the intersection of a closed set of 
    \( X \) with \( Y \).

    \baseRule

    \begin{proofBox}*
        \wrapBox{\( \implies \)}
        Let's assume that \( A \) is closed in \( Y \). 
        By definition, it follows that \( Y \setminus A \) is open in \( Y \),
        which by definition again means that \( Y \setminus A \) must equal 
        the intersection of an open set \( U \) of \( X \) with \( Y \).
        Thus, we have that \( Y \setminus A = Y \cap U \). 
        From here, we see the following:
        \begin{equation*}
            Y \setminus ( Y \setminus A ) = Y \setminus ( Y \cap U )
            \iff 
            A = Y \cap ( X \setminus U )
        \end{equation*}
        Furthermore, it follows that \( X \setminus U \) is closed in \( X \) 
        as \( X \setminus ( X \setminus U ) = U \) is open in \( X \).
        Thus, we see that \( A \) equals the intersection of a closed set of 
        \( X \) with \( Y \).

        \baseSkip

        \wrapBox{\( \impliedby \)}
        Let's now assume that \( A = Y \cap C \), where \( C \) is closed in 
        \( X \).
        It follows by definition that \( X \setminus C \) is open in \( X \),
        meaning that \( Y \cap ( X \setminus C ) \) is open in \( Y \), which 
        follows by definition of the subspace topology.
        This is where we note that 
        \begin{equation*}
            Y \setminus A
            =
            Y \setminus ( Y \cap C )
            \iff 
            Y \setminus A
            =
            Y \cap ( X \setminus C )
        \end{equation*}
        Furthermore, if follows that \( X \setminus C \) is open in \( X \) 
        since \( X \setminus ( X \setminus C ) = C \) is closed.
        Thus, we have that \( Y \setminus A \) is open in \( Y \) by definition
        of the subspace topology, which implies that \( A \) is closed in 
        \( Y \) by definition as well.
    \end{proofBox}
\end{thmBox}

\begin{thmBox}{17.3}[thm:17.3]
    Let \( Y \) be a subspace of \( X \). If \( A \) is closed in \( Y \) and 
    \( Y \) is closed in \( X \), then \( A \) is closed in \( X \).

    \baseRule

    \begin{proofBox}
        If \( A \) is closed in \( Y \), then we see by [\hyperlink{thm:17.2}
        {Theorem 17.2}] that there exists some closed set \( C \) of \( X \) 
        such that \( A = Y \cap C \).
        Since we know that \( Y \) is closed in \( X \), we have that 
        \( Y \cap C \) is also closed in \( X \) due to arbitrary intersections
        of closed sets of \( X \) being closed in \( X \).
        Thus, we have that \( A \) is closed in \( X \).
    \end{proofBox}
\end{thmBox}

\begin{thmBox}{17.4}[thm:17.4]
    Let \( Y \) be a subspace of \( X \), let \( A \) be a subset of \( Y \), 
    let \( \overline{ A } \) denote the closure of \( A \) in \( X \).
    Then the closure of \( A \) in \( Y \) equals \( \overline{ A } \cap Y \).

    \baseRule

    \begin{proofBox}
        Let \( B \) denote the closure of \( A \) in \( Y \).
        We know that the set \( \overline{ A } \) is closed in \( X \), which 
        means that \( Y \cap \overline{ A } \) is closed in \( Y \) by 
        [\hyperlink{thm:17.2}{Theorem 17.2}].
        Because \( A \subset Y \) and \( A \subset \overline{ A } \), we have 
        that \( A \subset Y \cap \overline{ A } \).
        We now note that since \( B \) is the closure of \( A \) in \( Y \),
        it follows by definition that \( B \) equals the intersection of 
        \textit{all} closed subsets of \( Y \) containing \( A \).
        Thus, it must be the case that 
        \( B \subset ( Y \cap \overline{ A } ) \).

        \baseSkip

        On the other hand, we know that \( B \) is closed in \( Y \). 
        Thus, by [\hyperlink{thm:17.2}{Theorem 17.2}], we have that
        \( B = Y \cap C \) for some closed set \( C \) of \( X \). 
        Since \( B \) contains \( A \), it must be the case that \( C \) 
        contains \( A \) as well.
        As a result, we have that \( C \) is a closed set of \( X \) 
        containing \( A \), which means that it is in the collection of closed
        sets containing \( A \).
        By definition, \( \overline{ A } \) is the intersection of such a 
        collection, which means that \( \overline{ A } \subset C \).
        Therefore, we have that \( ( Y \cap \overline{ A } ) \subset 
        ( Y \cap C ) = B \).

        \baseSkip 

        Putting everything together gives us that 
        \( B = Y \cap \overline{ A } \).
    \end{proofBox}
\end{thmBox}

\begin{thmBox}{17.5}[thm:17.5]
    Let \( A \) be a subset of the topological space \( X \).
    \begin{enumerate}[label = (\alph*)]
        \item Then \( x \in \overline{ A } \) if and only if every open set 
            \( U \) containing \( x \) intersects \( A \).
        \item Supposing the topology of \( X \) is given be a basis, then 
            \( x \in \overline{ A } \) if and only if every basis element 
            \( B \) containing \( x \) intersects \( A \).
    \end{enumerate}

    \baseRule

    \begin{proofBox}*
        \wrapBox{a}
        We shall prove this statement via contrapositive -- that is, we are 
        proving the following: \( x \notin \overline{ A } \) if and only if 
        there exists an open set \( U \) containing \( x \) that does not 
        intersect with \( A \).

        \baseSkip
        
        (\( \implies \)) If \( x \notin \overline{ A } \), then we have that 
        \( U = X \setminus \overline{ A } \) is an open set containing \( x \) 
        that does not intersect \( A \), as desired.
        
        \baseSkip

        (\( \impliedby \)) If there exists an open set \( U \) containing 
        \( x \) which does not intersect \( A \), then we see that 
        \( X \setminus U \) is a closed set that contains \( A \).
        By definition of the closure \( \overline{ A } \), we have that 
        \( \overline{ A } \subset ( X \setminus U ) \).
        Since \( x \notin X \setminus U \), it follows that \( x \notin 
        \overline{ A } \).

        \baseSkip

        \wrapBox{b}
        This statement follows readily from (a).

        \baseSkip

        (\( \implies \)) If every open set containing \( x \) intersects \( A \)
        , then so does
        every basis element \( B \) containing \( x \), because \( B \) is 
        an open set.

        \baseSkip

        (\( \impliedby \)) If every basis element containing \( x \) intersects
        \( A \), then so does every open set \( U \) containing \( x \), because
        \( U \) contains a basis element that contains \( x \).
    \end{proofBox}
\end{thmBox}

\begin{thmBox}{17.6}[thm:17.6]
    Let \( A \) be a subset of the topological space \( X \), and let 
    \( A' \) be the set of all limit points of \( A \).
    Then 
    \begin{equation*}
        \overline{ A }
        =
        A \cup A'
    \end{equation*}

    \baseRule

    \begin{proofBox}*
        \wrapBox{\( A \cup A' \subset \overline{ A } \)}
        Let's suppose that \( x \in A \cup A' \).
        If \( x \in A \), then we see that \( x \in \overline{ A } \) since 
        \( A \subset \overline{ A } \) by definition.
        Now if \( x \in A' \), then we see that every neighborhood of \( x \)
        intersects \( A \) (in a point different from \( x \)).
        Therefore, by [\hyperlink{thm:17.5}{Theorem 17.5}], we have that 
        \( x \in \overline{ A } \).
        Hence, \( A' \subset A \).
        Putting everything together results in 
        \( A \cup A' \subset \overline{ A } \).

        \baseSkip

        \wrapBox{\( \overline{ A } \subset ( A \cup A' ) \)}
        We now suppose \( x \in \overline{ A } \).
        Our goal is to show that \( x \in A \cup A' \).
        If \( x \) happens to lie in \( A \), it is trivial that
        \( x \in A \cup A' \).
        Let's now suppose that \( x \in \overline{ A } \) and \( x \notin A \).
        Since \( x \in \overline{ A } \), we know by by [\hyperlink{thm:17.5}
        {Theorem 17.5}] that every neighborhood \( U \) of \( x \) intersects 
        \( A \);
        because \( x \notin A \), we have the set \( U \) must intersect \( A \)
        at a point different from \( x \).
        Thus, \( x \in A' \) by definition, and we have 
        \( x \in A \cup A' \).

        \baseSkip

        Putting everything together results in \( \overline{ A } = A \cup A' \).
    \end{proofBox}
\end{thmBox}

\begin{thmBox}[Corollary]{17.7}[cor:17.7]
    A subset of a topological space is closed if and only if it contains all its limit points. 

    \baseRule

    \begin{proofBox}*
        (\( \implies \))
        The set \( A \) is closed if and only if \( A = \overline{ A } \), 
        meaning that \( A \) contains all its limits points by 
        [\hyperlink{thm:17.6}{Theorem 17.6}].

        \baseSkip 

        (\( \impliedby \))
        If \( A \) contains all its limit points, then we see that 
        \( A' \subset A \). 
        Thus, we have that \( A \cup A' = A \), which equivalently means that 
        \( \overline{ A } = A \) by [\hyperlink{thm:17.6}{Theorem 17.6}].
    \end{proofBox}
\end{thmBox}

\begin{thmBox}{17.8}[thm:17.8]
    Every finite point set in a Hausdorff space \( X \) is closed.

    \baseRule

    \begin{proofBox}
        Notice that every finite point set can be viewed as the union of 
        one-point sets -- e.g., \( \{ a, b, c \} = \{ a \} \cup \{ b \} \cup 
        \{ c \} \). 
        Thus, it suffices to show that every one-point set \( \{ x_{ 0 } \} \)
        is closed since every finite point set will then be a finite union 
        of closed sets, which is still closed.

        \baseSkip

        If \( x \) is a point of \( X \) different from \( x_{ 0 } \), then 
        \( x \) and \( x_{ 0 } \) have disjoint neighborhoods \( U \) and 
        \( V \), respectively.
        Since \( U \) does not intersect \( \{ x_{ 0 } \} \), the point \( x \)
        cannot belong to the closure of the set \( \{ x_{ 0 } \} \).
        As a result, the closure of the set \( \{ x_{ 0 } \} \) is 
        \( \{ x_{ 0 } \} \) itself, so that it is closed.
    \end{proofBox}
\end{thmBox}

\begin{thmBox}{17.9}[thm:17.9]
    Let \( X \) be a space satisfying the \( T_{ 1 } \) axiom; let \( A \) be a 
    subset of \( X \).
    Then the point \( x \) is limit point of \( A \) if and only if every
    neighborhood of \( x \) contains infinitely many points of \( A \).

    \baseRule

    \begin{proofBox}*
        \wrapBox{\( \implies \)} 
        For this direction, we shall prove by contradiction.
        Suppose that \( x \) is a limit point of \( A \).
        Suppose as well that some neighborhood \( U \) of \( x \) intersects
        \( A \) in only finitely many points.
        Then we have that \( U \) also intersects \( A \setminus \{ x \} \)
        in finitely many points, which we shall denote as 
        \( x_{ 1 } , \ldots , x_{ m } \).
        It follows by the \( T_{ 1 } \) axiom that \( \{ x_{ 1 } , \ldots , 
        x_{ m } \} \) is closed as it is a finite point set.
        Thus, the set \( X \setminus \{ x_{ 1 } , \ldots , x_{ m } \} \) is an
        open set of \( X \).
        From this, we have that
        \begin{equation*}
            U \cap ( X \setminus \{ x_{ 1 } , \ldots , x_{ m } \} )
        \end{equation*}
        is a neighborhood of \( x \) that does not intersect with the set 
        \( A \setminus \{ x \} \); we have essentially constructed a set that 
        removes all the intersection points of \( A \setminus \{ x \} \) and 
        \( U \), while maintaining such a set to be open.
        This contradicts the assumption that \( x \) is a limit point of 
        \( A \) since we have found a neighborhood of \( x \) that does not 
        intersect \( A \) in some other point other than \( x \) itself.

        \baseSkip 

        \wrapBox{\( \impliedby \)} 
        Now, if every neighborhood of \( x \) intersects \( A \) in infinitely 
        many points, it certainly must intersect \( A \) in some point other 
        than \( x \) itself.
        Hence, \( x \) must be a limit point of \( A \).
    \end{proofBox}
\end{thmBox}

\begin{thmBox}{17.10}[thm:17.10]
    If \( X \) is a Hausdorff space, then a sequence of points of \( X \) 
    converges to at most one point of \( X \).

    \baseRule

    \begin{proofBox}
        Let's suppose that \( ( x_{ n } ) \) is a sequence of points of \( X \)
        that converges to \( x \). 
        If we now take any other point \( y \neq x \), then \( X \) being 
        Hausdorff allows for us to find two disjoint neighborhoods \( U \) and
        \( V \) of \( x \) and \( y \), respectively. 
        Since we know that \( U \) contains \( x_{ n } \) for all but finitely
        many values of \( n \) -- that is, there is a positive integer \( N \) 
        such that \( x_{ n } \in U \) for all \( n \geq N \) -- we have that 
        it is impossible for the set \( V \) to also contain \( x_{ n } \) for all but finitely many values of \( n \) since it is disjoint to 
        \( U \).
        Thus, \( ( x_{ n } ) \) cannot converge to \( y \).
        Since \( y \) was arbitrary, we have that \( ( x_{ n } ) \) converges 
        to at most one point of \( X \).
    \end{proofBox}
\end{thmBox}

\begin{thmBox}{17.11}[thm:17.11]
    Every simply ordered set is a Hausdorff space in the order topology.
    The product of two Hausdorff spaces is a Hausdorff space. 
    A subspace of a Hausdorff space is a Hausdorff space.

    \baseRule

    \begin{proofBox}*
        \wrapBox{Subspace of a Hausdorff Space is Hausdorff}
        Let \( X \) be a Hausdorff space and \( Y \subset X \) be a subspace of 
        \( X \).
        Let \( x, y \in Y \) be two distinct points.
        Our goal is to show that there exists open neighborhoods of \( x \) and 
        \( y \), respectively, in \( Y \) such that these neighborhoods are 
        disjoint.

        \baseSkip

        Since \( x, y \) are also distinct elements in \( X \), we see that \( X \) 
        being Hausdorff tells us that there exists open neighborhoods 
        \( U \) and \( V \) of \( x \) and \( y \), respectively,
        in \( X \) such that \( U \cap V = \emptyset \).
        With this, \( x \in U \) and \( x \in Y \) tells us 
        that \( x \in Y \cap U \); similarly, we have that \( y \in Y 
        \cap V \).

        \baseSkip

        Notice that both \( Y \cap U \) and \( Y \cap V \) are 
        open in \( Y \) since \( U \) and \( V \) are open in \( X \). 
        Furthermore, we have that 
        \begin{equation*}
            ( Y \cap U ) \cap ( Y \cap V )
            =
            \underbrace{ ( Y \cap Y ) }_{ Y } \cap 
            \underbrace{ ( U \cap V ) }_{ \emptyset }
            =
            \emptyset
        \end{equation*}
        Putting everything together, we see that \( Y \cap U \) and \( Y \cap V \)
        are open neighborhoods of \( x \) and \( y \), respectively, in \( Y \) 
        that are also disjoint.
        Hence \( Y \) is Hausdorff.

        \baseSkip 

        \wrapBox{Product of Two Hausdorff Spaces is Hausdorff}

        Let \( X \) and \( Y \) be two Hausdorff spaces.
        Our goal is to show that \( X \times Y \) is Hausdorff.
        We start by letting \( \mathbf{x}_{ 1 }, \mathbf{x}_{ 2 } \in X \times 
        Y \)
        be any two distinct points -- that is, 
        \( \mathbf{x}_{ 1 } = ( x_{ 1 }, y_{ 1 } ) \) and \( \mathbf{x}_{ 2 }
        = ( x_{ 2 }, y_{ 2 } ) \) with \( x_{ 1 } \neq x_{ 2 } \) and 
        \( y_{ 1 } \neq y_{ 2 } \).

        \baseSkip 

        Because \( X \) is Hausdorff, we see that there exists 
        open neighborhoods \( U_{ 1 }, U_{ 2 } \) of \( x_{ 1 }, x_{ 2 } \) in 
        \( X \) such that \( U_{ 1 } \cap U_{ 2 } = \emptyset \); similarly, 
        because \( Y \) is Hausdorff, we see that there exists 
        open neighborhoods \( V_{ 1 }, V_{ 2 } \) of \( y_{ 1 }, y_{ 2 } \) in 
        \( Y \) such that \( V_{ 1 } \cap V_{ 2 } = \emptyset \).
        From this, we get that \( U_{ 1 } \times V_{ 1 } \) and \( U_{ 2 } 
        \times V_{ 2 } \) are two open neighborhoods of \( \mathbf{x}_{ 1 } \) 
        and \( \mathbf{x}_{ 2 } \) that are disjoint from each other:
        \begin{equation*}
            ( U_{ 1 } \times V_{ 1 } ) \cap ( U_{ 2 } \times V_{ 2 } )
            =
            \underbrace{ ( U_{ 1 } \cap U_{ 2 } ) }_{ \emptyset }
            \times 
            \underbrace{ ( V_{ 1 } \cap V_{ 2 } ) }_{ \emptyset }
            =
            \emptyset
        \end{equation*}
        Thus, it follows that \( X \times Y \) is Hausdorff.
    \end{proofBox}
\end{thmBox}