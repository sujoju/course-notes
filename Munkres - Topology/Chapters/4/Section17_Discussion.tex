\begin{remarkBox}{Topologies and Closed Sets}
    As we saw with open sets, a set closed in \( Y \) can also be closed in 
    \( X \) -- however, it is not always the case.
    There is, though, a special situation in which every set closed in \( Y \)
    is also closed in \( X \) -- this is [\hyperlink{thm:17.3}{Theorem 17.3}].
\end{remarkBox}

\begin{remarkBox}{Some Convenient Terminology}
    We shall say that a set \( A \) \textbf{intersects} a set \( B \) if the 
    intersection \( A \cap B \) is not empty.
\end{remarkBox}

\begin{remarkBox}{On the Alternative Characterization}
    If the topology on \( X \) is generated by a basis, then we have that the
    [\hyperlink{thm:17_alt_char}{alternative characterizations}] all hold if 
    the term \textit{open neighborhood} is replaced in each occurrence by the 
    term \textit{basic open neighborhood}, which is an open neighborhood that 
    is also a basic open set.

    \baseSkip 

    In fact, this replacement holds true all the time -- that is, whenever
    we are working with open sets, we can equivalently work with basic open
    sets.
\end{remarkBox}

\begin{remarkBox}{On [\hyperlink{thm:17.8}{Theorem 17.8}]}
    Notice that [\hyperlink{thm:17.8}{Theorem 17.8}] essentially tells us that 
    if \( X \) is a Hausdorff topological space, then \( X \) is also a 
    \( T_{ 1 } \)-space.
\end{remarkBox}

\begin{remarkBox}{On the Definition of \( T_{ 1 } \)}
    The definition of \( T_{ 1 } \) can instead be stated to as follows:
    a \( T_{ 1 } \)-space is a topological space in which all singleton
    subsets are closed.

    \baseSkip 

    Indeed, we see that any finite point sets can be represented as the union 
    of singleton sets. 
    Thus, if the singleton sets are closed, then so are finite point sets.
\end{remarkBox}