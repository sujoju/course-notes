\begin{egBox}{Some Examples of Closed Sets}[eg:17.1]
    The subset \( [ a, b ] \) of \( \mathbb{R} \) is closed because its 
    complement 
    \begin{equation*}
        \mathbb{R} \setminus [ a, b ]
        =
        ( - \infty, a ) \cup ( b, +\infty )
    \end{equation*}
    is open. 
    Similarly, we see that \( [ a, +\infty ) \) is closed because its 
    complement \( ( - \infty, a ) \) is open.
    Notice that \( [ a, b ) \) of \( \mathbb{R} \) is neither open nor 
    closed.

    \baseRule 

    In the place \( \mathbb{R}^{ 2 } \), the set 
    \begin{equation*}
        \{ x \times y \mid x \geq 0 \text{ and } y \geq 0 \}
    \end{equation*}
    is closed because its complement is the union of the two sets 
    \begin{equation*}
        ( -\infty, 0 ) \times \mathbb{R}
        \quad \mathrm{and} \quad
        \mathbb{R} \times ( - \infty, 0 )
    \end{equation*}
    each of which is a product of open sets of \( \mathbb{R} \) and is 
    therefore open in \( \mathbb{R}^{ 2 } \).

    \baseRule

    In the finite complement topology on a set \( X \), the closed sets consists
    of \( X \) itself and all finite subsets of \( X \).

    \baseRule

    In the discrete topology on the set \( X \), every set is open. 
    Thus it follows that every set is closed as well.
\end{egBox}

\begin{egBox}{Trivial Clopen Sets}[eg:17.2]
    Given any topological space \( X \), we know from [\hyperlink{thm:17-1}{Theorem 17.1}] that \( \emptyset \) and \( X \) are clopen. 
    We call these the \textbf{trivial clopen subsets}.
\end{egBox}

\begin{egBox}{On the Interior and Closure of a Set}[eg:17.3]
    Let's say that we are given a subset \( A \) of a topological space \( X \).
    It is clear to see that \( \mathrm{Int} \ A \) is an open set (an arbitrary
    union of open sets is still open).
    It is also clear to see that \( \overline{ A } \) is a closed set (an 
    arbitrary intersection of closed sets is still closed).

    \baseSkip

    Furthermore, we see by definition that 
    \begin{equation*}
        \mathrm{Int} \ A \subset A \subset \overline{ A }
    \end{equation*}
    If \( A \) is open, then we see that \( A = \mathrm{Int} \ A \).
    Indeed, 
    \( A \) being open means that it is in the union of all open sets contained 
    in \( A \) (note that \( A \) contains itself).
    Thus, the union of all open sets contained in \( A \) must equal \( A \)
    -- the union of any subsets of \( A \) cannot be larger than \( A \) itself!
    
    \baseSkip 
    
    Similarly, if \( A \) is closed, then we see that \( A = \overline{ A } \).
    Indeed, \( A \) being closed means that it is in the intersection of all 
    closed sets containing \( A \).
    Thus, the intersection of all closed sets containing \( A \) must equal 
    \( A \) -- the intersection of a collection of sets cannot be larger than 
    each of the individual sets!
\end{egBox}

\begin{egBox}{The Boundary of a Set is Closed in \( X \)}[eg:17.4]
    Let \( A \) be a subset of a topological space \( X \).
    We want to show that \( \partial A \) is closed in \( X \).
    To do so, we note that
    \begin{equation*}
        \partial A 
        =
        \overline{ A } \setminus \mathrm{Int} \ A
        =
        \overline{ A } \cap ( X \setminus \mathrm{Int} \ A )
    \end{equation*}
    Further notice that \( X \setminus \mathrm{Int} \ A \) is closed since 
    \( X \setminus ( X \setminus \mathrm{Int} \ A ) = \mathrm{Int} \ A \) is 
    open.
    Since arbitrary intersections of closed sets are still closed, we see that 
    \( \partial A \) is closed as well.
\end{egBox}

\begin{egBox}{Some Examples of Closures of Sets}[eg:17.5]
    Let \( X = \mathbb{R} \). 
    If \( A = ( 0, 1 ] \), then \( \overline{ A } = [ 0, 1 ] \).
    This follows from seeing that every neighborhood of \( 0 \) intersects 
    \( A \), while every point outside \( [ 0, 1 ] \) has a neighborhood
    disjoint from \( A \). Similar arguments apply to the following subsets
    of \( X \).

    \baseRule

    If \( B = \{ \frac{ 1 }{ n } \mid n \in \mathbb{Z}_{ + } \} \), then 
    we have that \( \overline{ B } = \{ 0 \} \cup B \).
    
    \baseRule

    If \( C = \{ 0 \} \cup ( 1, 2 ) \), then \( \overline{ C } = \{ 0 \} \cup 
    [ 1, 2 ] \).

    \baseRule

    If \( \mathbb{Q} \) is the set of rational numbers, then \( \overline{ Q }
    = \mathbb{R} \).

    \baseRule

    If \( \mathbb{Z}_{ + } \) is the set of positive integers, then
    \( \overline{ Z }_{ + } = \mathbb{Z}_{ + } \).

    \baseRule

    If \( \mathbb{R}_{ + } \) is the set of positive reals, then closure of 
    \( \mathbb{R}_{ + } \) is the set \( \mathbb{R}_{ + } \cup \{ 0 \} \).

    \baseRule

    Consider the subspace \( Y = ( 0, 1 ] \) of the real line \( \mathbb{R} \).
    The set \( A = ( 0, \frac{ 1 }{ 2 } ) \) is a subset of \( Y \).
    Its closure in \( \mathbb{R} \) is the set \( [ 0, \frac{ 1 }{ 2 } ] \), and
    its closure in \( Y \) is the set \( [ 0, \frac{ 1 }{ 2 } ] \cap Y =
    ( 0, \frac{ 1 }{ 2 } ] \).
\end{egBox}

\begin{egBox}{Examples of Limit Points}[eg:17.6]
    Consider the real line \( \mathbb{R} \).
    If \( A = ( 0, 1 ] \), then the point \( 0 \) is a limit point of \( A \)
    and so is the point \( \frac{ 1 }{ 2 } \) -- in fact, every point of the 
    interval \( [ 0, 1 ] \) is a limit point of \( A \), but no other points of \( \mathbb{R} \) is a limit point of \( A \).

    \baseRule

    If \( B = \{ \frac{ 1 }{ n } \mid n \in \mathbb{Z}_{ + } \} \), then 
    \( 0 \) is the only limit point of \( B \).
    Every other point \( x \) of \( \mathbb{R} \) has a neighborhood that either
    does not intersect \( B \) at all (in particular, for \( x \notin B \) and 
    \( x \neq 0 \)), or it intersects \( B \) only in the points \( x \) itself
    (in particular, for \( x \in B \) and \( x \neq 0 \)).

    \baseRule

    If \( C = \{ 0 \} \cup ( 1, 2 ) \), then the limit points of \( C \) are the
    points of the interval \( [ 1, 2 ] \).

    \baseRule

    If \( \mathbb{Q} \) is the set of rational numbers, every point of 
    \( \mathbb{R} \) is a limit point of \( \mathbb{Q} \) -- this follows from 
    \( \mathbb{Q} \) being dense in \( \mathbb{R} \).

    \baseRule

    If \( \mathbb{Z}_{ + } \) is the set of positive integers, no point of
    \( \mathbb{R} \) is a limit point of \( \mathbb{Z}_{ + } \).

    \baseRule

    If \( \mathbb{R}_{ + } \) is the set of positive reals, then every point of 
    \( \{ 0 \} \cup \mathbb{R}_{ + } \) is a limit point of 
    \( \mathbb{R}_{ + } \).
\end{egBox}

\begin{egBox}{\( T_{ 1 } \)-Spaces}[eg:17.7]
    We have that \( \mathbb{R} \) with the cofinite topology is a \( T_{ 1 } \)-
    space.
    This is because in the cofinite topology, sets whose complements are finite
    are defined to be open -- meaning that set that are finite are closed in 
    the cofinite topology.
    Hence, \( \mathbb{R} \) with the cofinite topology is a \( T_{ 1 } \)-space.

    \baseRule

    \( \mathbb{R} \) with the trivial topology (where the only open subsets
    are \( \emptyset \) and \( \mathbb{R} \) itself) is not a \( T_{ 1 } \)-
    space.
    This is because we have that \( \emptyset \) and \( \mathbb{R} \) are the 
    only closed sets in the trivial topology. 
    Thus, singleton sets (and thus, finite point sets) are not closed.

    \baseRule

    \( \mathbb{R} \) with the lower limit topology is \( T_{ 1 } \).
    Because the lower limit topology is finer than the standard topology, we 
    have that every subset of \( \mathbb{R} \) that is closed in the standard
    topology is also closed in the lower limit topology.
    Since singleton sets are closed in the standard topology, they are also
    closed in the lower limit topology as well.
    Hence \( T_{ 1 } \).
\end{egBox}

\begin{egBox}{Hausdorff Spaces}[eg:17.8]
    Given a metric space \( X \) with the metric topology, we have that \( X \)
    is Hausdorff.
    To show why, we need to show that for all distinct points \( x, y \in X \),
    there exists an open neighborhood \( U \) and \( V \) of \( x \) and \( y \)
    ,respectively, so that \( U \cap V \emptyset \).
    Let us define \( d \equiv d( x, y ) \).
    It can be shown easily that \( B_{ \frac{ d }{ 2 } } ( x ) \cap 
    B_{ \frac{ d }{ 2 } } ( y ) = \emptyset \), which tells us that \( X \)
    is Hausdorff.

    \baseRule

    If we now consider \( \mathbb{R} \) with the cofinite topology, then we have
    that it is not Hausdorff.
    The claim is that every pair of cofinite sets intersects nonemptily.
    To show why, let \( F \) and \( G \) be finite sets, which means that 
    \( \mathbb{R} \setminus F \) and \( \mathbb{R} \setminus G \) are cofinite.
    DeMorgan's law tells us that 
    \begin{equation*}
        ( \mathbb{R} \setminus F ) \cap ( \mathbb{R} \setminus G )
        =
        \mathbb{R} \setminus ( F \cup G )
    \end{equation*}
    Notice that \( F \cup G \) is still finite, meaning that 
    \( \mathbb{R} \setminus ( F \cup G ) \neq \emptyset \).
    Thus, \( \mathbb{R} \) with the cofinite topology is not Hausdorff.
\end{egBox}

\begin{egBox}{Converging Sequences}[eg:17.9]
    In Hausdorff spaces, we have shown in [\hyperlink{thm:17.10}{Theorem 17.10}]
    that all sequences converge to at most one point in that space.
    However, for arbitrary topological spaces, it is entirely possible that 
    a single sequence converges to multiple points.

    \baseSkip

    Take \( X = \mathbb{R} \) with the trivial topology and the sequence 
    \( ( 1, 1, 1, \ldots ) \).
    We have that the sequences converges to \( 1 \), but also to \( 2 \) --
    in fact, the sequences converges to any point in \( \mathbb{R} \)!
    This is because the only open neighborhood that contains each element in the
    sequence (that is, \( 1 \)) is \( \mathbb{R} \).
    Thus, it follows by the definition of [\hyperlink{def:17:converge}
    {convergence}] that our sequence converges to any point in \( \mathbb{R} \).
\end{egBox}