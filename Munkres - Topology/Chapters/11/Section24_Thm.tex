\begin{thmBox}{24.1}[thm:24.1]
    If \( L \) is a linear continuum in the order topology, then \( L \) is 
    connected, and so are the intervals and rays in \( L \).

    \baseRule

    \begin{proofBox}

    \end{proofBox}
\end{thmBox}

\begin{thmBox}[Corollary]{24.2}[cor:24.2]
    The real line \( \mathbb{R} \) is connected and so are intervals and rays 
    in \( \mathbb{R} \).

    \baseRule

    \begin{proofBox}

    \end{proofBox}
\end{thmBox}

\begin{thmBox}{24.3 (Intermediate Value Theorem)}[thm:24.3]
    Let \( f: X \rightarrow Y \) be a continuous map, where \( X \) is a 
    connected space and \( Y \) is an ordered set in the order topology.
    If \( a \) and \( b \) are two points of \( X \) and it \( r \) is a point
    of \( Y \) lying between \( f ( a ) \) and \( f ( b ) \), then
    there exists a point \( c \) of \( X \) such that \( f ( c ) = r \)

    \baseSkip 

    Note that the intermediate value theorem of calculus is the special case of
    this theorem that occurs when we take \( X \) to be a closed interval in 
    \( \mathbb{R} \) and \( Y \) to be \( \mathbb{R} \).

    \baseRule

    \begin{proofBox}

    \end{proofBox}
\end{thmBox}

\begin{thmBox}{Path Connected Implies Connected}[thm:24.5]
    A path-connected space \( X \) is connected.

    \baseRule

    \begin{proofBox}
        Towards a contradiction, let's assume that \( X \) is not connected.
        This means that there exists a separation of \( X \).
        Let \( A \) and \( B \) form this separation.
        Now, let \( f: [ a, b ] \rightarrow X \) be any path in \( X \).
        We know from [\hyperlink{thm:23.5}{Theorem 23.5}] that the image of a
        connected space under a continuous map is connected -- that is,
        we have that \( f ( [ a, b ] ) \) is connected.
        By [\hyperlink{lem:23.2}{Lemma 23.2}], we get that \( f ( [ a, b ] ) \)
        must be entirely contained in either \( A \) or \( B \).
        Thus, we have that there is no path in \( X \) joining a point of 
        \( A \) to a point of \( B \), which is a contradiction to our 
        assumption that \( X \) is path-connected.

        \baseRule

        Another way to prove this theorem is as follows:
        Suppose that there is a non-trivial clopen subset \( A \) of \( X \).
        We shall pick \( p \in A \) and \( q \in X \setminus A \).
        Since we know that \( X \) is path-connected, we see that there exists
        a path \( \gamma: [ a, b ] \rightarrow X \) such that 
        \( \gamma ( a ) = p \) and \( \gamma ( b ) = q \).
        We can see that \( \gamma^{ -1 } ( A ) \) is clopen in \( [ a, b ] \):
        since \( \gamma \) is continuous and \( A \) is clopen (hence, open), 
        we see that \( \gamma^{ -1 } ( A ) \) is open as well; also, since
        \( \gamma \) is continuous and \( A \) is clopen (hence, closed), we 
        see that \( \gamma^{ -1 } ( A ) \) is closed as well by 
        [\hyperlink{thm:18.2}{Theorem 18.2}].
        Furthermore, we see that \( \gamma^{ -1 } ( A ) \) is non-trivial
        since \( A \) is non-trivial. 
        
        \baseSkip

        Thus, we have that \( \gamma^{ -1 } ( A ) \) is a non-trivial clopen
        subset of \( [ a, b ] \).
        However this contradicts the connectedness of \( [ a, b ] \)
        Thus, \( X \) must be connected.
    \end{proofBox}
\end{thmBox}

\begin{thmBox}{Image of Path-Connected Space is Path-Connected}[thm:24.6]
    Let \( f: X \rightarrow Y \) be continuous.
    Equip \( f ( X ) \subset Y \) with the subspace topology.
    If \( X \) is path-connected, then \( f ( X ) \) is path-connected.

    \baseRule

    \begin{proofBox}
        Suppose that we have a pair of points \( x, y \in f ( X ) \).
        We want to show that there exists a continuous path connecting these
        two points.
        Since, \( x, y \in f ( X ) \), it follows that there exists some 
        \( x', y' \in X \) such that \( x = f ( x' ) \) and \( y = f ( y' ) \).
        Because \( X \) is path-connected, we have that there is a continuous
        map \( \gamma: [ a, b ] \rightarrow X \) such that 
        \( \gamma ( a ) = x' \) and \( \gamma ( b ) = y' \).

        \baseSkip

        It is here that we notice that \( f \circ \gamma \) is continuous 
        (since the composition of two continuous functions is still continuous)
        such that 
        \begin{equation*}
            ( f \circ \gamma ) ( a ) = x
            \quad \mathrm{and} \quad 
            ( f \circ \gamma ) ( b ) = y
        \end{equation*}
        which tells us that \( f \circ \gamma \) is a continuous path on 
        \( f ( X ) \) connecting the two points \( x \) and \( y \).
    \end{proofBox}
\end{thmBox}

\begin{thmBox}[Corollary]{}[cor:24.7]
    Connectedness and path-connectedness are 
    [\hyperlink{18_homeomorphism_invariant}{hoemomorphism invariants}].

    \baseRule

    \begin{proofBox}
        Let \( X \) be homeomorphic to \( Y \) and \( X \) be connected.
        We want to show that \( Y \) must be connected as well.
        Since \( X \) and \( Y \) are homeomorphic, there exists a 
        homeomorphism \( f: X \rightarrow Y \).
        Since \( f \) is bijective (namely, surjective), we see that 
        \( f ( X ) = Y \).
        Since \( X \) is connected and \( f \) is continuous, we see that 
        \( f ( X ) = Y \) must be connected as well.

        \baseSkip

        The argument for path-connectedness can be made by replacing the word 
        "connected" above with "path-connected" instead.
    \end{proofBox}
\end{thmBox}