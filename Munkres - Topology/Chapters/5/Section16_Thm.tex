\begin{thmBox}[Lemma]{16.1}[lem:16.1]
    If \( \mathcal{B} \) is a basis for the topology of \( X \) then the 
    collection 
    \begin{equation*}
        \mathcal{B}_{ Y }
        =
        \{ Y \cap B \mid B \in \mathcal{B} \}
    \end{equation*}
    is a basis for the subspace topology on \( Y \).

    \baseRule

    \begin{proofBox}
        We start by noting that \( \mathcal{B}_{ Y } \) is a collection of open 
        sets of \( Y \), where the topology on \( Y \) is the subspace topology.
        From here, we shall utilize [\hyperlink{lem:13.2}{Lemma 13.2}]; our goal
        is to show that for each open set \( Y \cap U \) of \( Y \) (where 
        \( U \) is an open set of \( X \)) and each \( y \in Y \cap U \), there 
        is an element \( Y \cap B \in \mathcal{B}_{ Y } \) such that 
        \( y \in Y \cap B \subset Y \cap U \).

        \baseSkip

        Let's say that we are given any \( U \) open in \( X \), \( Y \cap U \)
        open in \( Y \), and \( y \in Y \cap U \).
        By the definition of openness in \( X \) (under the basis 
        \( \mathcal{B} \)), we have that there is a basis element 
        \( B \in \mathcal{B} \) such that \( y \in B \subset U \).

        \baseSkip 
        
        Notice that \( B \subset U \) implies that \( Y \cap B \subset 
        Y \cap U \).
        Furthermore, knowing that \( y \in Y \cap U \) and 
        \( y \in B \subset U \) implies as well that \( y \in Y \cap B \subset 
        Y \cap U \).
        Thus, by [\hyperlink{lem:13.2}{Lemma 13.2}], we have that 
        \( \mathcal{B}_{ Y } \) is a basis for the subspace topology on \( Y \).
    \end{proofBox}
\end{thmBox}

\begin{thmBox}[Lemma]{16.2}[lem:16.2]
    Let \( Y \) be a subspace of \( X \). 
    If \( U \) is open in \( Y \) and \( Y \) is open in \( X \), then 
    \( U \) is open in \( X \).

    \baseRule

    \begin{proofBox}
        Since \( U \) is open in \( Y \), we have that there exists some \( V \)
        open in \( X \) so that \( U = Y \cap V \).
        Since \( Y \) and \( V \) are both open in \( X \), and \( X \) is a 
        topological space, we have that \( Y \cap V \) is open in \( X \) as 
        well (finite intersection of open sets is open).
        Thus, we have that \( U \) is open in \( X \).
    \end{proofBox}
\end{thmBox}

\begin{thmBox}{16.3}[thm:16.3]
    If \( A \) is a subspace of \( X \) and \( B \) is a subspace of \( Y \),
    then the product topology on \( A \times B \) is the same as the topology
    \( A \times B \) inherits as a subspace of \( X \times Y \).

    \baseRule

    \begin{proofBox}
        Let us start by examining the general basis element for the subspace
        topology on \( A \times B \).
        Such a basis element if of the following form: \( ( A \times B ) \cap 
        ( U \times V ) \), where \( U \times V \) is the general basis element
        for \( X \times Y \), where \( U \) is open in \( X \) and \( V \) is
        open in \( Y \).
        We can see that 
        \begin{equation*}
            ( A \times B ) \cap ( U \times V )
            =
            ( A \cap U ) \times ( B \cap V )
        \end{equation*}
        Now since \( A \cap U \) and \( B \cap V \) are the general open sets
        for the subspace topologies on \( A \) and \( B \), respectively, 
        we have the set \( ( A \cap U ) \times ( B \cap V ) \) to be the
        general basis element for the product topology on \( A \times B \).

        \baseSkip

        From this, we can conclude that the bases for the subspace topology on
        \( A \times B \) and for the product topology on \( A \times B \) are 
        the same.
        Hence, the topologies are the same as well.
    \end{proofBox}
\end{thmBox}

\begin{thmBox}{16.4}[thm:16.4]
    Let \( X \) be an ordered set in the order topology; 
    let \( Y \) be a subset of \( X \) that is convex in \( X \).
    Then the order topology on \( Y \) is the same as the topology \( Y \) 
    inherits as a subspace of \( X \).

    \baseRule

    \begin{proofBox}

    \end{proofBox}
\end{thmBox}

\begin{thmBox}{Closures and Interiors in Subspaces}[thm:16_clo_int_subspace]
    Let \( X \) be a topological space and let \( A \subset U \subset X \).
    The following hold:
    \begin{enumerate}[label = (\alph*)]
        \item If \( U \) is open in \( X \), then the interior of \( A \) in 
            \( U \) equals the interior of \( A \) in \( X \).
        \item If \( U \) is closed in \( X \), then the closure of \( A \) in 
            \( U \) equals the closure of \( A \) in \( X \)
    \end{enumerate}

    \baseRule

    \begin{proofBox}*
        \wrapBox{a}
        Let us start by noting that \( \mathrm{Int} \ A \) is defined to be the 
        union of all open subsets (either in \( U \) or \( X \)) contained in 
        \( A \). 
        We shall start by supposing that we have some open subset \( V \) in 
        \( U \) that is contained in \( A \).
        We want to show that \( V \) is also an open subset in \( X \) contained
        in \( A \). 
        It is immediate to see that the containment condition is trivially met,
        so we just need to show that \( V \) is open in \( X \).
        Note that being open in \( U \) (which is a subspace of \( X \)) means 
        that there exists some open set \( W \) of \( X \) such that 
        \( V = U \cap W \).
        Since both \( U \) and \( W \) are open sets of \( X \), and finite 
        intersections of open sets are open, we have that \( U \cap W \) (hence 
        \( V \)) is open in \( X \).
        Thus, we have that 
        \begin{equation*}
            \mathrm{Int} \ A \text{ in } U
            \subset 
            \mathrm{Int} \ A \text{ in } X
        \end{equation*}

        \baseSkip

        Let us now suppose that we have some open subset \( Z \) in \( X \) 
        that is contained in \( A \).
        We want to show that \( Z \) is also an open subset in \( U \) 
        contained in \( A \).
        Again, the containment condition is trivially met by construction, so 
        we just need to show that \( Z \) is open in \( U \).
        Because \( Z \) is contained in \( A \), which is a subset of \( U \) 
        (i.e., \( Z \subset A \subset U \)), we have that \( U \cap Z = Z \).
        Therefore, we see that \( Z \) itself is open in \( U \).
        Thus, we have that 
        \begin{equation*}
            \mathrm{Int} \ A \text{ in } U
            \supset 
            \mathrm{Int} \ A \text{ in } X
        \end{equation*}
        Putting everything together gives us
        \begin{equation*}
            \mathrm{Int} \ A \text{ in } U
            =
            \mathrm{Int} \ A \text{ in } X
        \end{equation*}

        \baseSkip

        \wrapBox{b}
        Let us now let \( U \) to be closed in \( X \).
        Furthermore, let us define the following:
        \begin{equation*}
            \mathrm{Cl}_{ U } \ A
            =
            \text{ closure of \( A \) in \( U \)}
            \quad \mathrm{and} \quad 
            \mathrm{Cl}_{ X } \ A
            =
            \text{ closure of \( A \) in \( X \)}
        \end{equation*}
        We first want to show that \( \mathrm{Cl}_{ U } \ A \subset 
        \mathrm{Cl}_{ X } \ A \).
        To do so, notice that \( \mathrm{Cl}_{ X } \ A \) and \( U \) are both 
        closed in \( X \) and contain \( A \). 
        Thus, it follows that \( \mathrm{Cl}_{ X } \ A \cap U \) is closed in 
        \( U \) under the subspace topology and contains \( A \)
        (c.f. [\hyperlink{thm:17.4}{Theorem 17.2}]).
        By definition, we have that \( \mathrm{Cl}_{ U } \ A \) is the smallest
        set closed in \( U \) containing \( A \), meaning that 
        \( \mathrm{Cl}_{ U } \ A \subset \mathrm{Cl}_{ X } \ A \cap U \).
        
        \baseSkip

        From here, we note that \( \mathrm{Cl}_{ X } \ A \cap U \subset 
        \mathrm{Cl}_{ X } \ A \) since the intersection must be contained in 
        each of the set that are being intersected.
        Putting everything together results in
        \begin{equation*}
            \mathrm{Cl}_{ U } \ A 
            \subset 
            \mathrm{Cl}_{ X } \ A \cap U 
            \subset 
            \mathrm{Cl}_{ X } \ A
        \end{equation*}

        \baseSkip 

        We now want to show that \( \mathrm{Cl}_{ X } \ A \subset 
        \mathrm{Cl}_{ U } \ A \).
        We start by noting that \( \mathrm{Cl}_{ U } \ A \) is closed in 
        \( U \), and that \( U \) is closed in \( X \).
        Thus, by the transitivity of closedness, we see that 
        \( \mathrm{Cl}_{ U } \ A \) is closed in \( X \).
        From here, we note that \( A \subset \mathrm{Cl}_{ U } \ A \), which 
        tells us that \( \mathrm{Cl}_{ U } \ A \) is a closed set in \( X \)
        that also contains \( A \).
        Since \( \mathrm{Cl}_{ X } \ A \) is defined to be the smallest closed 
        set in \( X \) that also contains \( A \), it follows that 
        \( \mathrm{Cl}_{ X } \ A \subset \mathrm{Cl}_{ U } \ A \).
        Putting everything together gives us
        \begin{equation*}
            \mathrm{Cl}_{ U } \ A
            =
            \mathrm{Cl}_{ X } \ A
        \end{equation*}

    \end{proofBox}
\end{thmBox}