\begin{remarkBox}{Closed Set in the Subspace Topology}
    Notice that [\hyperlink{thm:17.2}{Theorem 17.2}] gives us an alternative 
    definition of
    a closed set in the subspace topology: the closed subsets of 
    \( Y \) are exactly of the form \( Y \cap C \), where \( C \) is closed 
    in \( X \).
\end{remarkBox}

\begin{remarkBox}{Closure Notation}
    When dealing with a topological space \( X \) and a subspace \( Y \), we
    must be careful in taking closures of sets.
    If \( A \) is a subset of \( Y \), the closure of \( A \) in \( Y \) and 
    the closure of \( A \) in \( X \) will in general be different.
    In such a situation, we reserve the notation \( \overline{ A } \) to stand
    for the closure of \( A \) in \( X \).

    \baseSkip

    However, we have that [\hyperlink{thm:17.4}{Theorem 17.4}] tells us that the
    closure of \( A \) in \( Y \) can be expressed in terms of 
    \( \overline{ A } \).
\end{remarkBox}

\begin{remarkBox}{Topologies and Open Sets}
    Note a set open in \( Y \) can also be open in \( X \)
    -- however, it is not always the case.
    There is, though, a special situation in which every set open in \( Y \)
    is also open in \( X \) -- this is [\hyperlink{lem:16.2}{Lemma 16.2}].
\end{remarkBox}