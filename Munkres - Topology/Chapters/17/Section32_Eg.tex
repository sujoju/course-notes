\begin{egBox}{\( \mathbb{R}_{ \ell } \) is normal}[eg:32.1]
    Let two disjoint closed sets \( A, B \) of \( \mathbb{R}_{ \ell } \) be 
    given.
    To show that \( \mathbb{R}_{ \ell } \) is normal, we first need to show 
    that it is \( T_{ 1 } \); indeed, we have already shown this to be the 
    case.
    Now, because \( X \setminus B \) is open and contains \( A \), we have that
    for all \( a \in A \), there exists \( x_{ a } > a \) so
    \( [ a, x_{ a } ) \subset X \setminus B \).
    Let us define the following:
    \begin{equation*}
        U
        =
        \bigcup_{ a \in A } [ a, x_{ a } ).
    \end{equation*}
    which we see is an open neighborhood of \( A \).
    Similarly, because \( X \setminus A \) is open and contains \( B \), we 
    have that for all \( b \in B \), there exists \( x_{ b } > b \) so
    \( [ b, x_{ b } ) \subset X \setminus A \).
    Let us define the following:
    \begin{equation*}
        V
        =
        \bigcup_{ b \in B } [ b, x_{ b } ).
    \end{equation*}
    which we see is an open neighborhood of \( B \).

    \baseSkip

    We claim that \( U \cap V = \emptyset \).
    Towards a contradiction, let's assume that \( U \cap B \neq \emptyset \).
    This mean that 
    \begin{equation*}
        [ a, x_{ a } ) \cap [ b, x_{ b } ) \neq \emptyset
    \end{equation*}
    WLOG, let's say that \( a < b \).
    If the intersection is nonempty, then this means that we must have the 
    following situation
    \begin{equation*}
        a < b < x_{ a } < x_{ b }
        \implies
        x_{ a } \in [ b, x_{ b } )
        \implies
        x_{ a } \in B
    \end{equation*}
    Which is a contradiction.
\end{egBox}

\begin{egBox}{The Sorgenfrey Plane is not normal}[eg:32.2]
    Recall that the Sorgenfrey Plane was defined to be 
    \( \mathbb{R}_{ \ell } \times \mathbb{R}_{ \ell } \).
    Let us look at the line \( L: y = -x \).
    As a subspace, we see that \( L \) has the discrete topology.
    We can see that \( L \) is closed in 
    \( \mathbb{R}_{ \ell } \times \mathbb{R}_{ \ell } \).
    Furthermore, every subset of \( L \) is closed in \( L \).
    So, by transitivity, every subset of \( L \) is closed in 
    \( \mathbb{R}_{ \ell } \times \mathbb{R}_{ \ell } \).
    Let \( A = \{ \text{points in } L \text{ with rational coordinates} \} \).
    Let \( B = \{ \text{points in } L \text{ with irrational coordinates} \} \).
    We can see that both \( A \) and \( B \) are disjoint and closed, but not separable by disjoint open neighborhoods.
    The technical explanation is given in Munkres.
\end{egBox}