\begin{defBox}{Basis for a Topology}[def:13_basis]
    If \( X \) is a set, then a \textbf{basis} for a topology on \( X \) is a collection \( \mathcal{B} \) of subsets \( X \) 
    (called \textbf{basis elements}) such that 
    \begin{enumerate}
        \item For each \( x \in X \), there is at least one basis element \( B \) containing \( x \).
        \item If \( x \) belongs to the intersection of two basis elements \( B_{ 1 } \) and \( B_{ 2 } \), then there is a 
            basis element \( B_{ 3 } \) containing \( x \) such that \( B_{ 3 } \subset B_{ 1 } \cap B_{ 2 } \).
    \end{enumerate}
\end{defBox}

\begin{defBox}{Topology Generated by a Basis}[def:13_generated_top]
    Given a basis \( \mathcal{B} \), we define the \textbf{topology} \( \mathcal{T} \) \textbf{generated by} \( \mathcal{B} \) 
    as follows: A subset \( U \) of \( X \) is said to be open in \( X \) (that is, \( U \in \mathcal{T} \)) if for each 
    \( x \in U \), there is a basis element \( B \in \mathcal{B} \) such that \( x \in B \) and \( B \subset U \).

    \baseSkip 

    Note that each basis element is itself an element of \( \mathcal{T} \).
\end{defBox}

\begin{defBox}{Topologies on \( \mathbb{R} \)}[def:13_R_top]
    If \( \mathcal{B} \) is the collection of all open intervals in 
    \( \mathbb{R} \), i.e., of the form 
    \begin{equation*}
        ( a, b )
        =
        \{ x \mid a < x < b \}
    \end{equation*}
    Then the topology generated by \( \mathcal{B} \) is called the 
    \textbf{standard topology} on \( \mathbb{R} \).

    \baseSkip 

    If \( \mathcal{B}' \) is the collection of all half-open intervals of the 
    form 
    \begin{equation*}
        [ a, b )
        =
        \{ x \mid a \leq x < b \}
    \end{equation*} 
    where \( a < b \), then the topology generated by \( \mathcal{B}' \) is 
    called the \textbf{lower limit topology} on \( \mathbb{R} \). 
    When \( \mathbb{R} \) is given the lower limit topology, we shall denote it 
    by \( \mathbb{R}_{ \ell } \).

    \baseSkip

    Finally, let \( K \) denote the set of all numbers of the form
    \( \frac{ 1 }{ n } \) for \( n \in \mathbb{Z}_{ + } \), and let 
    \( \mathcal{B}'' \) be the collection of all open intervals \( ( a, b ) \), 
    along with all sets of the form \( ( a, b ) \setminus K \).
    The topology generated by \( \mathcal{B}'' \) will be called the 
    \( K \)-\textbf{topology} on \( \mathbb{R} \).
    When \( \mathbb{R} \) is given this topology, we shall denote it by 
    \( \mathbb{R}_{ K } \).
\end{defBox}

\begin{defBox}{Subbasis Topology}[def:13_subbasis]
    A \textbf{subbasis} \( \mathcal{S} \) for a topology on \( X \) is a 
    collection of subsets of \( X \) whose union equals \( X \).

    \baseSkip

    The \textbf{topology generated by the subbasis} \( \mathcal{S} \) is defined to be the collection \( \mathcal{T} \) of all unions of finite 
    intersections of elements of \( \mathcal{S} \).
\end{defBox}