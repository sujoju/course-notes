\begin{thmBox}[Lemma]{13.1}[lem:13.1]
    Let \( X \) be a set; let \( \mathcal{B} \) be a basis for a topology 
    \( \mathcal{T} \) on \( X \). Then \( \mathcal{T} \) equals the collection 
    of all unions of elements of \( \mathcal{B} \). 

    \baseRule

    \begin{proofBox}
        Let's say that we are given any collection of elements of 
        \( \mathcal{B} \), which we shall denote as 
        \( \{ B_{ i } \}_{ i \in I } \) where \( I \) is some indexing set. 
        By definition, we have basis elements to be themselves an element of 
        \( \mathcal{T} \), which means that \( B_{ i } \in \mathcal{T} \) 
        for all \( i \in I \).
        Since \( \mathcal{T} \) is a topology, we have that the union of these 
        basis elements are also in \( \mathcal{T} \) -- that is, we have 
        \( \bigcup_{ i \in I } B_{ i } \in \mathcal{T} \).
        Because we have showed that the union of an arbitrary collection of 
        elements of \( \mathcal{B} \) was contained \( \mathcal{T} \), we have 
        that the collection of all unions of elements of \( \mathcal{B} \) is 
        a subset of \( \mathcal{T} \).
        
        \baseSkip 

        Conversely, let us now be given any \( U \in \mathcal{T} \), where 
        \( \mathcal{B} \) is the basis for \( \mathcal{T} \).
        Since \( U \) is open, we have that for each \( x \in U \), there is a 
        basis element \( B_{ x } \in \mathcal{B} \) such that 
        \( x \in B_{ x } \) and \( B_{ x } \subset U \).
        Since \( B_{ x } \subset \bigcup_{ x \in U } B_{ x } \), it follows 
        that every \( x \in U \) must also be contained in 
        \( \bigcup_{ x \in U } B_{ x } \) -- that is,
        \begin{equation*}
            U \subset \bigcup_{ x \in U } B_{ x }
        \end{equation*}
        Furthermore, since we have that \( B_{ x } \subset U \) for all 
        \( x \in U \), it follows as well that 
        \begin{equation*}
            \bigcup_{ x \in U } B_{ x } \subset U
        \end{equation*}
        which results in \( U = \bigcup_{ x \in U } B_{ x } \).
        I.e., we have that \( U \) equals a union of elements of 
        \( \mathcal{B} \), which tells us that \( \mathcal{T} \) is a subset of 
        the collection of all unions of elements of \( \mathcal{B} \).

        \baseSkip 

        Putting everything together proves our lemma.
    \end{proofBox}
\end{thmBox}

\begin{thmBox}[Lemma]{13.2}[lem:13.2]
    Let \( X \) be a topological space. Suppose that \( \mathcal{C} \) is a 
    collection of open sets of \( X \) such that for each open set \( U \) of 
    \( X \) and each \( x \in U \), there is an element \( C \in \mathcal{C} \)
    such that \( x \in C \subset U \). Then \( \mathcal{C} \) is a basis for
    the topology of \( X \). 

    \baseRule

    \begin{proofBox}
        The proof will come in two main parts: the first deals with showing 
        that \( \mathcal{C} \) is indeed a basis; the second deals with 
        showing that given a topology \( \mathcal{T} \) of \( X \), the 
        topology generated by \( \mathcal{C} \) equals \( \mathcal{T} \).

        \baseSkip 

        We need to show that \( \mathcal{C} \) is a basis. 
        The first condition is met easily: since \( X \) is itself an open set,
        the hypothesis tells us that for any \( x \in X \), there is an element 
        \( C \in \mathcal{C} \) such that \( x \in C \subset \mathcal{C} \).
        For the second condition, we let \( x \) to belong to 
        \( C_{ 1 } \cap C_{ 2 } \), where 
        \( C_{ 1 }, C_{ 2 } \in \mathcal{C} \).
        Since we have both \( C_{ 1 } \) and \( C_{ 2 } \) are open, it follows 
        that \( C_{ 1 } \cap C_{ 2 } \) is open as it is a finite intersection
        of open sets. 
        Therefore, we have by hypothesis again that we are able to find an 
        element \( C_{ 3 } \in \mathcal{C} \) such that 
        \( x \in C_{ 3 } \subset C_{ 1 } \cap C_{ 2 } \).
        
        \baseSkip

        Now we want to show that the topology generated by \( \mathcal{C} \), 
        say \( \mathcal{T}' \), equals \( \mathcal{T} \). 
        We first start by noting that if \( U \) belongs to \( \mathcal{T} \),
        then by definition, \( U \) must be open.
        If \( x \in U \), then the hypothesis tells us that there is an 
        element \( C \in \mathcal{C} \) such that \( x \in C \subset U \).
        Thus, it follows by definition of a topology being generated by a basis
        that \( U \in \mathcal{T}' \) -- i.e., we have 
        \( \mathcal{T} \subset \mathcal{T}' \).

        \baseSkip

        Conversely, let's say that \( W \in \mathcal{T}' \). By 
        [\hyperlink{lem:13.1}{Lemma 13.1}], we have that \( W \) equals a union
        of elements of \( \mathcal{C} \). Because we know that each element of
        \( \mathcal{C} \) belongs to \( \mathcal{T} \), and \( \mathcal{T} \) 
        is a topology, it follows that \( W \in \mathcal{T} \) -- 
        i.e., we have \( \mathcal{T'} \subset \mathcal{T} \).

        \baseSkip

        Putting everything together results in \( \mathcal{T} = \mathcal{T}' \).
    \end{proofBox}
\end{thmBox}

\begin{thmBox}[Lemma]{13.3}[lem:13.3]
    Let \( \mathcal{B} \) and \( \mathcal{B}' \) be bases for the topologies 
    \( \mathcal{T} \) and \( \mathcal{T}' \), respectively, on \( X \).
    Then the following are equivalent: 
    \begin{enumerate}
        \item \( \mathcal{T}' \) is finer and \( \mathcal{T} \)
        \item For each \( x \in X \) and each basis element 
            \( B \in \mathcal{B} \) containing \( x \), there is a basis element
            \( B' \in \mathcal{B}' \) such that \( x \in B' \subset B \).
    \end{enumerate}

    \baseRule

    \begin{proofBox}
        \baseSkip 

        \wrapBox{\( ( 2 ) \implies ( 1 ) \)}
        Recall that \( \mathcal{T}' \) being finer than \( \mathcal{T} \) means 
        that \( \mathcal{T}' \supset \mathcal{T} \).
        Thus, given an element \( U \in \mathcal{T} \), we wish to show that 
        \( U \in \mathcal{T}' \).
        Let \( x \in U \). 
        Since we know that \( \mathcal{B} \) generates \( \mathcal{T} \), there 
        is an element \( B \in \mathcal{B} \) such that 
        \( x \in B \subset U \). 
        The second condition of this lemma tells us that there exists an element
        \( B' \in \mathcal{B}' \) such that \( x \in B' \subset B \).
        Thus, we have \( B' \subset B \subset U \), and by the definition of a 
        topology being generated by a basis, we further have that 
        \( U \in \mathcal{T}' \).

        \baseSkip
        \wrapBox{\( ( 1 ) \implies ( 2 ) \)}
        Let's now say that we are given \( x \in X \) and 
        \( B \in \mathcal{B} \), with \( x \in B \). 
        Now \( B \) belongs to \( \mathcal{T} \) by definition and
        \( \mathcal{T} \subset \mathcal{T}' \) by the first condition;
        therefore, we have \( B \in \mathcal{T}' \).
        Since \( \mathcal{T}' \) is generated by \( \mathcal{B}' \), it follows 
        that there is an element \( B' \in \mathcal{B}' \) such that
        \( x \in B' \subset B \).
    \end{proofBox}
\end{thmBox}

\begin{thmBox}[Lemma]{13.4}[lem:13.4]
    The topologies of \( \mathbb{R}_{ \ell } \) and \( \mathbb{R}_{ K } \)
    are strictly finer than the standard topology on \( \mathbb{R} \), but are 
    not comparable with one another.

    \baseRule

    \begin{proofBox}
        We shall let \( \mathcal{T} \), \( \mathcal{T}' \), and 
        \( \mathcal{T}'' \) be the topologies of \( \mathbb{R} \), 
        \( \mathbb{R}_{ \ell } \) and \( \mathbb{R}_{ K } \), respectively. 
        Given a basis element \( ( a, b ) \) for \( \mathcal{T} \) and a point 
        \( x \in ( a, b ) \), we have that the basis element \( [ x, b ) \) for 
        \( \mathcal{T}' \) contains \( x \) and lies in \( ( a, b ) \).
        However, given the basis element \( [ x, d ) \) for \( \mathcal{T}' \),
        we find that there is no open interval \( ( a, b ) \) such that it 
        contains \( x \) and lies in \( [ x, d ) \) (\( a \) will always be to 
        the left of \( x \)).
        Thus, we find that \( \mathcal{T}' \) is strictly finer than 
        \( \mathcal{T} \).

        \baseSkip 

        We now focus our attention to \( \mathbb{R}_{ K } \) and 
        \( \mathbb{R} \). 
        Given a basis element \( ( a, b ) \) for \( \mathcal{T} \) and a point 
        \( x \) of \( ( a, b ) \), this same interval is a basis element for 
        \( \mathcal{T}'' \) that contains \( x \).
        However, given the basis element \( B = ( -1, 1 ) \setminus K \) for 
        \( \mathcal{T}'' \) and the point \( 0 \) of \( B \), 
        we find that there is 
        no open interval that contains \( 0 \) and lies in \( B \) -- any open 
        interval that contains \( 0 \) will contain numbers of the form 
        \( \frac{ 1 }{ n } \) for \( n \in \mathbb{Z}_{ + } \), which would 
        make it impossible for such an interval to lie in \( B \).
        Thus, we find that \( \mathcal{T}'' \) is strictly finer than 
        \( \mathcal{T} \).

        \baseSkip 

        Finally, we want to show that the topologies of 
        \( \mathbb{R}_{ \ell } \) and \( \mathbb{R}_{ K } \) are not 
        comparable.
        To do so, we just need to find an open set in each that is not open in 
        the other.
        To start, we see that \( [ 2, 3 ) \) is open in 
        \( \mathbb{R}_{ \ell } \), but not in \( \mathbb{R}_{ K } \).
        Furthermore, we have that \( \mathbb{R} \setminus K \) is open in 
        \( \mathbb{R}_{ K } \), but not in \( \mathbb{R}_{ \ell } \) since 
        every open set containing \( 0 \) contains numbers of the form 
        \( \frac{ 1 }{ n } \) for \( n \in \mathbb{Z}_{ + } \).
    \end{proofBox}
\end{thmBox}