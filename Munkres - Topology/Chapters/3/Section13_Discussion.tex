\begin{remarkBox}{On Basis}
    Basis can be referred to as a \textbf{basis of subsets}, and their 
    elements can also be referred to as a \textbf{basic open set}.

    \baseSkip   

    Furthermore, the first and second condition for a basis is sometimes    referred to as \textbf{fullness} and \textbf{smallness}, respectively.
\end{remarkBox}

\begin{remarkBox}{Why do we care about working with a basis?}
    For each of the examples that we discussed in Section 12, we were able to specify the topology by describing the entire 
    collection \( \mathcal{T} \) of open sets. However, this is usually too difficult -- in most cases, one specifies instead 
    a smaller collection of subsets \( X \) (i.e., a basis) and defines the topology in terms of being generated by such a basis.
\end{remarkBox}

\begin{remarkBox}{On Open Sets}
    When we are working with open sets, we \textbf{always} need to ask which 
    with respect to which basis our sets are open.
\end{remarkBox}

\begin{remarkBox}{Openness in a basis and its implications}
    When working with a topology, we have by definition its elements to be open.
    However, when we are working with a basis, we now have a condition for 
    openness.
    Thus, it makes sense for a topology to be generated by a basis, since the 
    condition for openness inherited by a basis is what will generate which 
    sets within \( X \) will be open or not -- i.e., the choice of basis will 
    determine which sets are open because of the condition for openness that 
    we associate with such a basis, which will further determine what the 
    topology on \( X \) will be.
\end{remarkBox}

\begin{remarkBox}{Standard Topology on \( \mathbb{R} \)}
    Whenever we consider \( \mathbb{R} \), we shall suppose the topology on 
    \( \mathbb{R} \) to be the standard one unless we specifically state 
    otherwise.
\end{remarkBox}

\begin{remarkBox}{On [\hyperlink{lem:13.1}{Lemma 13.1}]}
    This lemma states that every open set \( U \) in \( X \) can be expressed 
    as a unions of basis elements.
    However, the expression for \( U \) is not unique.
\end{remarkBox}

\begin{remarkBox}{On [\hyperlink{lem:13.2}{Lemma 13.2}]}
    This lemma is particularly important as it allows for us to obtain a basis 
    for a given topology.

    \baseSkip

    The main point is that a basis should include open sets that are 
    \textit{small} enough to capture arbitrary nearness to any given point.
    Thus, every open neighborhood of \( x \) contains a basic open
    neighborhood of \( x \).
\end{remarkBox}

\begin{remarkBox}{On [\hyperlink{lem:13.3}{Lemma 13.3}]}
    The main importance of this lemma is that it allows for us to determine 
    whether one topology is finer than another. 

    \baseSkip 

    The direction of the inclusion may be difficult to remember.
    It may be easier to remember if we recall the analogy between
    a topological space and a truckload full of gravel. 
    Think of the pebbles as the basis elements of the topology; 
    after the pebbles are smashed to dust, the dust particles are
    the basis elements of the new topology. 
    The new topology is finer than the old one, and each dust particle was 
    contained inside a pebble, as the criterion states.
\end{remarkBox}