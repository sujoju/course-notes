%-------------------------------------------------------------------------------
%	PACKAGES
%-------------------------------------------------------------------------------

\usepackage{fontspec}
\usepackage[T1]{fontenc}
\usepackage{kpfonts}
\usepackage[skins, breakable, many]{tcolorbox}
\usepackage{paracol}
\usepackage{geometry}
\usepackage{mathtools}
\usepackage{microtype}
\usepackage{siunitx}
\usepackage{array}
\usepackage{enumitem}
\usepackage{xparse}
\usepackage{caption}
\usepackage{import}
\usepackage{fancyhdr}
\usepackage{titlesec}
\usepackage{tikz}
\usepackage{tikz-cd}
\usepackage{float}
\usepackage{xcolor}
\usepackage{hyperref}

\allowdisplaybreaks

%-------------------------------------------------------------------------------
%	COLORS
%-------------------------------------------------------------------------------

\definecolor{xred}{HTML}{BD4242}
\definecolor{xblue}{HTML}{4268BD}
\definecolor{xgreen}{HTML}{52B256}
\definecolor{xpurple}{HTML}{7F52B2}
\definecolor{xorange}{HTML}{FD9337}
\definecolor{xdotted}{HTML}{999999}
\definecolor{xgray}{HTML}{777777}
\definecolor{xcyan}{HTML}{80F5DC}
\definecolor{xpink}{HTML}{f690ea}
\definecolor{xgraycyan}{HTML}{82bceb}

%-------------------------------------------------------------------------------
%	SECTION OPTIONS
%-------------------------------------------------------------------------------

%This sets the counter for how "deep" the numbering for the section, subsection, and subsubsection will go. 
%(Max is 5 - which goes to subparagraph)
\setcounter{secnumdepth}{3}
%This goes into when the numbering for the sections will start. 
%I.e. starting at 0 is equivalent numbering at 1 since it starts counting once it is called\
\setcounter{chapter}{0}
\setcounter{section}{0}
\setcounter{subsection}{0}
\setcounter{subsubsection}{0}

%These lines just set the section font size along with it being boldfaced.
\titleformat*{\section}
{\fontsize{18}{1em}\selectfont\bfseries}
\titleformat*{\subsection}
{\fontsize{16}{1em}\selectfont\bfseries}
\titleformat*{\subsubsection}
{\fontsize{14}{1em}\selectfont\bfseries}

%-------------------------------------------------------------------------------
%	COLUMN SECTIONS (Only used with 1.5cm margins)
%-------------------------------------------------------------------------------

%This breaks up the page into columns of 6cm and then 12 cm
%\setcolumnwidth{6cm, 12cm}  
\setlength{\columnsep}{5mm}
%This is the column separation distance
%\setlength{\columnseprule}{0.4pt} 
%This gives us the following vertical line.

%-------------------------------------------------------------------------------
%	ENVIRONMENTS AND TCOLORBOXES
%-------------------------------------------------------------------------------

%{Frame Color}{Width of Frame, range: [0, 1]}{Title}{Description}[label]
\DeclareTColorBox{titleBoxDefault}{ m m m m o }
{%
    breakable,
    pad at break* = 2mm,
    colframe = {#1},
    coltitle = white,
    colback = white,
    colbacktitle = {#1},
    width = #2 \linewidth,
    title = {#3: #4},
    enhanced,
    attach boxed title to top left={
            yshift=-0.2cm,
            xshift=0.4cm,
        },
    boxed title style={
            boxrule=0pt,
            colframe=white,
            arc=5pt,
            outer arc=4pt,
        },
    separator sign={~~},
    IfNoValueTF={#5}{}{ phantom = \phantomsection\hypertarget{#5} }
}

%{Frame Color}{Width of Frame, range: [0, 1]}[label]
\DeclareTColorBox{blockBoxDefault}{ m m o }
{%
    breakable,
    pad at break* = 2mm,
    colback = white,
    colframe = #1,
    width = #2 \linewidth,
    IfNoValueTF={#3}{}{ phantom = \phantomsection\hypertarget{#3} }
}

%Note that this is exactly the same as titleBoxDefault, but with better initial spacing. 
%s(*) - controls initial spacing 
%D<>{xgray} - controls frame color / Default option is xgray
%D(){1} - controls frame width, range: [0, 1] / Default option is 1
%m - controls the Title
%m - controls the Description
%o - controls the label
\NewDocumentEnvironment{titleBox}{ s D<>{xgray} D(){1} m m o }
{%
\IfBooleanTF{#1}{}{\baseSkip[0.5]}

\IfNoValueTF{#6}
{%
    \begin{titleBoxDefault}{#2}{#3}{#4}{#5}
}
{%
    \begin{titleBoxDefault}{#2}{#3}{#4}{#5}[#6]
        }
        }
        {%
    \end{titleBoxDefault}%
}

%Note that this is exactly the same as blockBoxDefault, but with better initial spacing. 
%s(*) - controls initial spacing 
%D<>{xgray} - controls frame color / Default option is xgray
%D(){1} - controls frame width, range: [0, 1] / Default option is 1
%o - controls the label
\NewDocumentEnvironment{blockBox}{ s D<>{xgray} D(){1} o }
{%
\IfBooleanTF{#1}{}{\baseSkip[0.5]}

\IfNoValueTF{#3}
{%
    \begin{blockBoxDefault}{#2}{#3}
}
{%
    \begin{blockBoxDefault}{#2}{#3}[#4]
        }
        }
        {%
    \end{blockBoxDefault}%
}

%s(*) - controls initial spacing 
%D(){1} - controls frame width, range: [0, 1] / Default option is 1
%m - controls the Description
%o - controls the label
\NewDocumentEnvironment{qBox}{ s D(){1} m o }
{%
\IfBooleanTF{#1}
{
\IfNoValueTF{#4}
{%
    \begin{titleBox}*(#2){Problem}{#3}
}
{%
    \begin{titleBox}*(#2){Problem}{#3}[#4]
}
}
{
\IfNoValueTF{#4}
{%
    \begin{titleBox}{Problem}{#3}
}
{%
    \begin{titleBox}{Problem}{#3}[#4]
        }
        }
        }
        {%
    \end{titleBox}%
}

%s(*) - controls initial spacing 
%D(){1} - controls frame width, range: [0, 1] / Default option is 1
%m - controls the Description
%o - controls the label
\NewDocumentEnvironment{defBox}{ s D(){1} m o }
{%
\IfBooleanTF{#1}
{
\IfNoValueTF{#4}
{%
    \begin{titleBox}*<xred>(#2){Definition}{#3}
}
{%
    \begin{titleBox}*<xred>(#2){Definition}{#3}[#4]
}
}
{
\IfNoValueTF{#4}
{%
    \begin{titleBox}<xred>(#2){Definition}{#3}
}
{%
    \begin{titleBox}<xred>(#2){Definition}{#3}[#4]
        }
        }
        }
        {%
    \end{titleBox}%
}

%s(*) - controls initial spacing 
%D(){1} - controls frame width, range: [0, 1] / Default option is 1
%O{Theorem} - controls Title / Default is Theorem
%m - controls the Description
%o - controls the label
\NewDocumentEnvironment{thmBox}{ s D(){1} O{Theorem} m o }
{%
\IfBooleanTF{#1}
{
\IfNoValueTF{#5}
{%
    \begin{titleBox}*<xblue>(#2){#3}{#4}
}
{%
    \begin{titleBox}*<xblue>(#2){#3}{#4}[#5]
}
}
{
\IfNoValueTF{#5}
{%
    \begin{titleBox}<xblue>(#2){#3}{#4}
}
{%
    \begin{titleBox}<xblue>(#2){#3}{#4}[#5]
        }
        }
        }
        {%
    \end{titleBox}%
}

%s(*) - controls initial spacing 
%D(){1} - controls frame width, range: [0, 1] / Default option is 1
%m - controls the Description
%o - controls the label
\NewDocumentEnvironment{egBox}{ s D(){1} m o }
{%
\IfBooleanTF{#1}
{
\IfNoValueTF{#4}
{%
    \begin{titleBox}*<xgreen>(#2){Example}{#3}
}
{%
    \begin{titleBox}*<xgreen>(#2){Example}{#3}[#4]
}
}
{
\IfNoValueTF{#4}
{%
    \begin{titleBox}<xgreen>(#2){Example}{#3}
}
{%
    \begin{titleBox}<xgreen>(#2){Example}{#3}[#4]
        }
        }
        }
        {%
    \end{titleBox}%
}

%s(*) - controls initial spacing 
%D(){1} - controls frame width, range: [0, 1] / Default option is 1
%O{Theorem} - controls Title / Default is Theorem
%m - controls the Description
%o - controls the label
\NewDocumentEnvironment{remarkBox}{ s D(){1} m o }
{%
\IfBooleanTF{#1}
{
\IfNoValueTF{#4}
{%
    \begin{titleBox}*(#2){Remark}{#3}
}
{%
    \begin{titleBox}*(#2){Remark}{#3}[#4]
}
}
{
\IfNoValueTF{#4}
{%
    \begin{titleBox}(#2){Remark}{#3}
}
{%
    \begin{titleBox}(#2){Remark}{#3}[#4]
        }
        }
        }
        {%
    \end{titleBox}%
}

%s(*) - controls initial spacing 
\NewDocumentEnvironment{proofBox}{ s }
{%
    \IfBooleanTF{#1}
    {
        \wrapBox{Proof}
        \baseSkip

    }
    {
        \wrapBox{Proof}
    }
}
{%
}

%This is used for sectioning off paragraph writing. Its only function is for tidying up the code.
\NewDocumentEnvironment{newParagraph}{}
{%
}
{%
}

%s(*) - controls initial spacing 
%m - controls the Description
\NewDocumentEnvironment{qPart}{ s m }
{%
    \textbf{\wrapBox{#2}}
}
{%
    \IfBooleanTF{#1}
    {%
    }
    {%

        \baseRule
        \baseSkip[0.2]
    }
}

%This environment simplifies the code for each section
%s(*) - controls spacing at the end
%D<>{6cm} - controls width of first column / Default is 6cm
%D(){12cm} - controls width of second column / Default is 12cm
%O{0} - controls which column we start at / Default is 0 (i.e first column)
\NewDocumentEnvironment{newColumn}{ s D<>{6cm} D(){12cm} O{0} }
{%
    \setcolumnwidth{#2, #3}
    %The 2 indicates how many columns that we have - the dimensions will be set via parameters
    \begin{paracol}{2}
        %This controls which column we start with first. Note that \switchcolumn* keeps the alignment when switching+
        \switchcolumn[#4]
        }
        {%
    \end{paracol}
    %This provides some space between the sections/topics via a horizontal line and vertical spacing.
    \IfBooleanTF{#1}{}{\baseRule}
}

%NOTE: \tcblower allows for you to draw a dotted horizontal line within each of the boxes. 
%Note that this command can only be used once.

%-------------------------------------------------------------------------------------------------------------------------------
%	COMMANDS
%-------------------------------------------------------------------------------------------------------------------------------

%This command allows for the user to input in a figure via inkscape. 
%Note that the inkscape file must be saved as .pdf and .pdf_tex. 
\NewDocumentCommand{\incfig}{ m O{1} }
{%
    \def\svgwidth{#2\linewidth}
    \import{./figures/}{#1.pdf_tex}
}

%This command creates a new centered, horizontal line with a length that I can choose. 
%Right now, it is set to be \linewidth
\NewDocumentCommand{\baseRule}{ O{1} }
{%
    \makebox[0pt][l]{\rule[0\baselineskip]{#1\linewidth}{.3mm}}%

    \baseSkip[0.5]%
}

%This provides the vertical spacing if needed%
\NewDocumentCommand{\baseSkip}{ s O{1} }
{%
    \IfBooleanTF{#1}
    {%
        \vspace*{#2\baselineskip}%
    }
    {%
        \vspace{#2\baselineskip}%
    }
}

\NewDocumentCommand{\baseIndent}{ s O{1} }
{%
    \IfBooleanTF{#1}
    {%
        \hspace*{#2 em}%
    }
    {%
        \hspace{#2 em}%
    }
}

\NewDocumentCommand{\wrapBox}{D<>{xgray} m}
{%
    \tcbox[size = small, colback = white,
        colframe = #1,
        on line]{#2}%
}